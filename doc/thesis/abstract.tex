Dynamically typed languages such as Lua avoid static types in favor of
simplicity and flexibility, because the absence of static types means
that programmers do not need to bother with abstracting types that
should be validated by a type checker.
In contrast, statically typed languages provide the early detection of
many bugs, and a better framework for structuring large programs.
These are two advantages of static typing that may lead programmers
to migrate from a dynamically typed to a statically typed language,
when their simple scripts evolve into complex programs.
Optional type systems allow combining dynamic and static typing in
the same language, without affecting its original semantics,
making easier this code evolution from dynamic to static typing.
Designing an optional type system for a dynamically typed language
is challenging, as it should feel natural to programmers that are
already familiar with this language.
In this work we present and formalize the design of Typed Lua,
an optional type system for Lua that introduces novel features
to statically type check some Lua idioms and features.
Even though Lua shares several characteristics with other dynamically
typed languages such as JavaScript, Lua also has several unusual features
that are not present in the type system of these languages.
These features include functions with flexible arity, multiple assignment,
functions that are overloaded on the number of return values, and the
incremental evolution of record and object types.
We discuss how Typed Lua handles these features and our design decisions.
Finally, we present the evaluation results that we achieved while using
Typed Lua to type existing Lua code.
