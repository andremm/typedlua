In this chapter we present statistics about the usage of Lua
features and idioms.
We collected statistics about how programmers use tables, functions,
dynamic type checking, object-oriented programming, and modules.
We shall see that these statistics informed important design decisions
on our optional type system.

We used the LuaRocks repository to build our statistics database;
LuaRocks \citep{hisham2013luarocks} is a package manager for Lua
modules.
We downloaded the 3928 \texttt{.lua} files that were available in
the LuaRocks repository at February 1st 2014.
However, we ignored files that are not compatible with Lua 5.2,
the latest version of Lua.
We also ignored \emph{machine-generated} files and test files,
because these files may not represent idiomatic Lua code,
and might skew our statistics towards non-typical uses of Lua.
This left 2598 \texttt{.lua} files from 262 different projects for
our statistics database;
we parsed these files and processed their abstract syntax tree
to gather the statistics that we show in this section.

To verify how programmers use tables, we measured how they
initialize, index, and iterate tables.
We present these statistics in the next three paragraphs to discuss
their influence on our type system.

The table constructor appears 23185 times.
In 36\% of the occurrences it is a constructor that initializes a
record (e.g., \texttt{\{ x = 120, y = 121 \}});
in 29\% of the occurrences it is a constructor that initializes a
list (e.g., \texttt{\{ "one", "two", "three", "four" \}});
in 8\% of the occurrences it is a constructor that initializes a
record with a list part;
and in less than 1\% of the occurrences (4 times) it is a constructor
that uses only the booleans \texttt{true} and \texttt{false} as indexes.
At all, in 73\% of the occurrences it is a constructor that uses
only literal keys;
in 26\% of the occurrences it is the empty constructor;
in 1\% of the occurrences it is a constructor with non-literal keys
only, that is, a constructor that uses variables and function calls
to create the indexes of a table;
and in less than 1\% of the occurrences (19 times) it is a constructor
that mixes literal keys and non-literal keys.

The indexing of tables appears 130448 times:
86\% of them are for reading a table field while
14\% of them are for writing into a table field.
We can classify the indexing operations that are reads as follows:
89\% of the reads use a literal string key,
4\% of the reads use a literal number key,
and less than 1\% of the reads (10 times) use a literal boolean key.
At all, 93\% of the reads use literals to index a table while
7\% of the reads use non-literal expressions to index a table.
It also worth mentioning that 45\% of the reads are actually
function calls.
More precisely, 25\% of the reads use literals to call a function,
20\% of the reads use literals to call a method, that is,
a function call that uses the colon syntactic sugar, 
and less than 1\% of the reads (195 times) use non-literal expressions
to call a function.
We can also classify the indexing operations that are writes as follows: 
69\% of the writes use a literal string key,
2\% of the writes use a literal number key,
and less than 1\% of the writes (1 time) uses a literal boolean key.
At all, 71\% of the writes use literals to index a table while
29\% of the writes use non-literal expressions to index a table.

We also measured how many files have code that iterate over tables to
observe how frequently iteration is used.
We observed that 23\% of the files have code that iterate over keys
of any value, that is, the call to \texttt{pairs} appears at least
once in these files (the median is twice per file);
21\% of the files have code that iterate over integer keys, that is,
the call to \texttt{ipairs} appears at least once in these files
(the median is also twice per file);
and 10\% of the files have code that use the numeric \texttt{for}
along with the length operator (the median is once per file).

The numbers about table initialization, indexing, and iteration
show us that tables are mostly used to represent records, lists,
and associative arrays.
Therefore, Typed Lua should include a table type for handling
these uses of Lua tables.
Even though the statistics show that programmers initialize tables
more often than they use the empty constructor to
dynamically initialize tables, the statistics of the empty
constructor are still expressive and indicate that Typed Lua should
also include a way to handle this style of defining table types.

We found a total of 24858 function declarations in our database
(the median is six per file).
Next, we discuss how frequently programmers use dynamic type
checking and multiple return values inside these functions.

We observed that 9\% of the functions perform dynamic type checking
on their input parameters, that is, these functions use \texttt{type}
to inspect the tags of Lua values (the median is once per function).
We randomly selected 20 functions to sample how programmers are
using \texttt{type}, and we got the following data:
50\% of these functions use \texttt{type} for asserting the tags of
their input parameters, that is, they raise an error when the tag of a
certain parameter does not match the expected tag, and
50\% of these functions use \texttt{type} for overloading, that is,
they execute different code according to the inspected tag.

These numbers show us that Typed Lua should include union types,
because the use of the \texttt{type} idiom shows that disjoint unions
will help programmers define data structures that can hold a value of
several different, but fixed types.
Typed Lua should also use \texttt{type} as a mechanism for decomposing
unions, though it may be restricted to base types only.

We observed that 10\% of the functions explicitly return multiple
values.
We also observed that 5\% of the functions return \texttt{nil} plus
something else, for signaling an unexpected behavior;
and 1\% of the functions return \texttt{false} plus something else,
also for signaling an unexpected behavior.

Typed Lua should include function types to represent Lua functions,
and tuple types to represent the signatures of Lua functions,
multiple return values, and multiple assignments.
Tuple types will require some special attention, because Typed Lua
should be able to adjust tuple types during compile-time, in a
similar way to what Lua does with function calls and multiple
assignments during run-time.
In addition, the number of functions that return \texttt{nil} and
\texttt{false} plus something else show us that overloading on the
return type will be useful to the type system.

We also measured how frequently programmers use the object-oriented
paradigm in Lua.
We observed that 23\% of the function declarations are actually
method declarations.
More precisely, 14\% of them use the colon syntactic sugar while
9\% of them use \texttt{self} as their first parameter.
We also observed that 63\% of the projects extend tables with
metatables, that is, they call \texttt{setmetatable} at least once,
and 27\% of the projects access the metatable of a given table,
that is, they call \texttt{getmetatable} at least once.
In fact, 45\% of the projects extend tables with metatables and
declare methods:
13\% using the colon syntactic sugar, 14\% using \texttt{self}, and
18\% using both.

Based on these observations, Typed Lua should include interfaces and
classes as syntactic sugar for table types that encode objects.
Proposing an optional type system for Lua that includes classes may
sound unusual at first glance, because Lua does not have standard
policies for object-oriented programming.
However, Lua provides mechanisms that allow programmers to abstract
their code in terms of objects, and our statistics confirm that
an expressive number of programmers are relying on these mechanisms to
use the object-oriented paradigm in Lua.
Typed Lua should include interfaces and classes as merely syntactic
mechanisms that help the compiler type check object-oriented code,
and they should not influence the semantics of Lua.

We also measured how programmers are defining modules.
We observed that 38\% of the files use the current way of defining
modules, that is, these files return a table that contains the
exported members of the module at the end of the file;
22\% of the files still use the deprecated way of defining modules,
that is, these files call the function \texttt{module};
and 1\% of the files use both ways.
At all, 61\% of the files are modules while 39\% of the files are
plain scripts.
The number of plain scripts is high considering the origin of
our database.
However, we did not ignore sample scripts, which usually serve to
help the users of a given module on how to use this module, and
that is the reason why we have a high number of plain scripts.

Based on these observations, Typed Lua should also use interfaces as
syntactic sugar for table types that define modules.
Typed Lua should also support the deprecated style of module
definition, using global names as exported members of the module.

Besides using interfaces for encoding objects and defining modules,
Typed Lua should also use interfaces for defining the types of
userdata, as programmers often use userdata in an object-oriented
style.

The last statistics that we collected were about variadic functions
and vararg expressions.
We observed that 8\% of the functions are variadic, that is,
their last parameter is the vararg expression.
We also observed that 5\% of the initialization of lists
(or 2\% of the occurrences of the table constructor) use solely the
vararg expression.
Typed Lua should include a \emph{vararg type} to handle variadic
functions and vararg expressions.


