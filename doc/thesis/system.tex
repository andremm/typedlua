
In the previous chapter we presented an informal overview of Typed Lua.
We showed that programmers can use Typed Lua to combine static and dynamic
typing in the same code.

In this chapter we present the formalization of Typed Lua.
We use typing rules to show how we implemented the features that we
discussed in the previous chapter.

\section{Types}

\begin{figure}[!ht]
\textbf{Type Language}\\
\dstart
$$
\begin{array}{rlr}
\multicolumn{3}{c}{\textbf{First-level types}}\\
t ::= & \;\; l & \textit{literal types}\\
& | \; b & \textit{base types}\\
& | \; \Nil & \textit{nil type}\\
& | \; \Value & \textit{top type}\\
& | \; \Any & \textit{dynamic type}\\
& | \; \Self & \textit{self type}\\
& | \; t \cup t & \textit{disjoint union types}\\
& | \; p \rightarrow r & \textit{function types}\\
& | \; \{f_{i}, ..., f_{n}\}_{c|o|u} & \textit{table types}\\
& | \; x & \textit{type variables}\\
& | \; \mu x.t & \textit{recursive types}\\
& | \; x_{i} & \textit{projection types}\\
l ::= & \False \; | \; \True \; | \; {\it number} \; | \; {\it string} & \\
b ::= & \Boolean \; | \; \Number \; | \; \String & \\
f ::= & k:t \; | \; \Const \; k:t & \textit{field types}\\ 
k ::= & l \; | \; b \; | \; \Any & \textit{key types}\\
\multicolumn{3}{c}{}\\
\multicolumn{3}{c}{\textbf{Second-level types}}\\
s ::= & \;\; t &\\
& | \; t* & \textit{variadic types}\\
& | \; t \times s & \textit{tuple types}\\
p ::= & \Void \; | \; s & \textit{parameters list type}\\
r ::= & \Void \; | \; s \; | \; s \cup r & \textit{return types}\\
\end{array}
$$
\dend
\caption{The abstract syntax of Typed Lua types}
\label{fig:typelang}
\end{figure}

Figure \ref{fig:typelang} presents the abstract syntax of the
Typed Lua types.
Typed Lua splits types into two categories:
\emph{first-level types} and \emph{second-level types}.
Typed Lua uses first-level types to represent Lua values and
second-level types to represent expression lists, and it also
uses them to type multiple assignments and function applications.
First-level types include literal types, base types, the type $\Nil$,
the top type $\Value$, the dynamic type $\Any$, the type $\Self$,
union types, function types, table types, recursive types, and
projection types.
Second-level types include the type $\Void$, tuples of first-level
types optionally ending in a variadic type, and union of these tuples.

\begin{figure}[!ht]
\textbf{Subtyping}\\
\dstart
\begin{footnotesize}
$$
\begin{array}{c}
\begin{array}{c}
\mylabel{S-LITERAL}\\
\senv \vdash L \subtype L
\end{array}
\;
\begin{array}{c}
\mylabel{S-FALSE}\\
\senv \vdash \False \subtype \Boolean
\end{array}
\;
\begin{array}{c}
\mylabel{S-TRUE}\\
\senv \vdash \True \subtype \Boolean
\end{array}
\\ \\
\begin{array}{c}
\mylabel{S-NUMBER}\\
\senv \vdash {\it number} \subtype \Number
\end{array}
\;
\begin{array}{c}
\mylabel{S-STRING}\\
\senv \vdash {\it string} \subtype \String
\end{array}
\;
\begin{array}{c}
\mylabel{S-BASE}\\
\senv \vdash B \subtype B
\end{array}
\\ \\
\begin{array}{c}
\mylabel{S-NIL}\\
\senv \vdash \Nil \subtype \Nil
\end{array}
\;
\begin{array}{c}
\mylabel{S-VALUE}\\
\senv \vdash T \subtype \Value
\end{array}
\;
\begin{array}{c}
\mylabel{S-ANY}\\
\senv \vdash \Any \subtype \Any
\end{array}
\;
\begin{array}{c}
\mylabel{S-SELF}\\
\senv \vdash \Self \subtype \Self
\end{array}
\\ \\
\begin{array}{c}
\mylabel{S-UNION1}\\
\dfrac{\senv \vdash T_{1} \subtype T \;\;\;
       \senv \vdash T_{2} \subtype T}
      {\senv \vdash T_{1} \cup T_{2} \subtype T}
\end{array}
\;
\begin{array}{c}
\mylabel{S-UNION2}\\
\dfrac{\senv \vdash T \subtype T_{1}}
      {\senv \vdash T \subtype T_{1} \cup T_{2}}
\end{array}
\;
\begin{array}{c}
\mylabel{S-UNION3}\\
\dfrac{\senv \vdash T \subtype T_{2}}
      {\senv \vdash T \subtype T_{1} \cup T_{2}}
\end{array}
\\ \\
\begin{array}{c}
\mylabel{S-FUNCTION}\\
\dfrac{\senv \vdash P_{2} \subtype P_{1} \;\;\;
       \senv \vdash R_{1} \subtype R_{2}}
      {\senv \vdash P_{1} \rightarrow R_{1} \subtype P_{2} \rightarrow R_{2}}
\end{array}
\;
\begin{array}{c}
\mylabel{S-TABLE1}\\
\dfrac{\forall i \in 1..n \; \exists j \in 1..m \;\;\;
       \senv \vdash F_{j} \subtype F_{i}}
      {\senv \vdash \{F_{1}, ..., F_{m}\}_{c} \subtype \{F_{1}, ..., F_{n}\}}
\end{array}
\\ \\
\begin{array}{c}
\mylabel{S-TABLE2}\\
\dfrac{\forall i \in 1..m \; \senv \vdash F_{i}' \subtype F_{i}}
      {\senv \vdash \{F_{1}, ..., F_{m}\}_{o} \subtype \{F_{1}', ..., F_{m}', ..., F_{n}'\}}
\end{array}
\;
\begin{array}{c}
\mylabel{S-TABLE3}\\
\dfrac{\forall i \in 1..m \; \senv \vdash F \subtype F_{i}}
      {\senv \vdash \{F_{1}, ..., F_{m}\}_{o} \subtype \{F\}} 
\end{array}
\\ \\
\begin{array}{c}
\mylabel{S-FIELD1}\\
\dfrac{\senv \vdash K_{2} \subtype K_{1} \;\;\;
       \senv \vdash T_{1} \subtype T_{2} \;\;\;
       \senv \vdash T_{2} \subtype T_{1}}
      {\senv \vdash K_{1}:T_{1} \subtype K_{2}:T_{2}}
\end{array}
\;
\begin{array}{c}
\mylabel{S-FIELD2}\\
\dfrac{\senv \vdash K_{2} \subtype K_{1} \;\;\;
       \senv \vdash T_{1} \subtype T_{2}}
      {\senv \vdash \Const \; K_{1}:T_{1} \subtype \Const \; K_{2}:T_{2}}
\end{array}
\\ \\
\begin{array}{c}
\mylabel{S-FIELD3}\\
\dfrac{\senv \vdash K_{2} \subtype K_{1} \;\;\;
       \senv \vdash T_{1} \subtype T_{2}}
      {\senv \vdash K_{1}:T_{1} \subtype \Const \; K_{2}:T_{2}}
\end{array}
\;
\begin{array}{c}
\mylabel{S-VARIABLE}\\
\dfrac{X_{1} \subtype X_{2} \in \senv}
      {\senv \vdash X_{1} \subtype X_{2}}
\end{array}
\;
\begin{array}{c}
\mylabel{S-RECURSIVE1}\\
\dfrac{\senv, X_{1} \subtype X_{2} \vdash T_{1} \subtype T_{2}}
      {\senv \vdash \mu X_{1}.T_{1} \subtype \mu X_{2}.T_{2}}
\end{array}
\\ \\
\begin{array}{c}
\mylabel{S-RECURSIVE2}\\
\dfrac{X \subtype X \not\in \senv \;\;\;
      \senv, X \subtype X \vdash T_{1} \subtype T_{2}}
      {\senv \vdash \mu X.T_{1} \subtype T_{2}}
\end{array}
\;
\begin{array}{c}
\mylabel{S-RECURSIVE3}\\
\dfrac{X \subtype X \in \senv \;\;\;
       \senv \vdash X \subtype T_{2}}
      {\senv \vdash \mu X.T_{1} \subtype T_{2}}
\end{array}
\\ \\
\begin{array}{c}
\mylabel{S-RECURSIVE4}\\
\dfrac{X \subtype X \not\in \senv \;\;\;
       \senv, X \subtype X \vdash T_{1} \subtype T_{2}}
      {\senv \vdash T_{1} \subtype \mu X.T_{2}}
\end{array}
\;
\begin{array}{c}
\mylabel{S-RECURSIVE5}\\
\dfrac{X \subtype X \in \senv \;\;\;
       \senv \vdash T_{1} \subtype X}
      {\senv \vdash T_{1} \subtype \mu X.T_{2}}
\end{array}
\\ \\
\begin{array}{c}
\mylabel{S-VOID}\\
\senv \vdash \Void \subtype \Void
\end{array}
\;
\begin{array}{c}
\mylabel{S-VARARG}\\
\dfrac{\senv \vdash T_{1} \cup \Nil \subtype T_{2} \cup \Nil}
      {\senv \vdash T_{1}* \subtype T_{2}*}
\end{array}
\;
\begin{array}{c}
\mylabel{S-TUPLE1}\\
\dfrac{\senv \vdash T_{1} \subtype T_{2} \;\;\;
       \senv \vdash S_{1} \subtype S_{2}}
      {\senv \vdash T_{1} \times S_{1} \subtype T_{2} \times S_{2}}
\end{array}
\\ \\
\begin{array}{c}
\mylabel{S-TUPLE2}\\
\dfrac{\senv \vdash T_{1} \cup \Nil \subtype T_{2} \;\;\;
       \senv \vdash T_{1}* \subtype S_{2}}
      {\senv \vdash T_{1}* \subtype T_{2} \times S_{2}}
\end{array}
\;
\begin{array}{c}
\mylabel{S-TUPLE3}\\
\dfrac{\senv \vdash T_{1} \subtype T_{2} \cup \Nil \;\;\;
       \senv \vdash S_{1} \subtype T_{2}*}
      {\senv \vdash T_{1} \times S_{1} \subtype T_{2}*}
\end{array}
\end{array}
$$
\end{footnotesize}
\dend
\caption{The subtyping relation}
\label{fig:subtyping}
\end{figure}

\begin{figure}[!ht]
\textbf{Consistent-Subtyping}\\
\dstart
\begin{footnotesize}
$$
\begin{array}{c}
\begin{array}{c}
\mylabel{C-LITERAL}\\
\senv \vdash L \lesssim L
\end{array}
\;
\begin{array}{c}
\mylabel{C-FALSE}\\
\senv \vdash \False \lesssim \Boolean
\end{array}
\;
\begin{array}{c}
\mylabel{C-TRUE}\\
\senv \vdash \True \lesssim \Boolean
\end{array}
\\ \\
\begin{array}{c}
\mylabel{C-NUMBER}\\
\senv \vdash {\it number} \lesssim \Number
\end{array}
\;
\begin{array}{c}
\mylabel{C-STRING}\\
\senv \vdash {\it string} \lesssim \String
\end{array}
\;
\begin{array}{c}
\mylabel{C-BASE}\\
\senv \vdash B \lesssim B
\end{array}
\\ \\
\begin{array}{c}
\mylabel{C-NIL}\\
\senv \vdash \Nil \lesssim \Nil
\end{array}
\;
\begin{array}{c}
\mylabel{C-VALUE}\\
\senv \vdash T \lesssim \Value
\end{array}
\;
\begin{array}{c}
\mylabel{C-ANY1}\\
\senv \vdash T \lesssim \Any
\end{array}
\;
\begin{array}{c}
\mylabel{C-ANY2}\\
\senv \vdash \Any \lesssim T
\end{array}
\;
\begin{array}{c}
\mylabel{C-SELF}\\
\senv \vdash \Self \lesssim \Self
\end{array}
\\ \\
\begin{array}{c}
\mylabel{C-UNION1}\\
\dfrac{\senv \vdash T_{1} \lesssim T \;\;\;
       \senv \vdash T_{2} \lesssim T}
      {\senv \vdash T_{1} \cup T_{2} \lesssim T}
\end{array}
\;
\begin{array}{c}
\mylabel{C-UNION2}\\
\dfrac{\senv \vdash T \lesssim T_{1}}
      {\senv \vdash T \lesssim T_{1} \cup T_{2}}
\end{array}
\;
\begin{array}{c}
\mylabel{C-UNION3}\\
\dfrac{\senv \vdash T \lesssim T_{2}}
      {\senv \vdash T \lesssim T_{1} \cup T_{2}}
\end{array}
\\ \\
\begin{array}{c}
\mylabel{C-FUNCTION}\\
\dfrac{\senv \vdash P_{2} \lesssim P_{1} \;\;\;
       \senv \vdash R_{1} \lesssim R_{2}}
      {\senv \vdash P_{1} \rightarrow R_{1} \lesssim P_{2} \rightarrow R_{2}}
\end{array}
\;
\begin{array}{c}
\mylabel{C-TABLE1}\\
\dfrac{\forall i \in 1..n \; \exists j \in 1..m \;\;\;
       \senv \vdash F_{j} \lesssim F_{i}}
      {\senv \vdash \{F_{1}, ..., F_{m}\}_{c} \lesssim \{F_{1}, ..., F_{n}\}}
\end{array}
\\ \\
\begin{array}{c}
\mylabel{C-TABLE2}\\
\dfrac{\forall i \in 1..m \; \senv \vdash F_{i}' \lesssim F_{i}}
      {\senv \vdash \{F_{1}, ..., F_{m}\}_{o} \lesssim \{F_{1}', ..., F_{m}', ..., F_{n}'\}}
\end{array}
\;
\begin{array}{c}
\mylabel{C-TABLE3}\\
\dfrac{\forall i \in 1..m \; \senv \vdash F \lesssim F_{i}}
      {\senv \vdash \{F_{1}, ..., F_{m}\}_{o} \lesssim \{F\}} 
\end{array}
\\ \\
\begin{array}{c}
\mylabel{C-FIELD1}\\
\dfrac{\senv \vdash K_{2} \lesssim K_{1} \;\;\;
       \senv \vdash T_{1} \lesssim T_{2} \;\;\;
       \senv \vdash T_{2} \lesssim T_{1}}
      {\senv \vdash K_{1}:T_{1} \lesssim K_{2}:T_{2}}
\end{array}
\;
\begin{array}{c}
\mylabel{C-FIELD2}\\
\dfrac{\senv \vdash K_{2} \lesssim K_{1} \;\;\;
       \senv \vdash T_{1} \lesssim T_{2}}
      {\senv \vdash \Const \; K_{1}:T_{1} \lesssim \Const \; K_{2}:T_{2}}
\end{array}
\\ \\
\begin{array}{c}
\mylabel{C-FIELD3}\\
\dfrac{\senv \vdash K_{2} \lesssim K_{1} \;\;\;
       \senv \vdash T_{1} \lesssim T_{2}}
      {\senv \vdash K_{1}:T_{1} \lesssim \Const \; K_{2}:T_{2}}
\end{array}
\;
\begin{array}{c}
\mylabel{C-VARIABLE}\\
\dfrac{X_{1} \lesssim X_{2} \in \senv}
      {\senv \vdash X_{1} \lesssim X_{2}}
\end{array}
\;
\begin{array}{c}
\mylabel{C-RECURSIVE1}\\
\dfrac{\senv, X_{1} \lesssim X_{2} \vdash T_{1} \lesssim T_{2}}
      {\senv \vdash \mu X_{1}.T_{1} \lesssim \mu X_{2}.T_{2}}
\end{array}
\\ \\
\begin{array}{c}
\mylabel{C-RECURSIVE2}\\
\dfrac{X \lesssim X \not\in \senv \;\;\;
      \senv, X \lesssim X \vdash T_{1} \lesssim T_{2}}
      {\senv \vdash \mu X.T_{1} \lesssim T_{2}}
\end{array}
\;
\begin{array}{c}
\mylabel{C-RECURSIVE3}\\
\dfrac{X \lesssim X \in \senv \;\;\;
       \senv \vdash X \lesssim T_{2}}
      {\senv \vdash \mu X.T_{1} \lesssim T_{2}}
\end{array}
\\ \\
\begin{array}{c}
\mylabel{C-RECURSIVE4}\\
\dfrac{X \lesssim X \not\in \senv \;\;\;
       \senv, X \lesssim X \vdash T_{1} \lesssim T_{2}}
      {\senv \vdash T_{1} \lesssim \mu X.T_{2}}
\end{array}
\;
\begin{array}{c}
\mylabel{C-RECURSIVE5}\\
\dfrac{X \lesssim X \in \senv \;\;\;
       \senv \vdash T_{1} \lesssim X}
      {\senv \vdash T_{1} \lesssim \mu X.T_{2}}
\end{array}
\\ \\
\begin{array}{c}
\mylabel{C-VOID}\\
\senv \vdash \Void \lesssim \Void
\end{array}
\;
\begin{array}{c}
\mylabel{C-VARARG}\\
\dfrac{\senv \vdash T_{1} \cup \Nil \lesssim T_{2} \cup \Nil}
      {\senv \vdash T_{1}* \lesssim T_{2}*}
\end{array}
\;
\begin{array}{c}
\mylabel{C-TUPLE1}\\
\dfrac{\senv \vdash T_{1} \lesssim T_{2} \;\;\;
       \senv \vdash S_{1} \lesssim S_{2}}
      {\senv \vdash T_{1} \times S_{1} \lesssim T_{2} \times S_{2}}
\end{array}
\\ \\
\begin{array}{c}
\mylabel{C-TUPLE2}\\
\dfrac{\senv \vdash T_{1} \cup \Nil \lesssim T_{2} \;\;\;
       \senv \vdash T_{1}* \lesssim S_{2}}
      {\senv \vdash T_{1}* \lesssim T_{2} \times S_{2}}
\end{array}
\;
\begin{array}{c}
\mylabel{C-TUPLE3}\\
\dfrac{\senv \vdash T_{1} \lesssim T_{2} \cup \Nil \;\;\;
       \senv \vdash S_{1} \lesssim T_{2}*}
      {\senv \vdash T_{1} \times S_{1} \lesssim T_{2}*}
\end{array}
\end{array}
$$
\end{footnotesize}
\dend
\caption{The consistent-subtyping relation}
\label{fig:consistent_subtyping}
\end{figure}

\section{Typing rules}

\begin{figure}[!ht]
\textbf{Abstract Syntax}\\
\dstart
$$
\begin{array}{rl}
s ::= & \;\; \mathbf{skip} \; | \;
s_{1} \; ; \; s_{2} \; | \;
\vec{l} = \vec{e}  \; | \;
\mathbf{while} \; e \; \mathbf{do} \; s \; | \;
\mathbf{if} \; e \; \mathbf{then} \; s_{1} \; \mathbf{else} \; s_{2}\\
& | \; \mathbf{local} \; \vec{n{:}t} = \vec{e} \; \mathbf{in} \; s \; | \;
\mathbf{rec} \; n{:}t = f \; \mathbf{in} \; s \; | \;
\mathbf{return} \; \vec{e} \; | \;
a\\
e ::= & \;\; \mathbf{nil} \; | \;
\mathbf{false} \; | \;
\mathbf{true} \; | \;
{\it number} \; | \;
{\it string} \; | \;
{...} \; | \;
f \; | \;
\{ \; \vec{c} \; \} \\
& | \; e_{1} + e_{2} \; | \;
e_{1} \; {..} \; e_{2} \; | \;
e_{1} == e_{2} \; | \;
e_{1} < e_{2} \; | \;
e_{1} \; \mathbf{and} \; e_{2} \; | \;
e_{1} \; \mathbf{or} \; e_{2} \\
& | \; \mathbf{not} \; e \; | \;
- e \; | \;
\# \; e \; | \;
a \; | \;
l \; | \;
(e) \; | \;
{<}t{>} \; e\\
l ::= & \; n \; | \;
e_{1}[e_{2}]\\
a ::= & \; e_{1}(\vec{e_{2}}) \; | \;
e_{1}{:}n(\vec{e_{2}})\\
f ::= & \; \mathbf{fun} \; (){:}r \; s \; | \;
\mathbf{fun} \; ({...}{:}t){:}r \; s \; | \;
\mathbf{fun} \; (\vec{n{:}t}){:}r \; s \; | \;
\mathbf{fun} \; (\vec{n{:}t},{...}{:}t){:}r \; s\\
c ::= & \; [e_{1}] = e_{2}\\
n ::= & \; {\it name}\\
\end{array}
$$
\dend
\caption{Typed Lua abstract syntax}
\label{fig:syntax}
\end{figure}

Rule for checking if a table type is well formed:
\[
\forall i \not\exists j \; i \not= j \wedge K_{i} \lesssim K_{j}
\]


