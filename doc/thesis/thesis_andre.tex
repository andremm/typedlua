\documentclass[phd,oneside,british]{ThesisPUC_uk}

\usepackage[utf8]{inputenc}
\usepackage{amsmath}
\usepackage{amssymb}
\usepackage{url}
\usepackage{color}
\usepackage{multirow}

\newcommand{\Value}{\mathbf{value}}
\newcommand{\Any}{\mathbf{any}}
\newcommand{\Nil}{\mathbf{nil}}
\newcommand{\Self}{\mathbf{self}}
\newcommand{\False}{\mathbf{false}}
\newcommand{\True}{\mathbf{true}}
\newcommand{\Boolean}{\mathbf{boolean}}
\newcommand{\Integer}{\mathbf{integer}}
\newcommand{\Number}{\mathbf{number}}
\newcommand{\String}{\mathbf{string}}
\newcommand{\Void}{\mathbf{void}}
\newcommand{\Const}{\mathbf{const}}

\newcommand{\mylabel}[1]{\; (\textsc{#1})}
\newcommand{\env}{\Gamma}
\newcommand{\penv}{\Pi}
\newcommand{\senv}{\Sigma}
\newcommand{\subtype}{<:}
\newcommand{\ret}{\rho}
\newcommand{\self}{\sigma}

\newcommand{\minitab}[2][c]{\begin{tabular}{#1}#2\end{tabular}}

\def\dstart{\hbox to \hsize{\vrule depth 4pt\hrulefill\vrule depth 4pt}}
\def\dend{\hbox to \hsize{\vrule height 4pt\hrulefill\vrule height 4pt}}

\author{André Murbach Maidl}
\authorR{Maidl, André Murbach}
\adviser{Roberto Ierusalimschy}
\adviserR{Ierusalimschy, Roberto}
\coadviser{Fabio Mascarenhas de Queiroz}
\coadviserR{Queiroz, Fabio Mascarenhas de}
\coadviserInst{UFRJ}

\title{Typed Lua: An Optional Type System for Lua}
\titlebr{Typed Lua: um sistema de tipos opcional para Lua}

\day{10th} \month{April} \year{2015}

\city{Rio de Janeiro}
\CDD{004}
\department{Informática}
\program{Informática}
\school{Centro Técnico Científico}
\university{Pontifícia Universidade Católica do Rio de Janeiro}
\uni{PUC--Rio}

\jury {
\jurymember{Ana Lúcia de Moura}
           {Departamento de Informática --- PUC-Rio}
\jurymember{Edward Hermann Haeusler}
           {Departamento de Informática --- PUC-Rio}
\jurymember{Anamaria Martins Moreira}
           {UFRJ}
\jurymember{Roberto da Silva Bigonha}
           {UFMG}
\schoolhead{José Eugênio Leal}
}

\resume{
The author graduated in Computer Science from Pontifícia Universidade
Católica do Paraná --- PUCPR in 2004, and obtained the degree of Mestre at
Universidade Federal do Paraná --- UFPR in 2007.
He obtained the degree of Doutor at PUC-Rio in 2015,
where he worked in the field of programming languages.
}

\acknowledgment{
First of all I would like to thank three indispensable people in my life:
my wife, my mother, and my father.
Izabella is my inexhaustible source of inspiration, and she is
always pushing me forward.
My mother is a very brave person, and she inspires me to never give up.
My father is a very kind person, and he inspires me to try being more sensitive.
I am thankful to them not only because the support they gave me while
writing this thesis, but also because they are the reason of my life.

This work would have not happened without all the help from my advisors.
Roberto has a singular view of computer science that pushed me to
pursuit the best I could do, while Fabio gave me the confidence I
needed about the work we were doing.
I am glad that both of them had patience to guide me through this
work, and for all the nice meetings we had that led me to learn
a lot of new concepts and ideas.

It is hard to complete a thesis without having some leisure moments,
so I thank my friend and roommate André Ramiro for all the crazy and
fun moments that we had, which helped me to keep focused during working hours.

I also thank my friends at LabLua for the nice work environment,
specially Hisham and Pablo for all the support during the
hardest moments.

It may look easy moving to Rio, as it is a sunny place with lots
of things to do, but it may not be as easy as it looks for
someone that grew up in a cloudy and rainy town.
Thus, I thank my cousin Rebecca and my aunt Nazira for all the help
and affection they gave me during my adaptation in Rio, which
were fundamental to keep my goals in mind.

I thank the guys from Rio de Janeiro Fixed Gear (RJFG) for all
the fun we had during the night rides, and for helping me to discover Rio.

I am also most grateful to the professors Ana Lúcia de Moura,
Anamaria Martins Moreira, Edward Hermann Haeusler, and
Roberto da Silva Bigonha for their careful review that helped
to improve this work.

Finally, I would like to thank CAPES, Google Summer of Code, and PUC-Rio
for partially funding this work.

}

\abstract{
Dynamically typed languages such as Lua avoid static types in favor of
simplicity and flexibility, because the absence of static types means
that programmers do not need to bother with abstracting types that
should be validated by a type checker.
In contrast, statically typed languages provide the early detection of
many bugs, and a better framework for structuring large programs.
These are two advantages of static typing that may lead programmers
to migrate from a dynamically typed to a statically typed language,
when their simple scripts evolve into complex programs.
Optional type systems allow combining dynamic and static typing in
the same language, without affecting its original semantics,
making easier this code evolution from dynamic to static typing.
Designing an optional type system for a dynamically typed language
is challenging, as it should feel natural to programmers that are
already familiar with this language.
In this work we present and formalize the design of Typed Lua,
an optional type system for Lua that introduces novel features
to statically type check some Lua idioms and features.
Even though Lua shares several characteristics with other dynamically
typed languages such as JavaScript, Lua also has several unusual features
that are not present in the type system of these languages.
These features include functions with flexible arity, multiple assignment,
functions that are overloaded on the number of return values, and the
incremental evolution of record and object types.
We discuss how Typed Lua handles these features and our design decisions.
Finally, we present the evaluation results that we achieved while using
Typed Lua to type existing Lua code.

}

\keywords{
\key{Scripting languages}
\key{Lua}
\key{Type systems}
\key{Optional type systems}
\lastkey{Gradual typing}
}

\abstractbr{
Linguagens dinamicamente tipadas, tais como Lua, não usam tipos estáticos em
favor de simplicidade e flexibilidade, porque a ausência de tipos estáticos
significa que programadores não precisam se preocupar em abstrair tipos que
devem ser validados por um verificador de tipos.
Por outro lado, linguagens estaticamente tipadas ajudam na detecção prévia de
diversos \emph{bugs} e também ajudam na estruturação de programas grandes.
Tais pontos geralmente são vistos como duas vantagens que levam programadores
a migrar de uma linguagem dinamicamente tipada para uma linguagem estaticamente tipada,
quando os pequenos \emph{scripts} deles evoluem para programas complexos.
Sistemas de tipos opcionais nos permitem combinar tipagem dinâmica e estática na
mesma linguagem, sem afetar a semântica original da linguagem, tornando mais
fácil a evolução de código tipado dinamicamente para código tipado estaticamente.
Desenvolver um sistema de tipos opcional para uma linguagem dinamicamente tipada é
uma tarefa desafiadora, pois ele deve ser o mais natural possível para os programadores
que já estão familiarizados com essa linguagem.
Neste trabalho nós apresentamos e formalizamos Typed Lua, um sistema de tipos opcional
para Lua, o qual introduz novas características para tipar estaticamente alguns idiomas
e características de Lua.
Embora Lua compartilhe várias características com outras linguagens dinamicamente
tipadas, em particular JavaScript, Lua também possui várias características não usuais,
as quais não estão presentes nos sistemas de tipos dessas linguagens.
Essas características incluem funções com aridade flexível, atribuições múltiplas,
funções que são sobrecarregadas no número de valores de retorno e
a evolução incremental de registros e objetos.
Nós discutimos como Typed Lua tipa estaticamente essas características e
também discutimos nossas decisões de projeto.
Finalmente, apresentamos uma avaliação de resultados,
a qual obtivemos ao usar Typed Lua para tipar código Lua existente.

}

\keywordsbr{
\key{Linguagens de script}
\key{Lua}
\key{Sistemas de tipos}
\key{Sistemas de tipos opcionais}
\lastkey{Tipagem gradual}
}

\tablesmode{figtab}

\begin{document}

\chapter{Introduction}
\label{chap:intro}
Dynamically typed languages such as Lua avoid static types in favor of
simplicity and flexibility, because the absence of static types means
that programmers do not need to bother with abstracting types that
should be validated by a type checker.
Instead, dynamically typed languages use run-time \emph{type tags}
to classify the values they compute, so their implementation can use
these tags to perform run-time (or dynamic) type checking
\cite{pierce2002tpl}.

This simplicity and flexibility allows programmers to write code that
might require a complex type system to statically type check,
though it may also hide bugs that will be caught only after deployment
if programmers do not properly test their code.
In contrast, static type checking helps programmers detect many
bugs during the development phase.
Static types also provide a conceptual framework that helps
programmers define modules and interfaces that can be combined to
structure the development of programs.

Thus, early error detection and better program structure are two
advantages of static type checking that can lead programmers to
migrate their code from a dynamically typed to a statically
typed language, when their simple scripts evolve into complex programs
\cite{tobin-hochstadt2006ims}.
Dynamically typed languages certainly help programmers during the
beginning of a project, because their simplicity and flexibility
allows quick development and makes it easier to change code according to
changing requirements.
However, programmers tend to migrate from dynamically typed to
statically typed code as soon as the project has consolidated its
requirements, because the robustness of static types helps
programmers link requirements to abstractions.
This migration usually involves different languages that have
different syntaxes and semantics, which usually requires a complete
rewrite of existing programs instead of incremental evolution from
dynamic to static types.

Ideally, programming languages should offer programmers the
option to choose between static and dynamic typing:
\emph{optional type systems} \cite{bracha2004pluggable} and
\emph{gradual typing} \cite{siek2006gradual} are two similar
approaches for blending static and dynamic typing in the same
language.
The aim of both approaches is to offer programmers the option
to use type annotations where static typing is needed,
allowing the incremental migration from dynamic to static typing.
The difference between these two approaches is the way they treat
run-time semantics.
While optional type systems do not affect run-time semantics,
gradual typing uses run-time checks to ensure that dynamically typed
code does not violate the invariants of statically typed code.

Programmers and researchers sometimes use the term \emph{gradual typing}
to mean the incremental evolution of dynamically typed code into
statically typed code.
For this reason, gradual typing may also refer to optional type
systems and other approaches that blend static and dynamic typing to
help programmers incrementally migrate from dynamic to static typing
without having to switch to a different language, though all these
approaches differ in the way they handle static and dynamic typing
together.
We use the term \emph{gradual typing} to refer to the work of
Siek and Taha \cite{siek2006gradual}.

In this work we present the design and evaluation of Typed Lua:
an optional type system for Lua that is rich enough to
preserve some of the Lua idioms that programmers are already familiar with,
but that also includes new constructs that help programmers
structure Lua programs.

Lua is a small imperative language with first-class functions
(with proper lexical scoping) where the only data structure
mechanism is the \emph{table} --
an associative array that can represent arrays, records, maps, modules, objects, etc.
Tables also have syntactic sugar and metaprogramming support
through operator overloading built into the language.
Unlike other scripting languages, Lua has very limited coercion
among different data types.

Lua prefers to provide mechanisms instead of fixed policies due
to its primary use as an embedded language for configuration and
extension of other applications.
This means that even features such as a module system and
object orientation are a matter of convention instead of default
language constructs.
The result is a fragmented ecosystem of libraries, and different
ideas among Lua programmers on how they should use the language
features, or how they should structure programs.

The lack of standard policies is a challenge for the design of
an optional type system for Lua.
For this reason, we are not relying entirely on the semantics of
the language to design our type system.
We also run a mostly automated survey of Lua idioms used in a
large corpus of Lua libraries, which also has helped in the design of Typed Lua.

So far, Typed Lua is a Lua extension that allows statically typed
code to coexist and interact with dynamically typed code
through optional type annotations.
In addition, it adds default constructs that programmers can use
to better structure Lua programs.
The Typed Lua compiler warns programmers about type errors,
but always generates Lua code that runs in unmodified Lua implementations.
Programmers can enjoy some of the benefits of static types
even without converting existing Lua modules to Typed Lua --
they can export a statically typed interface to a dynamically typed module,
and statically typed users of the module can use the Typed Lua compiler
to check their use of the module.
Thus, implementing an optional type system for Lua offers Lua
programmers one way to obtain most of the advantages of static typing
without compromising the simplicity and flexibility of dynamic typing.
We have an implementation of the Typed Lua compiler that is
available online\footnote{https://github.com/andremm/typedlua}.

Typed Lua's intended use is as an application language, and
we believe that policies for organizing a program in modules and writing
object-oriented programs should be part of the language and
checked by its optional type system.
An application language is a programming language that helps
programmers develop applications from scratch until these
applications evolve into complex systems rather than just scripts.
We will show that Typed Lua introduces the refinement of
tables to support the common idioms that Lua programmers use
to encode both modules and objects.

We also believe that Typed Lua helps programmers give more
formal documentation to already existing Lua code, as static types
are also a useful source of documentation in languages that provide
type annotations, because type annotations are always validated by
the type checker and therefore never get outdated.
Thus, programmers can use Typed Lua to define axioms about the
interfaces and types of dynamically typed modules.
We enforce this point by using Typed Lua to statically type
the interface of the Lua standard library and other commonly used
Lua libraries, so our compiler can check Typed Lua code that uses
these libraries.

Typed Lua performs a very limited form of local type inference
\cite{pierce2000lti}, as static typing does not necessarily mean
that programmers need to insert type annotations in the code.
Several statically typed languages such as Haskell provide some
amount of type inference that automatically deduces the types of
expressions.
Still, Typed Lua only requires a small amount of type annotations
due to the nature of its optional type system.

Typed Lua does not deal with code optimization, although another
important advantage of static types is that they help the compiler
perform optimizations and generate more efficient code.
However, we believe that the formalization of our optional type
system is precise enough to aid optimization in some Lua implementations.

We use some of the ideas of gradual typing to formalize Typed Lua.
Even though Typed Lua is an optional type system and thus does not
include run-time checks between dynamic and static regions of the
code, we believe that using the foundations of gradual typing to
formalize our optional type system will allow us to include run-time
checks in the future.

Finally, we believe that designing an optional type system for Lua may
shed some light on optional type systems for scripting languages
in general, as Lua is a small scripting language that shares
some features with other scripting languages such as JavaScript.

This work is split into seven chapters.
In Chapter \ref{chap:review} we review the literature about blending
static and dynamic typing in the same language, we discuss the differences
between optional type systems and gradual typing, and we also
present the results of our survey on Lua idioms.
In Chapter \ref{chap:typedlua} we use code examples to present the
design of Typed Lua.
In Chapter \ref{chap:system} we present our type system.
In Chapter \ref{chap:evaluation} we discuss the evaluation
results that we obtained while using Typed Lua to type existing Lua code.
In Chapter \ref{chap:related} we present some related work.
In Chapter \ref{chap:conc} we outline our contributions.



\chapter{Blending static and dynamic typing}
\label{chap:review}
We begin this chapter presenting a little bit of the history behind
combining static and dynamic typing in the same language.
Then, we introduce optional type systems and gradual typing.
After that, we discuss why optional type systems and two
other approaches are often called gradual typing.
We end this chapter presenting some statistics about the usage of
some Lua features and idioms that helped us identify how we should
combine static and dynamic typing in Lua.

\section{A little bit of history}
\label{sec:history}

Common LISP \citep{steele1982ocl} introduced optional type annotations
in the early eighties, but not for static type checking.
Instead, programmers could choose to declare types of variables as
optimization hints to the compiler, that is, type declarations are
just one way to help the compiler to optimize code.
These annotations are unsafe because they can crash the program
when they are wrong.

\citet{abadi1989dts} extended the simply typed lambda calculus with the
\texttt{Dynamic} type and the \texttt{dynamic} and \texttt{typecase}
constructs, with the aim to safely integrate dynamic code in
statically typed languages.
The \texttt{Dynamic} type is a pair \texttt{(v,T)} where \texttt{v} is a
value and \texttt{T} is the tag that represents the type of \texttt{v}.
The constructs \texttt{dynamic} and \texttt{typecase} are explicit
injection and projection operations, respectively.
That is, \texttt{dynamic} builds values of type \texttt{Dynamic} and
\texttt{typecase} safely inspects the type of a \texttt{Dynamic} value.
Thus, migrating code between dynamic and static type checking requires
changing type annotations and adding or removing \texttt{dynamic} and
\texttt{typecase} constructs throughout the code.

The \emph{quasi-static} type system proposed by \citet{thatte1990qst}
performs implicit coercions and run-time checks to replace the
\texttt{dynamic} and \texttt{typecase} constructs that were proposed by
\citet{abadi1989dts}.
To do that, quasi-static typing relies on subtyping with a top type
$\Omega$ that represents the dynamic type, and splits type checking
into two phases.
The first phase inserts implicit coercions from the dynamic type to
the expected type, while the second phase performs what Thatte calls
\emph{plausibility checking}, that is, it rewrites the program to
guarantee that sequences of upcasts and downcasts always have a
common subtype.

\emph{Soft typing} \citep{cartwright1991soft} is another approach
to combine static and dynamic typing in the same language.
The main goal of soft typing is to add static type checking to
dynamically typed languages without compromising their flexibility.
To do that, soft typing relies on type inference for
translating dynamically typed code to statically typed code.
The type checker inserts run-time checks around inconsistent code and
warns the programmer about the insertion of these run-time checks,
as they indicate the existence of potential type errors.
However, the programmer is free to choose between inspecting the
run-time checks or simply running the code.
This means that type inference and static type checking do
not prevent the programmer from running inconsistent code.
One advantage of soft typing is the fact that the compiler for
softly typed languages can use the translated code to generate
more efficient code, as the translated code statically type checks.
One disadvantage of soft typing is that it can be cumbersome when
the inferred types are meaningless large types that just confuse the
programmer.

\emph{Dynamic typing} \citep{henglein1994dts} is an approach
that optimizes code from dynamically typed languages by eliminating
unnecessary checks of tags.
Henglein describes how to translate dynamically typed code into
statically typed code that uses a \texttt{Dynamic} type.
The translation is done through a coercion calculus that uses type
inference to insert the operations that are necessary to type check
the \texttt{Dynamic} type during run-time.
Although soft typing and dynamic typing may seem similar, they are not.
Soft typing targets statically type checking of dynamically typed
languages for detecting programming errors, while
dynamic typing targets the optimization of dynamically
typed code through the elimination of unnecessary run-time checks.
In other words, soft typing sees code optimization as a side effect
that comes with static type checking.

\citet{findler2002chf} proposed contracts for higher-order functions
and blame annotations for run-time checks.
Contracts perform dynamic type checking instead of static type checking,
but deferring all verifications to run-time can lead to defects
that are difficult to fix, because run-time errors can show a
stack trace where it is not clear to programmers if the cause of a
certain run-time error is in application code or library code.
Even if programmers identify that the source of a certain run-time
error is in library code, they still may have problems to identify
if this run-time error is due to a violation of library's contract
or due to a bug, when the library is poorly documented.
In this approach, programmers can insert assertions in the form of
contracts that check the input and output of higher-order functions;
and the compiler adds blame annotations in the generated code
to track assertion failures back to the source of the error.

BabyJ \citep{anderson2003babyj} is an object-oriented language
without inheritance that allows programmers to incrementally annotate
the code with more specific types.
Programmers can choose between using the dynamically typed version
of BabyJ when they do not need types at all, and the statically
typed version of BabyJ when they need to annotate the code.
In statically typed BabyJ, programmers can use the
\emph{permissive type} $*$ to annotate the parts of the code that
still do not have a specific type or the parts of the code that should
have dynamic behavior.
The type system of BabyJ is nominal, so types are either class names
or the permissive type $*$.
However, the type system does not use type equality or subtyping,
but the relation $\approx$ between two types.
The relation $\approx$ holds when both types have the same name or
any of them is the permissive type $*$.
Even though the permissive type $*$ is similar to the dynamic type
from previous approaches, BabyJ does not provide any way to add
implicit or explicit run-time checks.

\citet{ou2004dtd} specified a language that combines static types
with dependent types.
To ensure safety, the compiler automatically inserts coercions
between dependent code and static code.
The coercions are run-time checks that ensure static code does not
crash dependent code during run-time.

\section{Optional Type Systems}
\label{sec:optional}

Optional type systems \citep{bracha2004pluggable} are an approach for
plugging static typing in dynamically typed languages.
They use optional type annotations to perform compile-time type checking,
though they do not influence the original run-time semantics
of the language.
This means that the run-time semantics should still catch type errors
independently of the static type checking.
For instance, we can view the typed lambda calculus as an optional
type system for the untyped lambda calculus, because both have the
same semantic rules and the type system serves only for discarding
programs that may have undesired behaviors \citep{bracha2004pluggable}.

Strongtalk \citep{bracha1993strongtalk,bracha1996strongtalk} is
a version of Smalltalk that comes with an optional type system.
It has a polymorphic type system that programmers can use to annotate
Smalltalk code or leave type annotations out.
Strongtalk assigns a dynamic type to unannotated expressions and allows
programmers to cast unannotated expressions to any static type.
This means that the interaction of the dynamic type with the rest of
the type system is unsound, so Strongtalk uses the original run-time
semantics of Smalltalk when executing programs, even if programs are
statically typed.

\emph{Pluggable type systems} \citep{bracha2004pluggable} generalize
the idea of optional type systems that Strongtalk put in practice.
The idea is to have different optional type systems that can be layered
on top of a dynamically typed language without affecting its original
run-time semantics.
Although these systems can be unsound in their interaction with the
dynamically typed part of the language or even by design, their
unsoundness does not affect run-time safety, as the language run-time
semantics still catches any run-time errors caused by an unsound
type system.

Dart \citep{dart} and TypeScript \citep{typescript} are new
languages that are designed with an optional type system.
Both use JavaScript as their code generation target because
their main purpose is web development.
In fact, Dart is a new class-based object-oriented language with
optional type annotations and semantics that resembles the
semantics of Smalltalk, while TypeScript is a strict superset of
JavaScript that provides optional type annotations and class-based
object-oriented programming.
Dart has a nominal type system, while TypeScript has a structural
one, but both are unsound by design.
For instance, Dart has covariant arrays, while TypeScript has
covariant function return types,
besides the interaction between statically and dynamically
typed code that is also unsound.

There is no common formalization for optional type systems, and
each language ends up implementing its optional type system in
its own way.
Strongtalk, Dart, and TypeScript provide an informal description of
their optional type systems rather than a formal one.
In the next section we shall see that we can use some features
of gradual typing \citep{siek2006gradual,siek2007objects} to
formalize optional type systems.

\section{Gradual Typing}
\label{sec:gradual}

The main goal of gradual typing \citep{siek2006gradual} is to allow
programmers to choose between static and dynamic typing in the same
language.
To do that, \citet{siek2006gradual} extended the simply typed
lambda calculus with the dynamic type $?$, as we shall see in
Figure \ref{fig:gtlc}.
In gradual typing, type annotations are optional, so an untyped
variable is syntactic sugar for a variable whose declared type is
the dynamic type $?$, that is, $\lambda x.e$ is equivalent to $\lambda x{:}?.e$.
Under these circumstances, we view gradual typing as a way to add
a dynamic type to statically typed languages.

\begin{figure}[!ht]
\dstart
$$
\begin{array}{llr}
t ::= & & \emph{types}\\
& \;\; \Number & \emph{base type number}\\
& | \; \String & \emph{base type string}\\
& | \; ? & \emph{dynamic type}\\
& | \; t \rightarrow t & \emph{function types}\\
e ::= & & \emph{expressions}\\
& \;\; l & \emph{literals}\\
& | \; x & \emph{variables}\\
& | \; \lambda x{:}t{.}e & \emph{abstractions}\\
& | \; e_{1} e_{2} & \emph{application}
\end{array}
$$
\dend
\caption{Syntax of the gradually-typed lambda calculus}
\label{fig:gtlc}
\end{figure}

The central idea of gradual typing is the \emph{consistency}
relation, written $t_{1} \sim t_{2}$.
The consistency relation allows implicit conversions to and from the
dynamic type, and disallows conversions between inconsistent types
\citep{siek2006gradual}.
For instance, $\Number \sim \;?$, $? \sim \Number$,
$\String \sim \;?$, and $? \sim \String$,
but $\Number \not\sim \String$, and
$\String \not\sim \Number$.
The consistency relation is both reflexive and symmetric, but
it is neither commutative nor transitive.

\begin{figure}[!ht]
\dstart
$$
\begin{array}{c}
\begin{array}{c}
t \sim t \mylabel{C-REFL}
\end{array}
\;
\begin{array}{c}
t \sim \;? \mylabel{C-DYNR}
\end{array}
\;
\begin{array}{c}
? \sim t \mylabel{C-DYNL}
\end{array}
\\ \\
\begin{array}{c}
\dfrac{t_{3} \sim t_{1} \;\;\; t_{2} \sim t_{4}}
      {t_{1} \rightarrow t_{2} \sim t_{3} \rightarrow t_{4}} \mylabel{C-FUNC}
\end{array}
\end{array}
$$
\dend
\caption{The consistency relation}
\label{fig:consistency}
\end{figure}

Figure \ref{fig:consistency} defines the consistency relation.
The rule \textsc{C-REFL} is the reflexive rule.
Rules \textsc{C-DYNR} and \textsc{C-DYNL} are the rules that allow
conversions to and from the dynamic type $?$.
The rule \textsc{C-FUNC} resembles subtyping between function types,
because it is contravariant on the argument type and covariant on the
return type.

Figure \ref{fig:gts} uses the consistency relation in the typing rules
of the gradual type system of the simply typed lambda calculus extended
with the dynamic type $?$.
The environment $\env$ is a function from variables to types, and
the directive $type$ is a function from literal values to types.
The rule \textsc{T-VAR} uses the environment function $\env$ to get the
type of a variable $x$.
The rule \textsc{T-LIT} uses the directive $type$ to get the type of
a literal $l$.
The rule \textsc{T-ABS} evaluates the expression $e$ with an environment
$\env$ that binds the variable $x$ to the type $t_{1}$, and the resulting
type is the the function type $t_{1} \rightarrow t_{2}$.
The rule \textsc{T-APP1} handles function calls where the type of the
function is dynamically typed; in this case, the argument type may have
any type and the resulting type has the dynamic type.
The rule \textsc{T-APP2} handles function calls where the type of the
function is statically typed; in this case, the argument type should
be consistent with the argument type of the function's signature.

\begin{figure}[!ht]
\dstart
$$
\begin{matrix}
\dfrac{\env(x) = t}
      {\env \vdash x:t} \mylabel{T-VAR}
\;\;\;
\dfrac{type(l) = t}
      {\env \vdash l:t} \mylabel{T-LIT}
\\ \\
\dfrac{\env, x:t_{1} \vdash e:t_{2}}
      {\env \vdash \lambda x:t_{1}.e:t_{1} \rightarrow t_{2}} \mylabel{T-ABS}
\;\;\;
\dfrac{\env \vdash e_{1}:\;? \;\;\;
       \env \vdash e_{2}:t}
      {\env \vdash e_{1} e_{2}:\;?} \mylabel{T-APP1}
\\ \\
\dfrac{\env \vdash e_{1}:t_{1} \rightarrow t_{2} \;\;\;
       \env \vdash e_{2}:t_{3} \;\;\;
       t_{3} \sim t_{1}}
      {\env \vdash e_{1} e_{2}:t_{2}} \mylabel{T-APP2}
\end{matrix}
$$
\dend
\caption{Gradual type system gradually-typed lambda calculus}
\label{fig:gts}
\end{figure}

Gradual typing is similar to the approaches proposed by
\citet{abadi1989dts} and \citet{thatte1990qst} by including a
dynamic type in a statically typed language.
However, these three approaches differ in the way they handle the
dynamic type.
While \citet{siek2006gradual} rely on the consistency relation,
\citet{abadi1989dts} rely on type equality with explicit projections
and injections, and \citet{thatte1990qst} relies on subtyping.

The subtyping relation $\subtype$ is actually a pitfall on Thatte's
quasi-static typing, because it sets the dynamic type
as the top and the bottom of the subtying relation:
$t \subtype \;?$ and $? \subtype t$.
Subtyping is transitive, so we know that
\[
\frac{\Number \subtype \;? \;\;\;
      ? \subtype \String}
     {\Number \subtype \String}
\]
Therefore, downcasts combined with the transitivity of subtyping
accepts programs that should be rejected.

Later, \citet{siek2007objects} reported that the consistency relation is
orthogonal to the subtyping relation, so we can combine them to achieve
the \emph{consistent-subtyping} relation, written $t_{1} \lesssim t_{2}$.
This relation is essential for designing gradual type systems for
object-oriented languages.
Like the consistency relation, and unlike the subtyping relation,
the consistent-subtyping relation is not transitive.
Therefore, $\Number \lesssim \;?$, $? \lesssim \Number$,
$\String \lesssim \;?$, and $? \lesssim \String$,
but $\Number \not\lesssim \String$, and
$\String \not\lesssim \Number$.

Now, we will show how we can combine consistency and subtyping
to compose a consistent-subtyping relation for the simply typed
lambda calculus extended with the dynamic type $?$.

\begin{figure}[!ht]
\dstart
$$
\begin{array}{c}
\begin{array}{c}
\Number \subtype \Number \mylabel{S-NUM}
\end{array}
\;
\begin{array}{c}
\String \subtype \String \mylabel{S-STR}
\end{array}
\\ \\
\begin{array}{c}
? \subtype \;? \mylabel{S-ANY}
\end{array}
\;
\begin{array}{c}
\dfrac{t_{3} \subtype t_{1} \;\;\; t_{2} \subtype t_{4}}
      {t_{1} \rightarrow t_{2} \subtype t_{3} \rightarrow t_{4}} \mylabel{S-FUN}
\end{array}
\end{array}
$$
\dend
\caption{The subtyping relation}
\label{fig:subtyping}
\end{figure}

Figure \ref{fig:subtyping} presents the subtyping relation for the simply
typed lambda calculus extended with the dynamic type $?$.
Even though we could have used the reflexive rule $t \subtype t$ to express
the rules \textsc{S-NUM}, \textsc{S-STR}, and \textsc{S-ANY},
we did not combine them into a single rule to make explicit the
neutrality of the dynamic type $?$ to the subtyping rules.
The dynamic type $?$ must be neutral to subtyping to avoid the pitfall
from Thatte's quasi-static typing.
The rule \textsc{S-FUN} defines the subtyping relation for function types,
which are contravariant on the argument type and covariant on the return type.

\begin{figure}[!ht]
\dstart
$$
\begin{array}{c}
\begin{array}{c}
\Number \lesssim \Number \mylabel{C-NUM}
\end{array}
\;
\begin{array}{c}
\String \lesssim \String \mylabel{C-STR}
\end{array}
\\ \\
\begin{array}{c}
T \lesssim \;? \mylabel{C-ANY1}
\end{array}
\;
\begin{array}{c}
? \lesssim T \mylabel{C-ANY2}
\end{array}
\\ \\
\begin{array}{c}
\dfrac{T_{3} \lesssim T_{1} \;\;\; T_{2} \lesssim T_{4}}
      {T_{1} \rightarrow T_{2} \lesssim T_{3} \rightarrow T_{4}} \mylabel{C-FUN}
\end{array}
\end{array}
$$
\dend
\caption{The consistent-subtyping relation}
\label{fig:consistent-subtyping}
\end{figure}

Figure \ref{fig:consistent-subtyping} combines consistency and subtyping to
compose the consistent-subtyping relation for the simply typed
lambda calculus extended with the dynamic type $?$.
When we combine consistency and subtyping, we are making subtyping handle
what casts are safe among static types, and we are making consistency
handle the casts that involve the dynamic type $?$.
The consistent-subtyping relation is not transitive, and thus
the dynamic type $?$ is not neutral to this relation.

So far, gradual typing looks like a mere formalization to optional
type systems, as a gradual type system uses the consistency or
consistent-subtyping relation to statically check the interaction
between statically and dynamically typed code, without influencing
the run-time semantics.

However, another important feature of gradual typing is the theoretic
foundation that it provides for inserting run-time checks that
prove dynamically typed code does not violate the invariants of
statically typed code, thus preserving type safety.
To do that, \citet{siek2006gradual,siek2007objects} defined the
run-time semantics of gradual typing as a translation to an
intermediate language with explicit casts at the frontiers between
statically and dynamically typed code.
The semantics of these casts is based on the higher-order contracts
proposed by \citet{findler2002chf}.

\citet{herman2007sgt} showed that there is an efficiency
concern regarding the run-time checks, because there are two
ways that casts can lead to unbounded space consumption.
The first affects tail recursion while the second appears when
first-class functions or objects cross the border between
static code and dynamic code, that is, some programs can apply
repeated casts to the same function or object.
\citet{herman2007sgt} use the coercion calculus outlined in
\citet{henglein1994dts} to express casts as coercions and
solve the problem of space efficiency.
Their approach normalizes an arbitrary sequence of coercions to a
coercion of bounded size.

Another concern about casts is how to improve debugging support,
because a cast application can be delayed and the error related
to that cast application can appear considerable distance
from the real error.
\citet{wadler2009wpc} developed \emph{blame calculus} as a way
to handle this issue, and \citet{ahmed2011bfa} extended
blame calculus with polymorphism.
Blame calculus is an intermediate language to integrate
static and dynamic typing that uses the blame tracking approach
proposed by \citet{findler2002chf}.

On the one hand, blame calculus solves the issue regarding
error reporting;
on the other hand, it has the space efficiency problem reported
by \citet{herman2007sgt}.
Thus, \citet{siek2009casts} extended the coercion calculus outlined in
\citet{herman2007sgt} with blame tracking to achieve an
implementation of the blame calculus that is space efficient.
After that, \citet{siek2010blame} proposed a new solution that also
handles both problems.
This new solution is based on a concept called \emph{threesome},
which is a way to split a cast between two parties into two casts
among three parties.
A cast has a source and a target type (a \emph{twosome}),
so we can split any cast into a downcast from the source to an
intermediate type that is followed by an upcast from the intermediate
type to the target type (a \emph{threesome}).

There are some projects that incorporate gradual typing into some
programming languages.
Reticulated Python \citep{reticulated,vitousek2014deg} is a research
project that evaluates the costs of gradual typing in Python.
Gradualtalk \citep{allende2013gts} is a gradually-typed Smalltalk
that introduces a new cast insertion strategy for gradually-typed
objects \citep{allende2013cis}.
Grace \citep{black2012grace,black2013sg} is a new object-oriented,
gradually-typed, educational language.
In Grace, modules are gradually-typed objects, that is, modules
may have types and methods as attributes, and can have a state
\citep{homer2013modules}.
ActionScript \citep{moock2007as3} is one the first languages that
incorporated gradual typing to its implementation and
Perl 6 \citep{tang2007pri} is also being designed with gradual typing,
though there is little documentation about the gradual type systems
of these languages.

\section{Approaches that are often called Gradual Typing}
\label{sec:approaches}

Gradual typing is similar to optional type systems in that type
annotations are optional, and unannotated code is dynamically
typed, but unlike optional type systems, gradual typing changes
the run-time semantics to preserve type safety, and it is a way to
add a dynamic type to statically typed languages.
More precisely, programming languages that include a gradual type
system implement the semantics of statically typed languages, so
the gradual type system inserts casts in the translated code to
guarantee that types are consistent before execution, while
programming languages that include an optional type system still
implement the semantics of dynamically typed languages, so all
the type checking also belongs to the semantics of each operation.

Still, we can view gradual typing as a way to formalize an optional
type system when the gradual type system does not insert run-time
checks.
BabyJ \citep{anderson2003babyj} and Alore \citep{lehtosalo2011alore}
are two examples of object-oriented languages that have an
optional type system with a formalization that relates to gradual typing,
though the optional type systems of both BabyJ and Alore are nominal.
BabyJ uses the relation $\approx$ that is similar to the consistency
relation while Alore combines subtyping along with the consistency
relation to define a \emph{consistent-or-subtype} relation.
The consistent-or-subtype relation is different from the
consistent-subtyping relation of \citet{siek2007objects}, but it is
also written $t_{1} \lesssim t_{2}$.
The consistent-or-subtype relation holds when $t_{1} \sim t_{2}$
or $t_{1} <: t_{2}$, where $<:$ is transitive and $\sim$ is not.
Alore also extends its optional type system to include optional
monitoring of run-time type errors in the gradual typing style.

Hence, optional type annotations for software evolution are likely
the reason why optional type systems are commonly called
gradual type systems.
Typed Clojure \citep{bonnaire-sergeant2012typed-clojure} is an
optional type system for Clojure that is now adopting the
gradual typing slogan.

\citet{flanagan2006htc} introduced \emph{hybrid type checking},
an approach that combines static types and \emph{refinement} types.
For instance, programmers can specify the refinement type
$\{x:Int \;|\; x \ge 0\}$ when they need a type for natural numbers.
The programmer can also choose between explicit or implicit casts.
When casts are not explicit, the type checker uses a theorem prover
to insert casts.
In our example of natural numbers, a cast would be inserted to check
whether an integer is greater than or equal to zero.

Sage \citep{gronski2006sage} is a programming language that
extends hybrid type checking with a dynamic type to
support dynamic and static typing in the same language.
Sage also offers optional type annotations in the gradual typing
style, that is, unannotated code is syntactic sugar for
code whose declared type is the dynamic type.

Thus, the inclusion of a dynamic type in hybrid type checking
along with optional type annotations, and the insertion of run-time
checks are likely the reason why hybrid type checking is
also viewed as a form of gradual typing.

\citet{tobin-hochstadt2006ims} proposed another approach for gradually
migrating from dynamically typed to statically typed code,
and they coined the term \emph{from scripts to programs} for
referring to this kind of interlanguage migration.
In their approach, the migration from dynamically typed to
statically typed code happens module-by-module, so they designed
and implemented Typed Racket \citep{tobin-hochstadt2008ts} for
this purpose.
Typed Racket is a statically typed version of Racket
(a Scheme dialect) that allows the programmer to write typed modules,
so Typed Racket modules can coexist with Racket modules,
which are untyped.

The approach used by \citet{tobin-hochstadt2008ts} to design and
implement Typed Racket is probably also called gradual typing
because it allows the programmer to gradually migrate from untyped
scripts to typed programs.
However, Typed Racket is a statically typed language,
and what makes it gradual is a type system with a dynamic type
that handles the interaction between Racket and Typed Racket modules.

\section{Statistics about the usage of Lua}
\label{sec:statistics}

In this section we present statistics about the usage of Lua
features and idioms.
We collected statistics about how programmers use tables, functions,
dynamic type checking, object-oriented programming, and modules.
We shall see that these statistics informed important design decisions
on our optional type system.

We used the LuaRocks repository to build our statistics database;
LuaRocks \citep{hisham2013luarocks} is a package manager for Lua
modules.
We downloaded the 3928 \texttt{.lua} files that were available in
the LuaRocks repository at February 1st 2014.
However, we ignored files that were not compatible with Lua 5.2,
the latest version of Lua at that time.
We also ignored \emph{machine-generated} files and test files,
because these files may not represent idiomatic Lua code,
and might skew our statistics towards non-typical uses of Lua.
This left 2598 \texttt{.lua} files from 262 different projects for
our statistics database;
we parsed these files and processed their abstract syntax tree
to gather the statistics that we show in this section.

To verify how programmers use tables, we measured how they
initialize, index, and iterate tables.
We present these statistics in the next three paragraphs to discuss
their influence on our type system.

The table constructor appears 23185 times.
In 36\% of the occurrences it is a constructor that initializes a
record (e.g., \texttt{\{ x = 120, y = 121 \}});
in 29\% of the occurrences it is a constructor that initializes a
list (e.g., \texttt{\{ "one", "two", "three", "four" \}});
in 8\% of the occurrences it is a constructor that initializes a
record with a list part;
and in less than 1\% of the occurrences (4 times) it is a constructor
that uses only the booleans \texttt{true} and \texttt{false} as indexes.
At all, in 73\% of the occurrences it is a constructor that uses
only literal keys;
in 26\% of the occurrences it is the empty constructor;
in 1\% of the occurrences it is a constructor with non-literal keys
only, that is, a constructor that uses variables and function calls
to create the indexes of a table;
and in less than 1\% of the occurrences (19 times) it is a constructor
that mixes literal keys and non-literal keys.

The indexing of tables appears 130448 times:
86\% of them are for reading a table field while
14\% of them are for writing into a table field.
We can classify the indexing operations that are reads as follows:
89\% of the reads use a literal string key,
4\% of the reads use a literal number key,
and less than 1\% of the reads (10 times) use a literal boolean key.
At all, 93\% of the reads use literals to index a table while
7\% of the reads use non-literal expressions to index a table.
It also worth mentioning that 45\% of the reads are actually
function calls.
More precisely, 25\% of the reads use literals to call a function,
20\% of the reads use literals to call a method, that is,
a function call that uses the colon syntactic sugar, 
and less than 1\% of the reads (195 times) use non-literal expressions
to call a function.
We can also classify the indexing operations that are writes as follows: 
69\% of the writes use a literal string key,
2\% of the writes use a literal number key,
and less than 1\% of the writes (1 time) uses a literal boolean key.
At all, 71\% of the writes use literals to index a table while
29\% of the writes use non-literal expressions to index a table.

We also measured how many files have code that iterate over tables to
observe how frequently iteration is used.
We observed that 23\% of the files have code that iterate over keys
of any value, that is, the call to \texttt{pairs} appears at least
once in these files (the median is twice per file);
21\% of the files have code that iterate over integer keys, that is,
the call to \texttt{ipairs} appears at least once in these files
(the median is also twice per file);
and 10\% of the files have code that use the numeric \texttt{for}
along with the length operator (the median is once per file).

The numbers about table initialization, indexing, and iteration
show us that tables are mostly used to represent records, lists,
and associative arrays.
Therefore, Typed Lua should include a table type for handling
these uses of Lua tables.
Even though the statistics show that programmers initialize tables
more often than they use the empty constructor to
dynamically initialize tables, the statistics of the empty
constructor are still expressive and indicate that Typed Lua should
also include a way to handle this style of defining table types.

We found a total of 24858 function declarations in our database
(the median is six per file).
Next, we discuss how frequently programmers use dynamic type
checking and multiple return values inside these functions.

We observed that 9\% of the functions perform dynamic type checking
on their input parameters, that is, these functions use \texttt{type}
to inspect the tags of Lua values (the median is once per function).
We randomly selected 20 functions to sample how programmers are
using \texttt{type}, and we got the following data:
50\% of these functions use \texttt{type} for asserting the tags of
their input parameters, that is, they raise an error when the tag of a
certain parameter does not match the expected tag, and
50\% of these functions use \texttt{type} for overloading, that is,
they execute different code according to the inspected tag.

These numbers show us that Typed Lua should include union types,
because the use of the \texttt{type} idiom shows that disjoint unions
would help programmers define data structures that can hold a value of
several different, but fixed types.
Typed Lua should also use \texttt{type} as a mechanism for decomposing
unions, though it may be restricted to base types only.

We observed that 10\% of the functions explicitly return multiple
values.
We also observed that 5\% of the functions return \texttt{nil} plus
something else, for signaling an unexpected behavior;
and 1\% of the functions return \texttt{false} plus something else,
also for signaling an unexpected behavior.

Typed Lua should include function types to represent Lua functions,
and tuple types to represent the signatures of Lua functions,
multiple return values, and multiple assignments.
Tuple types require some special attention, because Typed Lua
should be able to adjust tuple types during compile-time, in a
similar way to what Lua does with function calls and multiple
assignments during run-time.
In addition, the number of functions that return \texttt{nil} and
\texttt{false} plus something else show us that overloading on the
return type is also useful to the type system.

We also measured how frequently programmers use the object-oriented
paradigm in Lua.
We observed that 23\% of the function declarations are actually
method declarations.
More precisely, 14\% of them use the colon syntactic sugar while
9\% of them use \texttt{self} as their first parameter.
We also observed that 63\% of the projects extend tables with
metatables, that is, they call \texttt{setmetatable} at least once,
and 27\% of the projects access the metatable of a given table,
that is, they call \texttt{getmetatable} at least once.
In fact, 45\% of the projects extend tables with metatables and
declare methods:
13\% using the colon syntactic sugar, 14\% using \texttt{self}, and
18\% using both.

Based on these observations, Typed Lua should include support
to object-oriented programming.
Even though Lua does not have standard policies for object-oriented
programming, it provides mechanisms that allow programmers to
abstract their code in terms of objects, and our statistics confirm
that an expressive number of programmers are relying on these mechanisms
to use the object-oriented paradigm in Lua.
Typed Lua should include some standard way of defining interfaces and classes
that the compiler can use to type check object-oriented code,
but without changing the semantics of Lua.

We also measured how programmers are defining modules.
We observed that 38\% of the files use the current way of defining
modules, that is, these files return a table that contains the
exported members of the module at the end of the file;
22\% of the files still use the deprecated way of defining modules,
that is, these files call the function \texttt{module};
and 1\% of the files use both ways.
At all, 61\% of the files are modules while 39\% of the files are
plain scripts.
The number of plain scripts is high considering the origin of
our database.
However, we did not ignore sample scripts, which usually serve to
help the users of a given module on how to use this module, and
that is the reason why we have a high number of plain scripts.

Based on these observations, Typed Lua should include a way
for defining table types that represent the type of modules.
Typed Lua should also support the deprecated style of module
definition, using global names as exported members of the module.

Typed Lua should also include some way to define the types of
userdata.
This feature should also allow programmers to define userdata
that can be used in an object-oriented style, as this is another
common idiom from modules that are written in C.

The last statistics that we collected were about variadic functions
and vararg expressions.
We observed that 8\% of the functions are variadic, that is,
their last parameter is the vararg expression.
We also observed that 5\% of the initialization of lists
(or 2\% of the occurrences of the table constructor) use solely the
vararg expression.
Typed Lua should include a \emph{vararg type} to handle variadic
functions and vararg expressions.




\chapter{Typed Lua}
\label{chap:typedlua}
%\documentclass[mathserif]{beamer}
\documentclass{beamer}

\usepackage[english]{babel}
\usepackage[utf8]{inputenc}
\usepackage{amsmath}
\usepackage{amssymb}
\usepackage{listings}
\usepackage{beamerthemesplit}
%\usecolortheme{dove}

\newcommand{\mylabel}[1]{\; (\textsc{#1})}
\newcommand{\subtype}{<:}
\newcommand{\pipe}{|\;}
\newcommand{\kw}[1]{\mathbf{#1} \;}
\newcommand{\env}{\Gamma}

\begin{document}

\title{Typed Lua}
\subtitle{An Optional Type System for Lua}
\author{André Murbach Maidl}
\institute{LabLua\\PUC-Rio}
\date{Colóquios do LabLua}

\frame{\titlepage}

\begin{frame}
\frametitle{What is Typed Lua?}
\begin{itemize}
\item A typed superset of Lua that compiles to plain Lua.
\item \textcolor{blue}{Type annotations}.
\item \textcolor{blue}{Compile-time type checking}.
\item \textcolor{gray}{Classes}.
\item \textcolor{gray}{Interfaces}.
\item \textcolor{gray}{Modules}.
\end{itemize}
\end{frame}

\begin{frame}
\frametitle{Why Optional and not Gradual?}
\begin{itemize}
\item Because, first of all, gradual is optional!
\end{itemize}
\end{frame}

\begin{frame}
\frametitle{Optional versus Gradual}
\begin{center}
\begin{tabular}{|r|c|c|}
\hline
& Optional & Gradual\\
\hline
Optional type annotations & Yes & Yes \\ 
\hline
Compile-time type checking & Yes & Yes \\
\hline
Influence the run-time semantics & No & Yes \\
\hline
\end{tabular}
\end{center}
\end{frame}

\begin{frame}
\frametitle{Gradual Typing}
Gradual type systems use consistency ($\sim$) instead of equality ($=$).
\begin{Large}
\[
\tau ::= \gamma \;|\; ? \;|\; \tau \rightarrow \tau
\]
\[
\tau \sim \tau
\]
\[
\tau \sim \;?
\]
\[
?\; \sim \tau
\]

\[
\frac{\sigma_{1} \sim \tau_{1} \;\;\; \sigma_{2} \sim \tau_{2}}
     {\sigma_{1} \rightarrow \sigma_{2} \sim \tau_{1} \rightarrow \tau_{2}}
\]
\end{Large}
\end{frame}

\begin{frame}
\frametitle{Gradual Typing}
Gradual type systems use consistency (\textcolor{blue}{$\sim$}) instead of equality ($=$).
\begin{Large}
\[
e ::= c \;|\; x \;|\; \lambda x:\tau.e \;|\; e\;e \;\;\;
(\lambda x.e \equiv \lambda x:?.e)
\]
\[
\frac{\Gamma x = \tau}
     {\Gamma \vdash_{G} x:\tau} \;\;\;\;\;
\frac{\Delta c = \tau}
     {\Gamma \vdash_{G} c:\tau}
\]

\[
\frac{\Gamma(x \mapsto \sigma) \vdash_{G} e:\tau}
     {\Gamma \vdash_{G} \lambda x:\sigma.e:\sigma \rightarrow \tau} \;\;\;\;\;
\frac{\Gamma \vdash_{G} e_{1}:\;? \;\;\; \Gamma \vdash_{G} e_{2}:\tau_{2}}
     {\Gamma \vdash_{G} e_{1}\;e_{2}:\;?}
\]

\[
\frac{\Gamma \vdash_{G} e_{1}:\tau \rightarrow \tau' \;\;\;
      \Gamma \vdash_{G} e_{2}:\tau_{2} \;\;\; \textcolor{blue}{\tau_{2} \sim \tau}}
     {\Gamma \vdash_{G} e_{1}\;e_{2}:\tau'}
\]
\end{Large}
\end{frame}

\begin{frame}
\frametitle{The levels of Gradual Typing according to Jeremy Siek}
\begin{center}
\begin{tabular}{|r|c|c|c|}
\hline
& Level 1 & Level 2 & Level 3\\
\hline
Optional type annotations & Yes & Yes & Yes \\ 
\hline
Compile-time type checking & Yes & Yes & Yes \\
\hline
Run-time checking$^{1}$ & No & Yes & Yes \\
\hline
Blame tracking$^{2}$ & No & No & Yes\\
\hline
\end{tabular}
\end{center}
$1$ -- A compiler for a gradually typed language (level $>$ 1) infers
where dynamic checks are needed and inserts casts into the intermediate
language to perform these checks.\\
$2$ -- Blame tracking solves the problem of tracing a run-time cast
failure back to the source of the error.
\end{frame}

\begin{frame}
\frametitle{Examples of Gradually Typed Languages}
\begin{enumerate}
\item Strongtalk, TypeScript, Dart, and Typed Lua.
\item ActionScript, Typed Clojure, and Reticulated Python (library).
\item Typed Racket and Gradualtalk.
\end{enumerate}
\end{frame}

\begin{frame}
\frametitle{Overview of the rest of the presentation}
\begin{itemize}
\item Changes in the syntax of Lua.
\item Typed Lua Type System.
\end{itemize}
\end{frame}

\begin{frame}
\frametitle{Changes in the syntax of Lua}
\begin{align*}
stat ::= & \; ... \; |\\
& \textcolor{blue}{varlist} \; \texttt{`='} \; explist \; |\\
& \; ... \; |\\
& \textbf{for} \; \textcolor{blue}{namelist} \; \textbf{in} \;
explist \; \textbf{do} \; block \; \textbf{end}\\
& \textbf{function} \; funcname \; \textcolor{blue}{funcbody} \; |\\
& \textbf{local} \; \textbf{function} \; Name \; \textcolor{blue}{funcbody} \; |\\
& \textbf{local} \; \textcolor{blue}{namelist} \; [\texttt{`='} \; explist]\\
exp ::= & \; ... \; |\\
& functiondef \; |\\
& \; ...\\
functiondef ::= & \; \textbf{function} \; \textcolor{blue}{funcbody}
\end{align*}
\end{frame}

\begin{frame}
\frametitle{Declaration of variables}
\begin{align*}
stat ::= & \; ... \; |\\
& \textcolor{blue}{varlist} \; \texttt{`='} \; explist \; |\\
& \; ... \; |\\
& \textbf{for} \; \textcolor{blue}{namelist} \; \textbf{in} \;
explist \; \textbf{do} \; block \; \textbf{end}\\
& \; ... \; |\\
& \textbf{local} \; \textcolor{blue}{namelist} \; [\texttt{`='} \; explist]\\
varlist ::= & \; \textcolor{blue}{var} \; \{\texttt{`,'} \; \textcolor{blue}{var}\}\\
var ::= & \; Name \; \textcolor{blue}{[\texttt{`:'} \; type]} \; |\\
& prefixexp \; \texttt{`['} \; exp \; \texttt{`]'} \; |\\
& prefixexp \; \texttt{`.'} \; Name\\
namelist ::= & \; Name \; \textcolor{blue}{[\texttt{`:'} \; type]} \;
\{\texttt{`,'} \; Name \; \textcolor{blue}{[\texttt{`:'} \; type]}\}
\end{align*}
\end{frame}

\begin{frame}
\frametitle{Declaration of functions}
\begin{align*}
stat ::= & \; ... \; |\\
& \textbf{function} \; funcname \; \textcolor{blue}{funcbody} \; |\\
& \textbf{local} \; \textbf{function} \; Name \; \textcolor{blue}{funcbody} \; |\\
& \; ... \\
exp ::= & \; ... \; |\\
& functiondef \; |\\
& \; ...\\
functiondef ::= & \; \textbf{function} \; \textcolor{blue}{funcbody}\\
funcbody ::= & \; \texttt{`('} \; [\textcolor{blue}{parlist}] \; \texttt{`)'} \;
\textcolor{blue}{[\texttt{`:'} \; typelist]} \; block \; \textbf{end}\\
parlist ::= & \; \textcolor{blue}{namelist} \; [\texttt{`,'} \; \texttt{`...'} \;
\textcolor{blue}{[\texttt{`:'} \; type]}] \; | \;\\
& \texttt{`...'} \; \textcolor{blue}{[\texttt{`:'} \; type]}\\
namelist ::= & \; Name \; \textcolor{blue}{[\texttt{`:'} \; type]} \;
\{\texttt{`,'} \; Name \; \textcolor{blue}{[\texttt{`:'} \; type]}\}
\end{align*}
\end{frame}

\begin{frame}
\frametitle{Type annotations}
\begin{align*}
type ::= & \; basetype \;|\; \textbf{any} \;|\; \textbf{nil} \;|\\
& uniontype \;|\; functiontype\\
basetype ::= & \; \textbf{boolean} \;|\; \textbf{number} \;|\; \textbf{string}\\
uniontype ::= & \; type \;\texttt{`|'}\; type\\
functiontype ::= & \; \texttt{`('} \; [typelist] \; \texttt{`)'} \;
\texttt{`->'} \; \texttt{`('} \; [typelist] \; \texttt{`)'}\\
typelist ::= & \; type \; \{\texttt{`,'} \; type\} \; [\texttt{`*'}]
\end{align*}
\end{frame}

\begin{frame}
\frametitle{Example}
{\tt
\textcolor{white}{$Number \;\rightarrow\; Number$}\\
local function abs (n:number) : number\\
\textcolor{white}{--} if n < 0 then return -n end\\
\textcolor{white}{--} return n\\
end\\
\textcolor{white}{--}\\
\textcolor{white}{$Any \;\times\; Any \;\rightarrow\; Any$}\\
local function dist (x, y)\\
\textcolor{white}{--} return abs(x - y)\\
end\\
}
\end{frame}

\begin{frame}
\frametitle{Example}
{\tt
\textcolor{white}{$Number \;\rightarrow\; Number$}\\
local function abs (n:number) : number\\
\textcolor{white}{--} if n < 0 then return -n end\\
\textcolor{white}{--} return n\\
end\\
\textcolor{white}{--}\\
\textcolor{white}{$Any \;\times\; Any \;\rightarrow\; Any$}\\
local function dist (x\textcolor{blue}{:any}, y\textcolor{blue}{:any}) \textcolor{blue}{: any}\\
\textcolor{white}{--} return abs(x - y)\\
end\\
}
\end{frame}

\begin{frame}
\frametitle{To be or not to be Lua?}
{\tt
\textcolor{blue}{$Number \;\rightarrow\; Number$}\\
local function abs (n:number) : number\\
\textcolor{white}{--} if n < 0 then return -n end\\
\textcolor{white}{--} return n\\
end\\
\textcolor{white}{--}\\
\textcolor{blue}{$Any \;\times\; Any \;\rightarrow\; Any$}\\
local function dist (x:any, y:any) : any\\
\textcolor{white}{--} return abs(x - y)\\
end\\
}
\end{frame}

\begin{frame}
\frametitle{To be Lua!}
{\tt
$Number \;\textcolor{blue}{\times\; {Any*}} \;\rightarrow\; Number \;\textcolor{blue}{\times\; {Nil*}}$\\
local function abs (n:number) : number\textcolor{blue}{, nil*}\\
\textcolor{white}{--} if n < 0 then return -n end\\
\textcolor{white}{--} return n\\
end\\
\textcolor{white}{--}\\
$Any \;\times\; Any \;\textcolor{blue}{\times\; {Any*}} \;\rightarrow\; {Any\textcolor{blue}{*}}$\\
local function dist (x:any, y:any) : any\textcolor{blue}{*}\\
\textcolor{white}{--} return abs(x - y)\\
end\\
}
\end{frame}

\begin{frame}
\frametitle{Typed Lua Type System}
\begin{itemize}
\item Our aim is to design an object oriented language.
\item Most OO languages use the following subsumption rule:
\[
\frac{\env \vdash e:T_{1} \;\;\; T_{1} \subtype T_{2}}
     {\env \vdash e:T_{2}}
\]
\item We also aim to have gradual typing, so we also
need the consistency relation.
\end{itemize}
\end{frame}

\begin{frame}
\frametitle{Consistent-subtyping}
We do not use the subsumption rule, but subtyping
instead of equality.
\[
\frac{\env \vdash e:S_{1} \rightarrow S_{2} \;\;\;
      \env \vdash el:S_{3} \;\;\;
      \textcolor{red}{S_{3} \subtype S_{1}}}
     {\env \vdash e(el):S_{2}}
\]
and also the consistency rule
\[
\frac{\env \vdash e:S_{1} \rightarrow S_{2} \;\;\;
      \env \vdash el:S_{3} \;\;\;
      \textcolor{red}{S_{3} \subtype S_{1}'} \;\;\;
      \textcolor{blue}{S_{1}' \sim S_{1}}}
     {\env \vdash e(el):S_{2}}
\]
so we need to compose the two relations:
\[
\frac{\env \vdash e:S_{1} \rightarrow S_{2} \;\;\;
      \env \vdash el:S_{3} \;\;\;
      \textcolor{red!50!blue}{S_{3} \lesssim S_{1}}}
     {\env \vdash e(el):S_{2}}
\]
\end{frame}

\begin{frame}
\frametitle{Type Language, Type Relations, and Typing Rules}
See type\_system.pdf
\end{frame}

\end{document}


\chapter{The type system}
\label{chap:system}

In the previous chapter we presented an informal overview of Typed Lua.
We showed that programmers can use Typed Lua to combine static and dynamic
typing in the same code, and it allows them to incrementally migrate from
dynamic to static typing.
This is a benefit to programmers that use dynamically typed languages
to build large applications, as static types detect many bugs
during the development phase, and also provide better documentation.

In this chapter we present the formalization of Typed Lua's type system.
Besides its practical contributions, Typed Lua also has some interesting
contributions to the field of optional type systems for scripting
languages.
They are novel type system features that let Typed Lua cover several Lua idioms
and features, such as refinement of tables, multiple return values,
and optional parameters.

\section{Types}
\label{sec:types}

\begin{figure}[!ht]
\textbf{Type Language}\\
\dstart
$$
\begin{array}{rlr}
t ::= & & \textsc{first-level types:}\\
& \;\; l & \textit{literal types}\\
& | \; b & \textit{base types}\\
& | \; \Nil & \textit{nil type}\\
& | \; \Value & \textit{top type}\\
& | \; \Any & \textit{dynamic type}\\
& | \; \Self & \textit{self type}\\
& | \; t \cup t & \textit{disjoint union types}\\
& | \; s \rightarrow s & \textit{function types}\\
& | \; \{k_{1}{:}v_{1}, ..., k_{n}{:}v_{n}\}_{unique|open|closed|regular} & \textit{table types}\\
& | \; x & \textit{type variables}\\
& | \; \mu x.t & \textit{recursive types}\\
& | \; \phi(t,t) & \textit{filter types}\\
& | \; \pi_{i}^{x} & \textit{projection types}\\
%\multicolumn{3}{c}{}\\
l ::= & & \textsc{{\small literal types:}}\\
& \;\; \False \; | \; \True \; | \; {\it int} \; | \; {\it float} \; | \; {\it string} &\\
%\multicolumn{3}{c}{}\\
b ::= & & \textsc{{\small base types:}}\\
& \;\; \Boolean \; | \; \Integer \; | \; \Number \; | \; \String &\\
%\multicolumn{3}{c}{}\\
k ::= & & \textsc{{\small key types:}}\\
& \;\; l \; | \; b \; | \; \Value &\\
%\multicolumn{3}{c}{}\\
v ::= & & \textsc{{\small value types:}}\\
& \;\; t \; | \; \Const \; t &\\ 
%\multicolumn{3}{c}{}\\
s ::= & & \textsc{second-level types:}\\
& \;\; p & \textit{tuple types}\\
& | \; s \sqcup s & \textit{unions of tuple types}\\
%\multicolumn{3}{c}{}\\
p ::= & & \textsc{{\small tuple types:}}\\
& \;\; \Void & \textit{void type}\\
& | \; t* & \textit{variadic types}\\
& | \; t \times p & \textit{pair types}
\end{array}
$$
\dend
\caption{The abstract syntax of Typed Lua types}
\label{fig:typelang}
\end{figure}

Figure \ref{fig:typelang} presents the abstract syntax of
Typed Lua types.
Typed Lua splits types into two categories:
\emph{first-level types} and \emph{second-level types}.
First-level types represent first-class Lua values and
second-level types represent tuples of values that appear in 
assignments and function applications.
First-level types include literal types, base types, the type $\Nil$,
the top type $\Value$, the dynamic type $\Any$, the type $\Self$,
union types, function types, table types, recursive types,
filter types, and projection types.
Second-level types include tuple types and unions of tuple types.
Tuple types include the type $\Void$, variadic types, and pair types.
Types are ordered by a subtype relationship that we introduce
in Section \ref{sec:subtyping}, so Lua values may belong to
several distinct types.

Literal types represent the type of literal values.
They can be the boolean values $\False$ and $\True$,
an integer value, a floating point value, or a string value.
We will see that literal types are important in our treatment of
table types as records.

Typed Lua includes four base types: $\Boolean$, $\Integer$, $\Number$, and $\String$.
The base types $\Boolean$ and $\String$ represent the values that
Lua tags as \texttt{boolean} and \texttt{string} during run-time.
Lua 5.3 introduced two internal representations to the tag \texttt{number}:
\texttt{integer} for integer numbers and \texttt{float} for real numbers.
Lua does automatic promotion of \texttt{integer} values to \texttt{float}
values as needed.
We introduced the base type $\Number$ to represent \texttt{float} values,
and the base type $\Integer$ to represent \texttt{integer} values.
In the next section we will show that $\Integer$ is a subtype of $\Number$.
This allows programmers to keep using \texttt{integer} values where
\texttt{float} values are expected.

The type $\Nil$ is the type of \texttt{nil}, the value that Lua uses for
undefined variables, missing parameters, and missing table keys.

The type $\Value$ is the top type, which represents any Lua value.
In Section \ref{sec:rules} we will show that this type,
along with variadic types, helps the type system to drop extra values
on assignments and function calls, thus preserving the
semantics of Lua in these cases.

Typed Lua uses the type $\Self$ to represent the \emph{receiver}
in object-oriented method definitions and method calls.
As we mentioned in Section \ref{sec:oop}, we need the type
$\Self$ to prevent programs from indexing a method without
calling it with the correct receiver.

Union types $t_{1} \cup t_{2}$ represent types that can hold a value
of two different types.

Function types have the form $s \rightarrow s$ and represent Lua functions,
where $s$ is a second-level type.

Second-level types are either tuple types or unions of tuple types.
Tuple types are tuples of first-level types that can end with
either an empty tuple or with a variadic type.
Typed Lua needs second-level types because tuples are not first-class
values in Lua, only appearing on argument passing, multiple returns,
and multiple assignments.
The type $\Void$ is the type of an empty tuple.
A variadic type $t*$ represents a sequence of values of type $t \cup \Nil$;
it is the type of a vararg expression.
Second-level types include unions of tuples because Lua programs
usually overload the return type of functions to denote error,
as we mentioned in Section \ref{sec:statistics}.
For clarity, we use the symbol $\sqcup$ to represent the union between
two different tuple types.
Note that $\cup$ represents the union between two first-level types,
while $\sqcup$ represents the union between two tuple types.

Back to first-level types, table types represent the various forms
that Lua tables can take.
The syntactical form of table types is $\{ k_{1}{:}v_{1}, ..., k_{n}{:}v_{n} \}_{tag}$,
where each $k_{i}$ represents the type of a table key,
and each $v_{i}$ represents the type of the value that table keys of type $k_{i}$ map to.
Key types can only be literal types, base types, or the top type.
We made this restriction to the type of the keys because the statistics
that we discussed in Section \ref{sec:statistics} showed that most
of the tables are records, lists, and hashes.
The type $\Value$ is an option when we need a loose table type.
For instance, $\{\Value:\Value\}_{regular}$ represents the type of a
table in which both indices and values can have any type.
Value types can be any first-level type, and can optionally include
the $\Const$ type to denote immutable values.

We also use the tags \emph{unique}, \emph{open}, \emph{closed}, and
\emph{regular} to classify table types.
The tag \emph{unique} represents tables with no keys that do not
inhabit one of the table's key types, and with no alias.
In particular, the type of the table constructor has this tag.
The tag \emph{open} represents \emph{unique} table types that
have at least one alias.
The tag \emph{closed} represents \emph{unique} table types
that may have aliases.
In particular, the type of a class has this tag.
The tag \emph{regular} represents table types that do not provide
any guarantees about keys with types not listed in the table type.
In particular, in the concrete syntax, type annotations, interface
declarations, and userdata declarations always describe \emph{regular} table types.
In the next sections we explain in more detail why we need
different table types.

Any table type has to be \emph{well-formed}.
Informally, a table type is well-formed if key types do not overlap.
In Section \ref{sec:rules} we formalize the definition of well-formed table types.
We delay the proper formalization of well-formed table types because we use
consistent-subtyping in this formalization.

Recursive types have the form $\mu x.t$,
where $t$ is a first-level type that $x$ represents.
For instance, $\mu x.\{``info":\Integer, ``next":x \;\cup\; \Nil\}_{regular}$
is a type for singly-linked lists of integers.
In Section \ref{sec:alias} we mentioned that we can use the following
interface declaration as an alias to this type:
\begin{verbatim}
    local interface Element
      info:integer
      next:Element?
    end
\end{verbatim}

Typed Lua includes filter types as a way to discriminate the type of local
variables inside conditions.
In Section \ref{sec:rules} we show in more detail how our type system
uses them to formalize the \texttt{type} predicates that we mentioned
in Section \ref{sec:unions}.

Typed Lua includes projection types as a way to project
unions of tuple types into unions of first-level types.
In Section \ref{sec:rules} we show in more detail how our type system
uses them as a mechanism for handling unions of tuple types,
when they appear on the right-hand side of the declaration of local variables,
as we mentioned in Section \ref{sec:unions}.
We also show how this feature allows our type system to constrain
the type of a local variable that depends on the type of another local variable.

Typed Lua includes the dynamic type $\Any$ for allowing programmers
to mix static and dynamic typing.

\section{Subtyping}
\label{sec:subtyping}

Our type system uses subtyping \citep{cardelli1984smi,abadi1996to} to order
types and consistent-subtyping \citep{siek2007objects,siek2013mutable}
to allow the interaction between statically and dynamically typed code.
We explain the subtyping and consistent-subtyping rules throughout this section.
However, we focus the discussion on the definition of subtyping because,
as we mentioned in Section \ref{sec:gradual}, we can combine the
consistency and subtyping relations to achieve consistent-subtyping.
The differences between subtyping and consistent-subtyping are the way
they handle the dynamic type, and the fact that subtyping is transitive,
but consistent-subtyping is not.

We present the subtyping rules as a deduction system for the
subtyping relation $\senv \vdash t_{1} \subtype t_{2}$.
The variable $\senv$ is a set of pairs of recursion variables.
We need this set to record the hypotheses that we assume when checking
recursive types.

The subtyping rules for literal types and base types include the rules
for defining that literal types are subtypes of their respective base types,
and that $\Integer$ is subtype of $\Number$:
\[
\begin{array}{c}
\begin{array}{c}
\mylabel{S-FALSE}\\
\senv \vdash \False \subtype \Boolean
\end{array}
\;
\begin{array}{c}
\mylabel{S-TRUE}\\
\senv \vdash \True \subtype \Boolean
\end{array}
\;
\begin{array}{c}
\mylabel{S-STRING}\\
\senv \vdash {\it string} \subtype \String
\end{array}
\\ \\
\begin{array}{c}
\mylabel{S-INT1}\\
\senv \vdash {\it int} \subtype \Integer
\end{array}
\;
\begin{array}{c}
\mylabel{S-INT2}\\
\senv \vdash {\it int} \subtype \Number
\end{array}
\;
\begin{array}{c}
\mylabel{S-FLOAT}\\
\senv \vdash {\it float} \subtype \Number
\end{array}
\\ \\
\begin{array}{c}
\mylabel{S-NUMBER}\\
\senv \vdash \Integer \subtype \Number
\end{array}
\end{array}
\]

Subtyping is reflexive and transitive;
therefore, we could have omitted the rule \textsc{S-INT2}.
More precisely, we could have defined a transitive rule for first-level
types instead of defining specific rules for transitive cases.
For instance, a transitive rule would allow us to derive that
\[
\dfrac{\senv \vdash 1 \subtype \Integer \;\;\;
       \senv \vdash \Integer \subtype \Number}
      {\senv \vdash 1 \subtype \Number}
\]

However, we are using the subtyping rules as the template for defining
the consistent-subtyping rules, and consistent-subtyping is not
transitive.
More precisely, we want the subtyping and consistent-subtyping rules
to differ only in the way they handle the dynamic type.
Thus, we define the subtyping rules using an algorithmic approach
that is close to the implementation, as this approach allows us to use
subtyping to easily formalize consistent-subtyping.

Our type system includes the top type $\Value$,
so any first-level type is a subtype of $\Value$:
\[
\begin{array}{c}
\mylabel{S-VALUE}\\
\senv \vdash t \subtype \Value
\end{array}
\]

Many programming languages include a bottom type to represent
an empty value that programmers can use as a default expression,
and we could have used the type $\Nil$ for this role.
However, making $\Nil$ the bottom type would lead to several expressions
that would pass the type checker, but that would fail during run-time
in the presence of a \texttt{nil} value.
Thus, our type system does not have a bottom type, and $\Nil$ is a
subtype only of itself and of $\Value$.

Another type that is only a subtype of itself and of the type $\Value$
is the type $\Self$.

The subtyping rules for union types are standard:
\[
\begin{array}{c}
\begin{array}{c}
\mylabel{S-UNION1}\\
\dfrac{\senv \vdash t_{1} \subtype t \;\;\;
       \senv \vdash t_{2} \subtype t}
      {\senv \vdash t_{1} \cup t_{2} \subtype t}
\end{array}
\;
\begin{array}{c}
\mylabel{S-UNION2}\\
\dfrac{\senv \vdash t \subtype t_{1}}
      {\senv \vdash t \subtype t_{1} \cup t_{2}}
\end{array}
\;
\begin{array}{c}
\mylabel{S-UNION3}\\
\dfrac{\senv \vdash t \subtype t_{2}}
      {\senv \vdash t \subtype t_{1} \cup t_{2}}
\end{array}
\end{array}
\]

The first rule shows that a union type $t_{1} \cup t_{2}$
is subtype of $t$ if both $t_{1}$ and $t_{2}$ are subtypes
of $t$;
and the other rules show that a type $t$ is subtype
of a union type $t_{1} \cup t_{2}$ if $t$ is subtype of
either $t_{1}$ or $t_{2}$.

The subtyping rule for function types is also standard:
\[
\begin{array}{c}
\mylabel{S-FUNCTION}\\
\dfrac{\senv \vdash s_{2} \subtype s_{1} \;\;\;
       \senv \vdash r_{1} \subtype r_{2}}
      {\senv \vdash s_{1} \rightarrow s_{1} \subtype s_{2} \rightarrow r_{2}}
\end{array}
\]

The rule \textsc{S-FUNCTION} shows that subtyping between
function types is contravariant on the type of the parameter list
and covariant on the return type.
In the previous section we explained why our type system uses
second-level types to represent the type of the parameter list
and the return type.
Now, we explain their subtyping rules.

The type $\Void$ is a subtype of itself and of a variadic type:
\[
\begin{array}{c}
\mylabel{S-VOID}\\
\senv \vdash \Void \subtype t{*}
\end{array}
\]

A variadic type $t{*}$ represents a sequence of values of type
$t \cup \Nil$, and the rule \textsc{S-VOID} handles the case where
a given sequence is empty.

The subtyping rule for pair types is the standard covariant rule:
\[
\begin{array}{c}
\mylabel{S-PAIR}\\
\dfrac{\senv \vdash t_{1} \subtype t_{2} \;\;\;
       \senv \vdash s_{1} \subtype s_{2}}
      {\senv \vdash t_{1} \times s_{1} \subtype t_{2} \times s_{2}}
\end{array}
\]

The subtyping rules for variadic types are not so obvious:
\[
\begin{array}{c}
\begin{array}{c}
\mylabel{S-VARARG1}\\
\senv \vdash \Nil{*} \subtype \Void
\end{array}
\\ \\
\begin{array}{c}
\mylabel{S-VARARG2}\\
\dfrac{\senv \vdash t_{1} \cup \Nil \subtype t_{2} \cup \Nil}
      {\senv \vdash t_{1}{*} \subtype t_{2}{*}}
\end{array}
\;
\begin{array}{c}
\mylabel{S-VARARG3}\\
\dfrac{\senv \vdash t_{1} \cup \Nil \subtype t_{2}}
      {\senv \vdash t_{1}{*} \subtype t_{2} \times \Void}
\end{array}
\;
\begin{array}{c}
\mylabel{S-VARARG4}\\
\dfrac{\senv \vdash t_{1} \subtype t_{2} \cup \Nil}
      {\senv \vdash t_{1} \times \Void \subtype t_{2}{*}}
\end{array}
\\ \\
\begin{array}{c}
\mylabel{S-VARARG5}\\
\dfrac{\senv \vdash t_{1}{*} \subtype t_{2} \times \Void \;\;\;
       \senv \vdash t_{1}{*} \subtype s_{2}}
      {\senv \vdash t_{1}{*} \subtype t_{2} \times s_{2}}
\end{array}
\;
\begin{array}{c}
\mylabel{S-VARARG6}\\
\dfrac{\senv \vdash t_{1} \times \Void \subtype t_{2}{*} \;\;\;
       \senv \vdash s_{1} \subtype t_{2}{*}}
      {\senv \vdash t_{1} \times s_{1} \subtype t_{2}{*}}
\end{array}
\end{array}
\]

We need six different subtyping rules for variadic types
to handle all the cases where they can appear.
The rule \textsc{S-VARARG1} is a special rule for handling the
case where we give a sequence of $\Nil$ to the empty tuple.
The rule \textsc{S-VARARG2} handles the case where both tuple types end
with variadic types, and shows that $t_{1}{*}$ is a subtype of $t_{2}{*}$
if $t_{1} \cup \Nil$ is a subtype of $t_{2} \cup \Nil$.
This rule explicitly includes $\Nil$ in both sides because otherwise
$\Nil{*}$ would not be a subtype of several other variadic types.
For instance, $\Nil{*}$ would not be subtype of $\Number{*}$,
as $\Nil \not\subtype \Number$.
The other rules handle the cases where only one tuple type ends with a variadic type.
Note that the case where both tuple types end with the type $\Void$ does
not require any special rule.
In the next section we will show that we use these subtyping rules,
along with the types $\Value$ and $\Nil$, to make our type system reflect
the semantics of Lua on discarding extra parameters and
replacing missing parameters.

The subtyping rules for unions of tuple types are similar to the
subtyping rules for unions of first-level types:
\[
\begin{array}{c}
\begin{array}{c}
\mylabel{S-UNION4}\\
\dfrac{\senv \vdash s_{1} \subtype s \;\;\;
       \senv \vdash s_{2} \subtype s}
      {\senv \vdash s_{1} \sqcup s_{2} \subtype s}
\end{array}
\;
\begin{array}{c}
\mylabel{S-UNION5}\\
\dfrac{\senv \vdash s \subtype s_{1}}
      {\senv \vdash s \subtype s_{1} \sqcup s_{2}}
\end{array}
\;
\begin{array}{c}
\mylabel{S-UNION6}\\
\dfrac{\senv \vdash s \subtype s_{2}}
      {\senv \vdash s \subtype s_{1} \sqcup s_{2}}
\end{array}
\end{array}
\]

Back to the subtyping rules between first-level types,
the subtyping rule among a \emph{closed} or \emph{regular}
table type and another \emph{regular} table type resembles the
standard subtyping rule between records:
\[
\begin{array}{c}
\mylabel{S-TABLE1}\\
\dfrac{\forall i \in 1..n \; \exists j \in 1..m \;\;\;
       \senv \vdash k_{j} \subtype k_{i}' \;\;\;
       \senv \vdash k_{i}' \subtype k_{j} \;\;\;
       \senv \vdash v_{j} \subtype_{c} v_{i}'}
      {\senv \vdash \{k_{1}{:}v_{1}, ..., k_{m}{:}v_{m}\}_{closed|regular} \subtype \{k_{1}'{:}v_{1}', ..., k_{n}'{:}v_{n}'\}_{regular}} \; m \ge n
\end{array}
\]

The rule \textsc{S-TABLE1} allows width subtyping and introduces the
auxiliary relation $\subtype_{c}$ to handle depth subtyping on the
type of the values stored in the table fields.
We need an auxiliary relation because the subtyping of the
type of the values stored in the table fields changes according to
the tags of the table types.
We define the relation $\subtype_{c}$ as follows:
\[
\begin{array}{c}
\begin{array}{c}
\mylabel{S-FIELD1}\\
\dfrac{\senv \vdash v_{1} \subtype v_{2} \;\;\;
       \senv \vdash v_{2} \subtype v_{1}}
      {\senv \vdash v_{1} \subtype_{c} v_{2}}
\end{array}
\;
\begin{array}{c}
\mylabel{S-FIELD2}\\
\dfrac{\senv \vdash v_{1} \subtype v_{2}}
      {\senv \vdash \Const \; v_{1} \subtype_{c} \Const \; v_{2}}
\end{array}
\\ \\
\begin{array}{c}
\mylabel{S-FIELD3}\\
\dfrac{\senv \vdash v_{1} \subtype v_{2}}
      {\senv \vdash v_{1} \subtype_{c} \Const \; v_{2}}
\end{array}
\end{array}
\]

These rules allow depth subtyping on $\Const$ fields.
The rule \textsc{S-FIELD1} defines that mutable fields are invariant,
while the rule \textsc{S-FIELD2} defines that immutable fields are covariant.
The rule \textsc{S-FIELD3} defines that it is safe to promote fields
from mutable to immutable.
We do not include a rule that allows promoting fields from immutable
to mutable because this would be unsafe due to variance.

There is a limitation on \emph{regular} table types that led us to
introduce \emph{open} and \emph{unique} table types.
If the table constructor had a \emph{regular} table type, then
programmers would not be able to use it to initialize a variable with
a table type that describes a more general type.
For instance,
\begin{verbatim}
    local t:{"x":integer, "y":integer?} = { x = 1, y = 2 }
\end{verbatim}
would not type check, as the type of the table constructor would not
be a subtype of the type in the annotation.
More precisely,
\[
\{``x":1, ``y":2\}_{regular} \not\subtype \{``x":\Integer, ``y":\Integer \cup \Nil\}_{regular}
\]

Simply promoting the type of each table value to its supertype would
not overcome this limitation, as it still would give to the table constructor
a regular table type without covariant mutable fields.
Thus, programmers would not be able to use the table constructor to
initialize a variable with a table type that includes an optional field.
Using the previous example,
\begin{align*}
& \{``x":\Integer, ``y":\Integer\}_{regular} \not\subtype \\
& \{``x":\Integer, ``y":\Integer \cup \Nil\}_{regular}
\end{align*}

We introduced \emph{unique} table types to avoid this limitation,
as they represent the type of tables with no keys that do not
inhabit one of the table's key types, and with no alias.
In particular, this is the case of the table constructor.
The following subtyping rule defines the subtyping relation among
\emph{unique} table types and \emph{regular} table types:
\[
\begin{array}{c}
\mylabel{S-TABLE2}\\
\dfrac{\begin{array}{c}
       \forall i \in 1..m \; \forall j \in 1..n \;
       \senv \vdash k_{i} \subtype k_{j}' \to \senv \vdash v_{i} \subtype_{u} v_{j}' \\
       \forall j \in 1..n \; \not\exists i \in 1..m \;
       \senv \vdash k_{i} \subtype k_{j}' \to \senv \vdash \Nil \subtype_{o} v_{j}'
       \end{array}}
      {\senv \vdash \{k_{1}{:}v_{1}, ..., k_{m}{:}v_{m}\}_{unique} \subtype
                    \{k_{1}'{:}v_{1}', ..., k_{n}'{:}v_{n}'\}_{regular}}
\end{array}
\]

The rule \textsc{S-TABLE2} allows width subtyping and covariant keys.
It allows covariant keys because we also want to use \emph{unique}
table types as a way to join table fields that inhabit \emph{regular} table types.
For instance, we want to use the table constructor to initialize
a variable with a table type that describes a hash.

The rule \textsc{S-TABLE2} introduced the auxiliary relations
$\subtype_{u}$ and $\subtype_{o}$.
The first allows depth subtyping on all fields,
while the second allows the omission of optional fields.
We define them as follows:
\[
\begin{array}{c}
\begin{array}{c}
\mylabel{S-FIELD4}\\
\dfrac{\senv \vdash v_{1} \subtype v_{2}}
      {\senv \vdash v_{1} \subtype_{u} v_{2}}
\end{array}
\;
\begin{array}{c}
\mylabel{S-FIELD5}\\
\dfrac{\senv \vdash v_{1} \subtype v_{2}}
      {\senv \vdash \Const \; v_{1} \subtype_{u} \Const \; v_{2}}
\end{array}
\;
\begin{array}{c}
\mylabel{S-FIELD6}\\
\dfrac{\senv \vdash v_{1} \subtype v_{2}}
      {\senv \vdash v_{1} \subtype_{u} \Const \; v_{2}}
\end{array}
\\ \\
\begin{array}{c}
\mylabel{S-FIELD7}\\
\dfrac{\senv \vdash \Nil \subtype v}
      {\senv \vdash \Nil \subtype_{o} v}
\end{array}
\;
\begin{array}{c}
\mylabel{S-FIELD8}\\
\dfrac{\senv \vdash \Nil \subtype v}
      {\senv \vdash \Nil \subtype_{o} \Const \; v}
\end{array}
\end{array}
\]

Using \emph{unique} table types to represent the type of the table
constructor allows our type system to type check the previous example.
More precisely,
\[
\{``x":1, ``y":2\}_{unique} \subtype \{``x":\Integer, ``y":\Integer \cup \Nil\}_{regular}
\]

Even though we allow width subtyping between \emph{unique} and \emph{regular}
table types, we do not allow it among \emph{unique} and other table types
because it would violate our definition of these other table types:
\[
\begin{array}{c}
\mylabel{S-TABLE3}\\
\dfrac{\begin{array}{c}
       \forall i \in 1..m \\
       \exists j \in 1..n \;
       \senv \vdash k_{i} \subtype k_{j}' \land \senv \vdash v_{i} \subtype_{u} v_{j}' \\
       \forall j \in 1..n \; \not\exists i \in 1..m \;
       \senv \vdash k_{i} \subtype k_{j}' \to \senv \vdash \Nil \subtype_{o} v_{j}'
       \end{array}}
      {\senv \vdash \{k_{1}{:}v_{1}, ..., k_{m}{:}v_{m}\}_{unique} \subtype
                    \{k_{1}'{:}v_{1}', ..., k_{n}'{:}v_{n}'\}_{unique|open|closed}}
\end{array}
\]

The rule that handles subtyping between \emph{open} and \emph{regular} table
types allows width subtyping:
\[
\begin{array}{c}
\mylabel{S-TABLE4}\\
\dfrac{\begin{array}{c}
       \forall i \in 1..m \; \forall j \in 1..n \;
       \senv \vdash k_{i} \subtype k_{j}' \to \senv \vdash v_{i} \subtype_{c} v_{j}' \\
       \forall j \in 1..n \; \not\exists i \in 1..m \;
       \senv \vdash k_{i} \subtype k_{j}' \to \senv \vdash \Nil \subtype_{o} v_{j}'
       \end{array}}
      {\senv \vdash \{k_{1}{:}v_{1}, ..., k_{m}{:}v_{m}\}_{open} \subtype
                    \{k_{1}'{:}v_{1}', ..., k_{n}'{:}v_{n}'\}_{regular}}
\end{array}
\]

However, the rule that handles subtyping among \emph{open} and
\emph{open} or \emph{closed} table types does not allow width subtyping:
\[
\begin{array}{c}
\mylabel{S-TABLE5}\\
\dfrac{\begin{array}{c}
       \forall i \in 1..m \\
       \exists j \in 1..n \;
       \senv \vdash k_{i} \subtype k_{j}' \land \senv \vdash v_{i} \subtype_{c} v_{j}' \\
       \forall j \in 1..n \; \not\exists i \in 1..m \;
       \senv \vdash k_{i} \subtype k_{j}' \to \senv \vdash \Nil \subtype_{o} v_{j}'
       \end{array}}
      {\senv \vdash \{k_{1}{:}v_{1}, ..., k_{m}{:}v_{m}\}_{open} \subtype
                    \{k_{1}'{:}v_{1}', ..., k_{n}'{:}v_{n}'\}_{open|closed}}
\end{array}
\]

The rules \textsc{S-TABLE4} and \textsc{S-TABLE5} allow joining fields
plus omitting optional fields.
Both rules use $\subtype_{c}$ to allow depth subtyping on $\Const$ fields only.

We introduced \emph{closed} table types because we needed a safe way
to represent the type of classes along with single inheritance.
The rule that handles subtyping between \emph{closed} table types
does not allow width subtyping, joining fields, and omitting fields,
but it allows depth subtyping on $\Const$ fields:
\[
\begin{array}{c}
\mylabel{S-TABLE6}\\
\dfrac{\forall i \in 1..n \; \exists j \in 1..n \;\;\;
       \senv \vdash k_{j} \subtype k_{i}' \;\;\;
       \senv \vdash k_{i}' \subtype k_{j} \;\;\;
       \senv \vdash v_{j} \subtype_{c} v_{i}'}
      {\senv \vdash \{k_{1}{:}v_{1}, ..., k_{n}{:}v_{n}\}_{closed} \subtype \{k_{1}'{:}v_{1}', ..., k_{n}'{:}v_{n}'\}_{closed}}
\end{array}
\]

In the next section we will show in more detail how our type system
uses \emph{unique}, \emph{open}, and \emph{closed} table types to
handle the refinement of table types.

We use the \emph{Amber rule} \citep{cardelli1986amber} to define
subtyping between recursive types:
\[
\begin{array}{c}
\begin{array}{c}
\mylabel{S-AMBER}\\
\dfrac{\senv[x_{1} \subtype x_{2}] \vdash t_{1} \subtype t_{2}}
      {\senv \vdash \mu x_{1}.t_{1} \subtype \mu x_{2}.t_{2}}
\end{array}
\;
\begin{array}{c}
\mylabel{S-ASSUMPTION}\\
\dfrac{x_{1} \subtype x_{2} \in \senv}
      {\senv \vdash x_{1} \subtype x_{2}}
\end{array}
\end{array}
\]

The rule \textsc{S-AMBER} also uses the rule \textsc{S-ASSUMPTION}
to check whether $\mu x_{1}.t_{1} \subtype \mu x_{2}.t_{2}$.
Both rules use the set of assumptions $\senv$,
where each assumption is a pair of recursion variables.
The rule \textsc{S-AMBER} extends $\senv$ with the assumption
$x_{1} \subtype x_{2}$ to check whether $t_{1} \subtype t_{2}$.
The rule \textsc{S-ASSUMPTION} allows the rule \textsc{S-AMBER}
to check whether an assumption is valid.

A recursive type may appear inside a first-level type, and our
type system includes subtyping rules to handle subtyping between
recursive types and other first-level types:
\[
\begin{array}{c}
\begin{array}{c}
\mylabel{S-UNFOLDR}\\
\dfrac{\senv \vdash t_{1} \subtype [x \mapsto \mu x.t_{2}]t_{2}}
      {\senv \vdash t_{1} \subtype \mu x.t_{2}}
\end{array}
\;
\begin{array}{c}
\mylabel{S-UNFOLDL}\\
\dfrac{\senv \vdash [x \mapsto \mu x.t_{1}]t_{1} \subtype t_{2}}
      {\senv \vdash \mu x.t_{1} \subtype t_{2}}
\end{array}
\end{array}
\]

As an example, the rule \textsc{S-UNFOLDR} allows our type system to
type check the function \texttt{insert} from Section \ref{sec:alias}:
\begin{verbatim}
    local function insert (e:Element?, v:integer):Element
      return { info = v, next = e }
    end
\end{verbatim}
that is, the type checker uses the rule \textsc{S-UNFOLDR} to verify whether
the type of the table constructor is a subtype of \texttt{Element}:
\begin{align*}
\{ & ``info":\Integer, \\
   & ``next":\mu x.\{``info":\Integer, ``next":x \;\cup\; \Nil\}_{regular} \cup \Nil \}_{unique} \subtype \\
& \mu x.\{``info":\Integer, ``next":x \;\cup\; \Nil\}_{regular}
\end{align*}

Filter types are subtypes only of themselves and of $\Value$.
More precisely, a filter type $\phi(t_{1},t_{2})$ is a subtype of
the same filter type $\phi(t_{1},t_{2})$, which shares the same
types $t_{1}$ and $t_{2}$, and it is also a subtype of $\Value$.

Projection types are subtypes only of themselves and of $\Value$.
More precisely, a projection type $\pi_{i}^{x}$ is a subtype of the
same projection type $\pi_{i}^{x}$, which shares the same union of
tuples $x$ and the same index $i$, and it is also a subtype of $\Value$.

The dynamic type $\Any$ is neither the bottom nor the top type,
but a separate type that is subtype only of itself and of $\Value$.

Even though the dynamic type $\Any$ does not interact with subtyping,
it does interact with consistent-subtyping.
We present the consistent-subtyping rules as a deduction system for
the consistent-subtyping relation $\senv \vdash t_{1} \lesssim t_{2}$.
As in the subtyping relation, $\senv$ is also a set of pairs of
recursion variables.
We define the consistent-subtyping rules for the dynamic type $\Any$
as follows:
\[
\begin{array}{c}
\begin{array}{c}
\mylabel{C-ANY1}\\
\senv \vdash t \lesssim \Any
\end{array}
\;
\begin{array}{c}
\mylabel{C-ANY2}\\
\senv \vdash \Any \lesssim t
\end{array}
\end{array}
\]

If we had set the type $\Any$ as both bottom and top types of our
subtyping relation, then any type $t_{1}$ would be a subtype of
any other type $t_{2}$.
The consequence of this is that all programs would type check without errors.
This would happen due to the transitivity of subtyping, that is,
we would be able to down-cast any type $t_{1}$ to $\Any$ and then up-cast
$\Any$ to any other type $t_{2}$.
The rules \textsc{C-ANY1} and \textsc{C-ANY2} are the rules that
allow the dynamic type to interact with other first-level types,
and thus allow dynamically typed code to coexist with statically
typed code.
Because of these two rules, consistent-subtyping cannot be transitive.
These two rules are the only rules that differ between
subtyping and consistent-subtyping, if we implement the subtyping rules
as we do in this section.

In the implementation of Typed Lua we also use consistent-subtyping to
normalize and simplify union types, though we let union types free in
the formalization.
For instance, the union type \texttt{boolean|any} results in the
type \texttt{any}, because \texttt{boolean} is consistent-subtype
of \texttt{any}.
Another example is the union type \texttt{number|nil|1} that
results in the union type \texttt{number|nil}, because
\texttt{1} is consistent-subtype of \texttt{number}.

\section{Typing rules}
\label{sec:rules}

In this section we use a reduced core of Typed Lua to present the
most interesting rules of our type system.
This core limits control flow to if and while statements,
has explicit type annotations, and explicit scope for variables.
It also has explicit method declarations and explicit method calls
instead of treating them as syntactic sugar.
We use this reduced core because it simplifies the presentation
of our type system.
Appendix \ref{app:rules} presents the full set of typing rules.

\begin{figure}[!ht]
\textbf{Abstract Syntax}\\
\dstart
$$
\begin{array}{rl}
s ::= & \;\; \mathbf{skip} \; | \;
s_{1} \; ; \; s_{2} \; | \;
\vec{l} = el \; | \;
m \; | \;
\mathbf{while} \; e \; \mathbf{do} \; s \; | \;
\mathbf{if} \; e \; \mathbf{then} \; s_{1} \; \mathbf{else} \; s_{2}\\
& | \; \mathbf{local} \; \vec{n{:}t} = el \; \mathbf{in} \; s \; | \;
\mathbf{rec} \; n{:}t = f \; \mathbf{in} \; s \; | \;
\mathbf{return} \; el \; | \;
e(el)_{s} \; | \;
e{:}n(el)_{s}\\
e ::= & \;\; \mathbf{nil} \; | \;
c \; | \;
{...}_{e} \; | \;
n_{e} \; | \;
e_{1}[e_{2}] \; | \;
{<}t{>} \; n \; | \;
f \; | \;
\{ \} \; | \;
\{ \; \vec{p} \; \} \; | \;
\{ \; {...}_{m} \; \} \; | \;
\{ \; \vec{p},{...}_{m} \; \}\\
& | \; e_{1} + e_{2} \; | \;
e_{1} \; {..} \; e_{2} \; | \;
e_{1} == e_{2} \; | \;
e_{1} < e_{2} \; | \;
e_{1} \; \mathbf{and} \; e_{2} \; | \;
e_{1} \; \mathbf{or} \; e_{2} \\
& | \; \mathbf{not} \; e \; | \;
- e \; | \;
\# \; e \; | \;
e(el)_{e} \; | \;
e{:}n(el)_{e}\\
l ::= & \;\; n_{l} \; | \;
e_{1}[e_{2}] \; | \;
n[c] \; {<}t{>}\\
c ::= & \;\; \mathbf{false} \; | \;
\mathbf{true} \; | \;
{\it int} \; | \;
{\it float} \; | \;
{\it string}\\
p ::= & \;\; [e_{1}] = e_{2}\\
el ::= & \;\; \mathbf{nothing} \; | \;
\vec{e} \; | \;
me \; | \;
\vec{e}, me \; | \;
me^{x} \; | \;
\vec{e}, me^{x}\\
me ::= & \;\; e(el)_{m} \; | \;
e{:}n(el)_{m} \; | \;
{...}_{m}\\
m ::= & \;\; \mathbf{fun} \; n_{1}{:}n_{2} \; (){:}r \; fb \; | \;
\mathbf{fun} \; n_{1}{:}n_{2} \; ({...}{:}t){:}r \; fb\\
& | \; \mathbf{fun} \; n_{1}{:}n_{2} \; (\vec{n{:}t}){:}r \; fb \; | \;
\mathbf{fun} \; n_{1}{:}n_{2} \; (\vec{n{:}t},{...}{:}t){:}r \; fb\\
f ::= & \;\; \mathbf{fun} \; (){:}r \; fb \; | \;
\mathbf{fun} \; ({...}{:}t){:}r \; fb \; | \;
\mathbf{fun} \; (\vec{n{:}t}){:}r \; fb \; | \;
\mathbf{fun} \; (\vec{n{:}t},{...}{:}t){:}r \; fb\\
fb ::= & \;\; s \;;\; \mathbf{return} \; el\\
n ::= & \;\; {\it name}\\
\end{array}
$$
\dend
\caption{Typed Lua abstract syntax}
\label{fig:syntax}
\end{figure}

Figure \ref{fig:syntax} presents the abstract syntax of core Typed Lua.
It splits the syntactic categories as follows:
$s$ are statements, $e$ are expressions, $l$ are left-hand values,
$el$ are expression lists, $c$ are literal constants, $p$ are table fields,
$me$ are expressions with multiple results, $f$ are function declarations,
$m$ are method declarations, $n$ are variable names,
$t$ are first-level types, and $r$ are second-level types.
The notation $\vec{n{:}t}$ denotes the non-empty list
$n_{1}{:}t_{1}, ..., n_{n}{:}t_{n}$.

Our reduced core also splits function application, method application, and
the vararg expression (${...}$) into different syntactic categories.
It uses $e(el)_{s}$ to denote function applications that produce no value
because they appear as statements,
$e(el)_{e}$ to denote function applications that produce only one value,
even if the function application returns multiple values,
and $e(el)_{m}$ to denote function applications that produces as many values
as the function application returns.
It uses the same categories for method applications.
For vararg expressions, it only uses ${...}_{e}$ and ${...}_{m}$,
as the vararg expression cannot appear as a statement.

We also include two kinds of type coercions in our core language:
the left-hand value $n[c] \; {<}t{>}$ and the expression ${<}t{>} \;n$.
Both allow the refinement of table types.
We also split variable names into two categories to have safe aliasing
of tables in the presence of refinement.
We use $n_{e}$ when they appear as expressions and $n_{l}$ when they
appear as left-hand values.

We present the typing rules as a deduction system for the typing relations
$\env_{1}, \penv \vdash s, \env_{2}$ and $\env_{1}, \penv \vdash e : t, \env_{2}$.
We use the first one for typing statements and the second one for typing expressions.
The first relation means that given a type environment $\env_{1}$
and a projection environment $\penv$, typing a statement $s$ produces
a new type environment $\env_{2}$.
The second relation means that given a type environment $\env_{1}$
and a projection environment $\penv$, typing a expression $e$ has
type $t$ and produces a new type environment $\env_{2}$.
We need two typing relations because only expressions have types,
though expressions and statements can modify a type environment.

Even though we can assign only first-level types to variables,
our type system should be able to project unions of first-level types
from unions of second-level types.
Thus, we need two environments because $\env$ maps variables to first-level types,
while $\penv$ maps variables to second-level types.
We will show how our type system uses the environment $\penv$ for
handling projection types.
We use $\env_{1}[n \mapsto t]$ to extend the environment $\env_{1}$
with the variable $n$ that maps to type $t$.
We use $\penv[x \mapsto r]$ to extend the environment $\penv$ with
the variable $x$ that maps to the second-level type $r$.

\subsection{Assignment and function application}
\label{sec:assignment}

Lua has multiple assignment, and the rule \textsc{T-ASSIGNMENT} uses
tuple types along with consistent-subtyping to type check this Lua feature:
\[
\begin{array}{c}
\mylabel{T-ASSIGNMENT}\\
\begin{array}{c}
\dfrac{\env_{1}, \penv \vdash el:r_{1}, \env_{2} \;\;\;
       \env_{2}, \penv \vdash \vec{l}:r_{2}, \env_{3} \;\;\;
       r_{1} \lesssim r_{2}}
      {\env_{1}, \penv \vdash \vec{l} = el,\env_{3}}
\end{array}
\end{array}
\]

As an example,
\[
x, y = 1, ``foo"
\]
type checks because
$1 \times ``foo" \times \Nil{*} \lesssim \Integer \times \String \times \Value{*}$,
assuming that $x$ and $y$ are in the environment with
types $\Integer$ and $\String$, respectively.

Our type system uses the type $\Value{*}$ to drop extra values
and the type $\Nil{*}$ to replace missing values.
The rule that type checks left-hand values list introduces the type
$\Value{*}$ to let the right-hand side produce more values than
expected in the left-hand side.
The rules that type check expression list introduce the type
$\Nil{*}$ to let the right-hand side produce fewer values than
expected in the left-hand side.
Our type system includes $\Nil{*}$ in the type of an expression list only
if its type does not end in another variadic type $t{*}$.
In the previous example, the rules \textsc{T-LHSLIST} and \textsc{T-EXPLIST2}
are the rules that introduce these types.
We define the rule \textsc{T-LHSLIST} as follows:
\[
\begin{array}{c}
\mylabel{T-LHSLIST}\\
\dfrac{\env, \penv \vdash l_{i}:t_{i}, \env_{i} \;\;\;
       \env_{m} = merge(\env_{1}, ..., \env_{n}) \;\;\;
       n = |\;\vec{l}\;|}
      {\env, \penv \vdash \vec{l}:t_{1} \times ... \times t_{n} \times \Value{*}, \env_{m}}
\end{array}
\]

First, this rule uses the same environment $\env$ to type check each
left-hand side value $l_{i}$.
Then, it takes each type $t_{i}$ and builds the tuple type for
the assignment rule, placing $\Value{*}$ in the end.
Table refinement can change the type environment while typing each left-hand value.
Thus, we use a partial auxiliary function \emph{merge} to collect
all the changes in a new environment $\env_{m}$, if there are no conflicts.

The rule \textsc{T-EXPLIST2} is similar to \textsc{T-LHSLIST}:
\[
\begin{array}{c}
\mylabel{T-EXPLIST2}\\
\dfrac{\env, \penv \vdash e_{i}:t_{i}, \env_{i} \;\;\;
       \env_{m} = merge(\env_{1}, ..., \env_{n}) \;\;\;
       n = |\;\vec{e}\;|}
      {\env, \penv \vdash \vec{e}:t_{1} \times ... \times t_{n} \times \Nil{*}, \env_{m}}
\end{array}
\]

We use \textsc{T-EXPLIST2} only when the last expression in the list
can produce a single value.
Rules \textsc{T-EXPLIST3} and \textsc{T-EXPLIST4} handle the case where
an expression list is just an expression that may produce multiple values:
\[
\begin{array}{c}
\begin{array}{c}
\mylabel{T-EXPLIST3}\\
\dfrac{\env_{1}, \penv \vdash me:t_{1} \times ... \times t_{n} \times \Void, \env_{2}}
      {\env_{1}, \penv \vdash me:t_{1} \times ... \times t_{n} \times \Nil{*}, \env_{2}}
\end{array}
\;
\begin{array}{c}
\mylabel{T-EXPLIST4}\\
\dfrac{\env_{1}, \penv \vdash me:t_{1} \times ... \times t_{n}{*}, \env_{2}}
      {\env_{1}, \penv \vdash me:t_{1} \times ... \times t_{n}{*}, \env_{2}}
\end{array}
\end{array}
\]

As an example, these rules allow our type system to type check the following example:
\[
x, y, z = f()_{m}
\]

In this example, we are assuming that $x$, $y$, $z$, and $f$ are in
the environment with types $\Integer$, $\String$, $\String \cup \Nil$, and
either $\Void \rightarrow \Integer \times \String \times \Void$ or
$\Value{*} \rightarrow \Integer \times \String \times \Nil{*}$,
depending on whether the programmer is using strict function types.
Thus, the example type checks because
\[
\Integer \times \String \times \Nil{*} \lesssim \Integer \times \String \times \String \cup \Nil \times \Value{*}
\]

Conversely,
\[
x = f()_{m}
\]
type checks because
\[
\Integer \times \String \times \Nil{*} \lesssim \Integer \times \Value{*}
\]

Rules for function applications are similar to the rule for multiple assignment.
The rule \textsc{T-APPLY1} handles the case where function applications
are expressions that produce multiple values:
\[
\begin{array}{c}
\mylabel{T-APPLY1}\\
\dfrac{\env_{1}, \penv \vdash e:r_{1} \rightarrow r_{2}, \env_{2} \;\;\;
       \env_{2}, \penv \vdash el:r_{3}, \env_{3} \;\;\;
       r_{3} \lesssim r_{1}}
      {\env_{1}, \penv \vdash e(el)_{m}:r_{2}, \env_{3}}
\end{array}
\]

We also use the rule \textsc{T-APPLY1} as the base case for the rules
that handle the cases where function applications are either statements
or expressions that produce only one value:
\[
\begin{array}{c}
\begin{array}{c}
\mylabel{T-STMAPPLY1}\\
\dfrac{\env_{1}, \penv \vdash e(el)_{m}:r, \env_{2}}
      {\env_{1}, \penv \vdash e(el)_{s},\env_{2}}
\end{array}
\;
\begin{array}{c}
\mylabel{T-EXPAPPLY1}\\
\dfrac{\env_{1}, \penv \vdash e(el)_{m}:r, \env_{2}}
      {\env_{1}, \penv \vdash e(el)_{e}:first(r), \env_{2}}
\end{array}
\end{array}
\]

The rule \textsc{T-STMAPPLY1} discards the produced values,
while the rule \textsc{T-EXPAPPLY1} uses the auxiliary function
\emph{first} to ensure that only one value is produced.
We can define \emph{first} inductively as follows:
\begin{align*}
first(\Void) & = \Nil\\
first(t{*}) & = t \cup \Nil\\
first(t \times r) & = t\\
first(r_{1} \sqcup r_{2}) & = first(r_{1}) \cup first(r_{2})
\end{align*}

Assuming that $f$ is a local function in the environment, and that $f$ has type
$\String \times \Integer \cup \Nil \times \Integer \cup \Nil \times \Value{*} \rightarrow \Integer{*}$,
the function call
\[
f(``foo")_{m}
\]
type checks through the rule \textsc{T-APPLY1}, because
\[
``foo" \times \Nil{*} \lesssim \String \times \Integer \cup \Nil \times \Integer \cup \Nil \times \Value{*}
\]
and the function call
\[
f(``foo",1,2,3)_{m}
\]
also type checks through the rule \textsc{T-APPLY1}, because
\[
``foo" \times 1 \times 2 \times 3 \times \Nil{*} \lesssim \String \times \Integer \cup \Nil \times \Integer \cup \Nil \times \Value{*}
\]

Our type system also catches arity mismatch.
For instance, assuming that $f$ has type
$\String \times \Integer \cup \Nil \times \Integer \cup \Nil \times \Void \rightarrow \Integer{*}$,
the function call
\[
f(``foo")_{m}
\]
type checks through the rule \textsc{T-APPLY1}, because
\[
``foo" \times \Nil{*} \lesssim \String \times \Integer \cup \Nil \times \Integer \cup \Nil \times \Void
\]
but the function call
\[
f(``foo",1,2,3)_{m}
\]
does not type check through the rule \textsc{T-APPLY1}, because
\[
``foo" \times 1 \times 2 \times 3 \times \Nil{*} \not\lesssim \String \times \Integer \cup \Nil \times \Integer \cup \Nil \times \Void
\]

When our type system type checks an expression list,
it always includes $\Nil{*}$ in the end of the type of this expression list
if its type does not end in a variadic type.
This behavior preserves the semantics of Lua on replacing missing values,
and it is necessary when we omit optional parameters in a function call,
like the previous example showed.

Using $\Nil{*}$ in the end of the type of expression lists also allows
our type system to catch arity mismatch in function calls without optional parameters.
For instance, assuming that $f$ has type
$\Integer \times \Integer \times \Void \rightarrow \Integer \times \Void$,
the function call
\[
f(1)_{m}
\]
does not type check through the rule \textsc{T-APPLY1}, because
\[
1 \times \Nil{*} \not\lesssim \Integer \times \Integer \times \Void
\]
and the function call
\[
f(1,2,3)_{m}
\]
also does not type check through the rule \textsc{T-APPLY1}, because
\[
1 \times 2 \times 3 \times \Nil{*} \not\lesssim \Integer \times \Integer \times \Void
\]

\subsection{Tables and refinement}
\label{sec:refinement}

The simplest expression involving tables in the empty table constructor.
Its type checking rule is straightforward:
\[
\begin{array}{c}
\mylabel{T-CONSTRUCTOR1}\\
\env, \penv \vdash \{\}:\{\}_{unique}, \env
\end{array}
\]

Our abstract syntax reduces the more complex uses of the table
constructor into three forms: $\{\;\vec{p}\;\}$, $\{\;{...}_{m}\;\}$,
and $\{\;\vec{p},{...}_{m}\;\}$.
The first one uses a list of table fields $[e_{1}] = e_{2}$,
the second one uses just the vararg expression, and the third one uses both.
We did not include in the abstract syntax a table constructor
that uses a list of expressions, because we can express it using the first form.
For instance, we can use the table constructor
$\{ [1] = ``x", [2] = ``y", [3] = ``z" \}$ to express $\{ ``x", ``y", ``z" \}$.

The rule \textsc{T-CONSTRUCTOR2} type checks a table type of the first form:
\[
\begin{array}{c}
\mylabel{T-CONSTRUCTOR2}\\
\dfrac{\env_{i}, \penv \vdash p_{i}:(k_{i},v_{i}), \env_{i+1} \;\;\;
       n = |\;\vec{p}\;| \;\;\;
       t = \{k_{1}{:}v_{1}, ..., k_{n}{:}v_{n}\}_{unique} \;\;\;
       wf(t)}
      {\env_{1}, \penv \vdash \{\;\vec{p}\;\}:t, \env_{n+1}}
\end{array}
\]

First, the rule \textsc{T-CONSTRUCTOR2} uses the auxiliary relation
$\env_{1}, \penv \vdash p : (k,v), \env_{2}$ to type check each table
field $p_{i}$.
This auxiliary relation means that given a type environment $\env_{1}$
and a projection environment $\penv$, typing a table field $p$
produces the pair $(k,v)$ and a new type environment $\env_{2}$. 
In the pair $(k,v)$, $k$ is type key type, while $v$ is the value type.

Then, the rule \textsc{T-CONSTRUCTOR2} uses each pair $(k_{i},v_{i})$
to build the table type that express the type of the given constructor, and
uses the predicate \emph{wf} to check whether this table type is well-formed.
Formally, a table type is well-formed if it obeys the following rule:
\[
\forall i \not\exists j \; i \not= j \wedge k_{i} \lesssim k_{j}
\]

Well-formed table types avoid ambiguity.
For instance, this rule detects that the table type
$\{1:\Number, \Integer:\String, \Any:\Boolean\}$ is ambiguous,
because the type of the value stored by key $1$ can be
$\Number$, $\String$, or $\Boolean$, as $1 \lesssim 1$,
$1 \lesssim \Integer$, and $1 \lesssim \Any$.
Moreover, the type of the value stored by a key of type $\Integer$,
which is not the literal type $1$, can be $\Number$ or $\Boolean$,
as $\Integer \lesssim \Integer$, and $\Integer \lesssim \Any$.

The definition of the auxiliary relation that type checks table
fields would be straightforward if our table types allowed any
first-level type in the type of the keys.
The rules \textsc{T-FIELD1}, \textsc{T-FIELD2}, and \textsc{T-FIELD3}
check if the type of the key is a subtype of $\Boolean$, $\Number$,
or $\String$, respectively:
\[
\begin{array}{c}
\begin{array}{c}
\mylabel{T-FIELD1}\\
\dfrac{\env_{1}, \penv \vdash e_{1}:t_{1}, \env_{2} \;\;\;
       \env_{2}, \penv \vdash e_{2}:t_{2}, \env_{3} \;\;\;
       t_{1} \subtype \Boolean}
      {\env_{1}, \penv \vdash [e_{1}] = e_{2}: (t_{1},rt(t_{2})), \env_{3}}
\end{array}
\\ \\
\begin{array}{c}
\mylabel{T-FIELD2}\\
\dfrac{\env_{1}, \penv \vdash e_{1}:t_{1}, \env_{2} \;\;\;
       \env_{2}, \penv \vdash e_{2}:t_{2}, \env_{3} \;\;\;
       t_{1} \subtype \Number}
      {\env_{1}, \penv \vdash [e_{1}] = e_{2}: (t_{1},rt(t_{2})), \env_{3}}
\end{array}
\\ \\
\begin{array}{c}
\mylabel{T-FIELD3}\\
\dfrac{\env_{1}, \penv \vdash e_{1}:t_{1}, \env_{2} \;\;\;
       \env_{2}, \penv \vdash e_{2}:t_{2}, \env_{3} \;\;\;
       t_{1} \subtype \String}
      {\env_{1}, \penv \vdash [e_{1}] = e_{2}: (t_{1},rt(t_{2})), \env_{3}}
\end{array}
\end{array}
\]

These rules use the function \emph{rt} in the type of the value
because our type system allows only \emph{regular} table types to
appear in the type of the values.
The function \emph{rt} promotes all table types to \emph{regular}.
We made this restriction because our type system does not keep track
of aliases to table fields.
This means that allowing \emph{unique}, \emph{open}, and \emph{closed}
table types to appear in the type of the values would allow the creation
of unsafe aliases.

The rule \textsc{T-FIELD4} is a fallback for the previous rules:
\[
\begin{array}{c}
\mylabel{T-FIELD4}\\
\dfrac{\env_{1}, \penv \vdash e_{1}:t_{1}, \env_{2} \;\;\;
       \env_{2}, \penv \vdash e_{2}:t_{2}, \env_{3}}
      {\env_{1}, \penv \vdash [e_{1}] = e_{2}: (\Value,rt(t_{2})), \env_{3}}
\end{array}
\]

The rule \textsc{T-FIELD4} shows that our type system uses the type $\Value$
as the type of the key whenever a key has a type that is neither a
subtype of $\Boolean$, nor a subtype of $\Number$, nor a subtype of $\String$.

As an example, the table constructor $\{[``x"] = 1, [``y"] = \{[``z"] = 2\}\}$
has type $\{``x":1, ``y":\{``z":2\}_{regular}\}_{unique}$ through the rules
\textsc{T-FIELD3} and \textsc{T-CONSTRUCTOR2}.

Typed Lua handles arrays as hashes that map integers to some type $t$.
In Lua, programmers often use the vararg expression to initialize arrays.
The rules \textsc{T-CONSTRUCTOR3} and \textsc{T-CONSTRUCTOR4} define
the behavior of the vararg expression for this case:
\[
\begin{array}{c}
\begin{array}{c}
\mylabel{T-CONSTRUCTOR3}\\
\dfrac{\env({...}) = t}
      {\env, \penv \vdash \{{...}_{m}\}:\{\Integer{:}t \cup \Nil\}_{unique}, \env}
\end{array}
\\ \\
\begin{array}{c}
\mylabel{T-CONSTRUCTOR4}\\
\dfrac{\begin{array}{c}
       \env_{i}, \penv \vdash p_{i}:(k_{i},v_{i}), \env_{i+1} \;\;\;
       n = |\;\vec{p}\;| \;\;\;
       \env_{i+1}({...}) = t_{v}\\
       t = \{k_{1}{:}v_{1}, ..., k_{n}{:}v_{n}, \Integer{:}t_{v} \cup \Nil\}_{unique} \;\;\;
       wf(t)
       \end{array}}
      {\env_{1}, \penv \vdash \{\;\vec{p},\;{...}_{m}\;\}:t, \env_{n+1}}
\end{array}
\end{array}
\]

The rule \textsc{T-CONSTRUCTOR3} type checks the case where we use
only the vararg expression inside a table constructor to initialize
an array.
If we assume that $...$ is in the environment and has type $\String$,
the table constructor $\{{...}_{m}\}$ has type $\{\Integer:\String \cup \Nil\}_{unique}$
through the rule \textsc{T-CONSTRUCTOR3}.

The rule \textsc{T-CONSTRUCTOR4} type checks the case where we use
the table constructor as a record that includes an array part.
If we assume that $...$ is in the environment and has type $\String$,
the table constructor $\{[``x"] = ``foo", [``y"] = \mathbf{true}, {...}_{m}\}$
has type $\{``x":``foo", ``y":\mathbf{true}, \Integer:\String \cup \Nil\}_{unique}$
through the rule \textsc{T-CONSTRUCTOR4}.
As another example, the rule \textsc{T-CONSTRUCTOR4} does not type check
the table constructor $\{ [1] = ``foo", [10] = \mathbf{true}, {...}_{m} \}$,
because the keys $1$ and $10$ overlap the array part,
and the resulting table type is not well-formed.

After discussing the typing rules of the table constructor,
we start the discussion of the rules that define the most
unusual feature of our type system: the refinement of table types.
The first kind of refinement allows programmers to add new
fields to \emph{unique} or \emph{open} table types through
field assignment.
For instance, in Section \ref{sec:tables} we presented the
following example:
\begin{verbatim}
    local person = {}
    person.firstname = "Lou"
    person.lastname = "Reed"
\end{verbatim}

We translate this example to our reduced core as follows:
\begin{center}
\begin{tabular}{ll}
\multicolumn{2}{l}{$\mathbf{local} \; person:\{\}_{unique} = \{\} \; \mathbf{in}$}\\
& \multicolumn{1}{l}{$person[``firstname"] \; {<}\String{>} = ``Lou";$}\\
& \multicolumn{1}{l}{$person[``lastname"] \; {<}\String{>} = ``Reed"$}
\end{tabular}
\end{center}

In this example, we assign the type $\{\}_{unique}$ to the variable
$person$, then we refine its type to $\{``firstname":\String\}_{unique}$,
and then we refine its type to $\{``firstname":\String, ``lastname":\String\}_{unique}$.
Rule \textsc{T-REFINE} type checks this use of refinement:
\[
\begin{array}{c}
\mylabel{T-REFINE}\\
\dfrac{\begin{array}{c}
       \env_{1}(n) = \{ k_{1}{:}v_{1}, ..., k_{n}{:}v_{n} \}_{open|unique}\\
       \env_{1}, \penv \vdash c:k, \env_{2} \;\;\;
       \not \exists i \in 1..n \; k \lesssim k_{i} \;\;\;
       v = rt(t)
       \end{array}}
      {\env_{1}, \penv \vdash n[c] {<}t{>}:t, \env_{2}[n \mapsto \{ k_{1}{:}v_{1}, ..., k_{n}{:}v_{n}, k{:}v\}_{open|unique}]}
\end{array}
\]

We restricted the refinement of table types to include only literal
keys, because its purpose is to make it easier the construction of
table types that represent records.

We use the refinement of table types to handle the declaration of
new global variables.
In Lua, the assignment \texttt{v = v + 1} translates to
\texttt{\string_ENV["v"] = \string_ENV["v"] + 1} when \texttt{v}
is not a local variable, where \texttt{\string_ENV} is a table
that stores the global environment.
For this reason, Typed Lua treats accesses to global variables as field accesses
to an open table in the top-level scope.
For instance,
\[
\string_ENV[``x"] \; {<}\String{>} = ``foo" \;;\; \string_ENV[``y"] \; {<}\Integer{>} = 1
\]
uses field assignment to add fields $``x"$ and $``y"$ to $\string_ENV$.
In this example and in the next examples we assume that
$\string_ENV$ is in the environment and has type $\{\}_{open}$.
Therefore, after these field assignments $\string_ENV$ has type
$\{``x":\String, ``y":\Integer\}_{open}$.

We do not allow the refinement of table types to add a field if it is
already present in the table's type.
For instance,
\[
\string_ENV[``x"] \; {<}\String{>} = ``foo" \;;\; \string_ENV[``x"] \; {<}\Integer{>} = 1
\]
does not type check, as we are trying to add $``x"$ twice.

We also do not allow the refinement of table types to introduce
fields with table types that are not \emph{regular}.
For instance,
\begin{center}
\begin{tabular}{l}
$\string_ENV[``x"] \; {<}\{\}_{unique}{>} = \{\}$
\end{tabular}
\end{center}
refines the type of $\string_ENV$ from $\{\}_{open}$ to $\{``x":\{\}_{regular}\}_{open}$.
Currently, our type system can only track \emph{unique} and
\emph{open} table types that are bound to local variables.

We can also use multiple assignment to refine table types:
\[
\string_ENV[``x"] \; {<}\String{>}, \string_ENV[``y"] \; {<}\Integer{>} = ``foo", 1
\]

This example type checks because all the environment changes are consistent, and
$``foo" \times 1 \times \Nil{*} \lesssim \String \times \Integer \times \Value{*}$.
However, the next example does not type check because it tries to add
the same field to $\string_ENV$, but with different types:
\[
\string_ENV[``x"] \; {<}\String{>}, \string_ENV[``x"] \; {<}\Integer{>} = ``foo", 1
\]

Aliasing an \emph{unique} or an \emph{open} table type produces a
\emph{closed} table type, as defined by rule \textsc{T-IDREAD}:
\[
\begin{array}{c}
\mylabel{T-IDREAD}\\
\dfrac{\env(n) = t_{1} \;\;\; t_{2} = read(\penv, t_{1})}
      {\env, \penv \vdash n_{e}:close(t_{2}), \env[n \mapsto open(t_{1})]}
\end{array}
\]

In Section \ref{sec:fap} we will explain that \textsc{T-IDREAD} uses
the auxiliary function \emph{read} because it may be accessing an
identifier that is bound to a filter or projection type.
As we mentioned in Section \ref{sec:unions}, our type system includes
a small set of \texttt{type} predicates that allow programmers to
discriminate union types.
In Section \ref{sec:fap} we will also explain that our type system
uses filter and projection types in the definition of these predicates
to handle the discrimination of unions of first-level and second-level
types.

The following example does not type check, as aliasing $a$ produces
the type $\{``foo":\Integer\}_{closed}$ that is not a subtype of
$\{``foo":\Integer \cup \Nil\}_{regular}$, the type of $b$:
\begin{center}
\begin{tabular}{lll}
\multicolumn{3}{l}{$\mathbf{local} \; a:\{``foo":\Integer\}_{unique} = \{ [``foo"] = 5 \} \; \mathbf{in}$}\\
& \multicolumn{2}{l}{$\mathbf{local} \; b:\{``foo":\Integer \cup \Nil \}_{regular} = a \; \mathbf{in}$}\\
& & \multicolumn{1}{l}{$b[``foo"] = \mathbf{nil}$}
\end{tabular}
\end{center}

As another example,
\begin{center}
\begin{tabular}{lll}
\multicolumn{3}{l}{$\mathbf{local} \; a:\{\}_{unique} = \{\} \; \mathbf{in}$}\\
& \multicolumn{2}{l}{$\mathbf{local} \; b:\{\}_{open} = a \; \mathbf{in}$}\\
& & \multicolumn{1}{l}{$a[``x"] \; {<}\String{>} = ``foo";$}\\
& & \multicolumn{1}{l}{$b[``x"] \; {<}\Integer{>} = 1$}\\
\end{tabular}
\end{center}
does not type check, as aliasing $a$ produces the type $\{\}_{closed}$
that is not a subtype of $\{\}_{open}$, the type of $b$.
Our type system has this behavior to warn programmers about
potential unsafe behaviors after this kind of alias.
In this example, it is unsafe to add the field $``x"$ to $b$,
as it changes the value that is stored in the field $``x"$ of $a$.

Aliasing an \emph{unique} table type while keeping the original
reference \emph{unique} can be unsafe.
For this reason, the rule \textsc{T-IDREAD} changes the type of
the original reference from \emph{unique} to \emph{open}.

We also need to close \emph{unique} and \emph{open} tables that
appear in the left-hand side of assignments, as defined by
rule \textsc{T-IDWRITE}:
\[
\begin{array}{c}
\mylabel{T-IDWRITE}\\
\dfrac{\env(n) = t_{1} \;\;\; t_{2} = write(t_{1})}
      {\env, \penv \vdash n_{l}:close(t_{2}), \env[n \mapsto close(t_{2})]}
\end{array}
\]

In Section \ref{sec:fap} we will explain that \textsc{T-IDWRITE} uses
the auxiliary function \emph{write} because it may be accessing an
identifier that is bound to a filter or projection type.
As we mentioned in Section \ref{sec:unions}, assignments restore
discriminated union types to their original types, and
function \emph{write} also works in this purpose.

As an example,
\begin{center}
\begin{tabular}{lll}
\multicolumn{3}{l}{$\mathbf{local} \; a:\{\}_{unique} = \{\} \; \mathbf{in}$}\\
& \multicolumn{2}{l}{$\mathbf{local} \; b:\{\}_{open} = \{\} \; \mathbf{in}$}\\
& & \multicolumn{1}{l}{$b = a;$}\\
& & \multicolumn{1}{l}{$a[``x"] \; {<}\String{>} = ``foo";$}\\
& & \multicolumn{1}{l}{$b[``x"] \; {<}\Integer{>} = 1$}\\
\end{tabular}
\end{center}
does not type check because we cannot add the field $``x"$ to $b$,
as its type is \emph{closed}, and thus does not allow changing the
value that is stored in the field $``x"$ of $a$.

We also have different rules for type checking table indexing to avoid
changing table types in these operations, as they cannot create aliases:
\[
\begin{array}{c}
\begin{array}{c}
\mylabel{T-INDEX1}\\
\dfrac{\begin{array}{c}
       \env_{1}(n) = t \;\;\;
       read(\penv, t) = \{k_{1}{:}v_{1}, ..., k_{n}{:}v_{n}\}\\
       \env_{1}, \penv \vdash e_{2}:k, \env_{2} \;\;\;
       \exists i \in 1{..}n \; k \lesssim k_{i}
       \end{array}}
      {\env_{1}, \penv \vdash n[e_{2}]:v_{i}, \env_{2}}
\end{array}
\\ \\
\begin{array}{c}
\mylabel{T-INDEX2}\\
\dfrac{\begin{array}{c}
       \env_{1}, \penv \vdash e_{1}:\{k_{1}{:}v_{1}, ..., k_{n}{:}v_{n}\}, \env_{2}\\
       \env_{2}, \penv \vdash e_{2}:k, \env_{3} \;\;\;
       \exists i \in 1{..}n \; k \lesssim k_{i}
       \end{array}}
      {\env_{1}, \penv \vdash e_{1}[e_{2}]:v_{i}, \env_{3}}
\end{array}
\end{array}
\]

A second form of refinement happens when we want to use an
\emph{unique} or \emph{open} table type in a context that expects a
\emph{closed} or \emph{regular} table type with a different shape.
This kind of refinement allows programmers to add optional fields
or merge existing fields.
To do that, Typed Lua includes a type coercion expression that
obeys the \textsc{T-COERCE} rule below:
\[
\begin{array}{c}
\mylabel{T-COERCE}\\
\dfrac{\env_{1}(n) \subtype t \;\;\;
       \env_{1}[n \mapsto t], \penv \vdash n_{e}:t_{1}, \env_{2}}
      {\env_{1}, \penv \vdash {<}t{>} \; n:t_{1}, \env_{2}}
\end{array}
\]

The rule \textsc{T-COERCE} allows our type system to type check the following example:
\begin{center}
\begin{tabular}{lll}
\multicolumn{3}{l}{$\mathbf{local} \; a:\{\}_{unique} = \{ \} \; \mathbf{in}$}\\
& \multicolumn{2}{l}{$a[``x"] \; {<}\String{>} = ``foo";$}\\
& \multicolumn{2}{l}{$a[``y"] \; {<}\String{>} = ``bar";$}\\
& \multicolumn{2}{l}{$\mathbf{local} \; b:\{``x":\String, ``y":\String \cup \Nil \}_{regular} =$}\\
& & \multicolumn{1}{l}{${<}\{``x":\String, ``y":\String \cup \Nil\}_{open}{>} \; a \; \mathbf{in} \; a[``z"] \; {<}\Integer{>} = 1$}
\end{tabular}
\end{center}

We can use $a$ to initialize $b$ because the coercion converts
the type of $a$ from $\{``x":\String, ``y":\String\}_{unique}$ to
$\{``x":\String, ``y":\String \cup \Nil\}_{open}$, and results in
$\{``x":\String, ``y":\String \cup \Nil\}_{closed}$,
which is a subtype of
$\{``x":\String, ``y":\String \cup \Nil\}_{regular}$, the type of $b$.
We can continue to refine the type of $a$ after aliasing it to $b$,
as it still holds an \emph{open} table.
At the end of this example, $a$ has type
$\{``x":\String, ``y":\String \cup \Nil, ``z":\Integer\}_{open}$.

We also need to make sure to close all \emph{unique} and \emph{open}
table types before we type check a nested scope.
The rule \textsc{T-FUNCTION3} illustrates this case:
\[
\begin{array}{c}
\mylabel{T-FUNCTION3}\\
\dfrac{closeall(\env_{1}[\vec{n} \mapsto \vec{t}]), \penv[\ret \mapsto r] \vdash s, \env_{2}}
      {\begin{array}{c}
       \env_{1}, \penv \vdash \mathbf{fun} \; (\vec{n{:}t}){:}r \; s:\vec{t} \times \Void \rightarrow r,\\
       openset(closeset(\env_{1}, fav(\mathbf{fun} \; (\vec{n{:}t}){:}r \; s)),rv(\mathbf{fun} \; (\vec{n{:}t}){:}r \; s))
       \end{array}}
\end{array}
\]

Rule \textsc{T-FUNCTION3} uses some auxiliary functions to change
the type of variables before type checking a nested scope and
also to change the type of assigned and referenced variables after
type checking a nested scope.
The function \emph{closeall} closes all \emph{unique} and \emph{open}
table types.
The function \emph{closeset} closes a given set of free assigned variables,
which is given by the function \emph{fav}.
The function \emph{openset} changes from \emph{unique} to \emph{open}
a given set of referenced variables, which is given by the function \emph{rv}.

This rule also extends the environment $\penv$, bounding the special
variable $\ret$ to the return type $r$.
Rule \textsc{T-RETURN} uses the type that is bound to $\ret$ in
$\penv$ to type check return statements:
\[
\begin{array}{c}
\mylabel{T-RETURN}\\
\dfrac{\env_{1} \vdash el:r_{1}, \env_{2} \;\;\;
       \penv(\ret) = r_{2} \;\;\;
       r_{1} \lesssim r_{2}}
      {\env_{1} \vdash \mathbf{return} \; el, \env_{2}}
\end{array}
\]

As an example, rule \textsc{T-FUNCTION3} prevents the following unsafe example to type check:
\begin{center}
\begin{tabular}{llll}
\multicolumn{4}{l}{$\mathbf{local} \; a:\{\}_{unique}, b:\{\}_{unique} = \{\}, \{\} \; \mathbf{in}$}\\
& \multicolumn{3}{l}{$\mathbf{local} \; f:\Integer \times \Void \rightarrow \Integer \times \Void =$}\\
& & \multicolumn{2}{l}{$\mathbf{fun} \; (x:\Integer):\Integer \times \Void$}\\
& & & \multicolumn{1}{l}{$b = a \;;\; \mathbf{return} \; x + 1$}\\
& \multicolumn{3}{l}{$\mathbf{in} \; a[``x"] \; {<}\Integer{>} = 1 \;;\; b[``x"] \; {<}\String{>} = ``foo" \;;\; f(a[``x"])_{s}$}
\end{tabular}
\end{center}

This example does not type check because we cannot add the field
$``x"$ to $b$, as its type is closed.
The assignment $b = a$ type checks because, at this point,
$a$ and $b$ have the same type: $\{\}_{closed}$.
Their type was closed by \emph{closeall} before type checking
the function body.
Their type would be restored to $\{\}_{unique}$ after type checking
the function body, but that assignment also triggers other two type changes.
First, the function \emph{fav} includes $b$ in the set of variables
that should be closed by \emph{closeset}.
Then, the function \emph{rv} includes $a$ in the set of variables
that should change from \emph{unique} to \emph{open} by \emph{openset}.
After declaring $f$, $a$ has type $\{\}_{open}$ and $b$ has type $\{\}_{closed}$,
so we can refine the type of $a$, but we cannot refine the type of $b$.

\subsection{Classes and objects}
\label{sec:cao}

Typed Lua also relies on the refinement of table types to type check
the idioms that Lua programmers use for incrementally building
classes and objects.
For instance, rule \textsc{T-METHOD3} illustrates how our type system
type checks method definitions:
\[
\begin{array}{c}
\mylabel{T-METHOD3}\\
\dfrac{\begin{array}{c}
       \env_{1}(n_{1}) = t_{s} \;\;\; t_{s} = \{k_{i}{:}v_{i}, ..., k_{n}{:}v_{n}\}_{unique|open}\\
       \env_{1}, \penv \vdash n_{2} : l, \env_{2} \;\;\;
       \not \exists i \in 1..n \; l \lesssim k_{i}\\
       closeall(\env_{1}[self \mapsto \Self, \vec{n} \mapsto \vec{t}, \self \mapsto t_{s}]),
       \penv[\ret \mapsto r] \vdash s, \env_{3}\\
       t_{o} = \{k_{i}{:}v_{i}, ..., k_{n}{:}v_{n}, l{:}\Const \; \Self \times \vec{t} \times \Void \rightarrow r\}_{unique|open}
       \end{array}}
      {\begin{array}{c}
       \env_{1}, \penv \vdash \mathbf{fun} \; n_{1}{:}n_{2} \; (\vec{n{:}t}){:}r \; s,\\
       openset(closeset(\env_{1}[n_{1} \mapsto t_{o}], fav(\mathbf{fun} \; (\vec{n{:}t}){:}r \; s)),rv(\mathbf{fun} \; (\vec{n{:}t}){:}r \; s))
       \end{array}}
\end{array}
\]

Rule \textsc{T-METHOD3} allows our type system to type check the
following example, which is a translation of the definition of the class
\texttt{Shape} from Section \ref{sec:oop}:
\begin{center}
\begin{tabular}{llll}
\multicolumn{4}{l}{$\mathbf{local} \; Shape{:}\{``x"{:}\Number, ``y"{:}\Number\}_{unique} = \{ [``x"] = 0.0, [``y"] = 0.0 \}$}\\
\multicolumn{4}{l}{$\mathbf{in}$}\\
& \multicolumn{3}{l}{$\mathbf{fun} \; Shape{:}new (x:\Number, y:\Number):\Self \times \Void$}\\
& & \multicolumn{2}{l}{$\mathbf{local} \; o:\Self = setmetatable(\{\}, \{ [``\string_\string_index"] = self \})$}\\
& & \multicolumn{2}{l}{$\mathbf{in} \; o[``x"] = x; \; o[``y"] = y; \; \mathbf{return} \; o$}\\
; & \multicolumn{3}{l}{$\mathbf{fun} \; Shape{:}move (x:\Number, y:\Number):\Void$}\\
& & \multicolumn{2}{l}{$self[``x"] = self[``x"] + x;$}\\
& & \multicolumn{2}{l}{$self[``y"] = self[``y"] + y;$}\\
& & \multicolumn{2}{l}{$\mathbf{return} \; \mathbf{nothing}$}
\end{tabular}
\end{center}

This example uses the type $\{``x":\Number, ``y":\Number\}_{unique}$ to
initialize the local variable $Shape$, which represents the class
that we are defining.
Then, it uses two method definitions to include $new$ and $move$ in our class.
These definitions refine the type of $Shape$, and it becomes the type below:
\begin{align*}
\{ & ``x":\Number, ``y":\Number,\\
   & \Const \; ``new":\Self \times \Number \times \Number \times \Void \rightarrow \Self \times \Void,\\
   & \Const \; ``move":\Self \times \Number \times \Number \times \Void \rightarrow \Void \}_{unique}
\end{align*}

Inside the definition of the methods \emph{new} and \emph{move},
we index variables that have type $\Self$.
This is possible because we include the rule \textsc{T-SELF}:
\[
\begin{array}{c}
\mylabel{T-SELF}\\
\dfrac{\env_{1}, \penv \vdash e:\Self, \env_{2} \;\;\;
       \env_{2}(\self) = t}
      {\env_{1}, \penv \vdash e:t, \env_{2}}
\end{array}
\]

When an expression has type $\Self$, our type system uses this
rule to produce the type that is bound to $\self$ in $\env$.
The variable $\self$ is a special variable that method definitions
and method calls use to bind the type of $\Self$ in the environment.

Inside the definition of the method \emph{new}, we use \emph{setmetatable}
to initialize $o$ with type $\Self$.
The rule \textsc{T-SETMETATABLE1} express this idea:
\[
\begin{array}{c}
\mylabel{T-SETMETATABLE1}\\
\dfrac{\env(n) = \Self}
      {\env, \penv \vdash setmetatable(\{\}, \{[``\string_\string_index"] = n\}):\Self, \env}
\end{array}
\]

In Section \ref{sec:oop} we also mentioned that our type system
uses \emph{setmetatable} to handle single inheritance.
Rule \textsc{T-SETMETATABLE2} allows refining the type of a new
class based on the type of an already defined class,
while rule \textsc{T-SETMETATABLE3} allows overriding a constructor:
\[
\begin{array}{c}
\begin{array}{c}
\mylabel{T-SETMETATABLE2}\\
\dfrac{\env(n) = \{k_{1}{:}v_{1}, ..., k_{n}{:}v_{n}\}_{closed}}
      {\env, \penv \vdash setmetatable(\{\}, \{[``\string_\string_index"] = n\}):\{k_{1}{:}v_{1}, ..., k_{n}{:}v_{n}\}_{open}, \env}
\end{array}
\\ \\
\begin{array}{c}
\mylabel{T-SETMETATABLE3}\\
\dfrac{\env_{1}, \penv \vdash e = t, \env_{2} \;\;\;
       t = \{k_{1}{:}v_{1}, ..., k_{n}{:}v_{n}\}_{closed} \;\;\;
       \env_{1}(n) = \Self \;\;\; \env_{1}(\sigma) \subtype t}
      {\env_{1}, \penv \vdash setmetatable(e, \{[``\string_\string_index"] = n\}):\Self, \env_{2}[\sigma \mapsto t]}
\end{array}
\end{array}
\]

After we define our class, we can use it to create instances
of this class and call their methods.
Rule \textsc{T-INVOKE1} handles the case where method calls
are expressions that produce multiple values:
\[
\begin{array}{c}
\mylabel{T-INVOKE1}\\
\dfrac{\begin{array}{c}
       \env_{1}, \penv \vdash e:t_{s}, \env_{2}\\
       \env_{2}[\sigma \mapsto t_{s}], \penv \vdash e[n]:\Const \; r_{1} \rightarrow r_{2}, \env_{3}\\
       \env_{3}[\sigma \mapsto t_{s}], \penv \vdash el:r_{3}, \env_{4} \;\;\;
       t_{s} \times r_{3} \lesssim r_{1}
       \end{array}}
      {\env_{1}, \penv \vdash e{:}n(el)_{m}:[\sigma \mapsto t]r_{2}, \env_{4}}
\end{array}
\]

We also use the rule \textsc{T-INVOKE1} as the base case for the rules
that handle the cases where method calls are either statements
or expressions that produce only one value:
\[
\begin{array}{c}
\begin{array}{c}
\mylabel{T-STMINVOKE1}\\
\dfrac{\env_{1}, \penv \vdash e{:}n(el)_{m}:r, \env_{2}}
      {\env_{1}, \penv \vdash e{:}n(el)_{s},\env_{2}}
\end{array}
\;
\begin{array}{c}
\mylabel{T-EXPINVOKE1}\\
\dfrac{\env_{1}, \penv \vdash e{:}n(el)_{m}:r, \env_{2}}
      {\env_{1}, \penv \vdash e{:}n(el)_{e}:first(r), \env_{2}}
\end{array}
\end{array}
\]

The rule \textsc{T-STMINVOKE1} discards the produced values,
while the rule \textsc{T-EXPINVOKE1} uses the predicate \emph{first} to
ensure that only one value is produced.

As an example, assuming that the object $o$ is in the environment and
has the type of the class $Shape$, the method call
\[
o{:}move(10, 10)_{m}
\]
type checks through the rule \textsc{T-INVOKE1}, because
\[
Shape \times 10 \times 10 \times \Nil{*} \lesssim Shape \times \Number \times \Number \times \Void
\]
assuming that $Shape$ is an alias to the type of the class $Shape$.
Note that rule \textsc{T-INVOKE1} uses the object type to bind
the type $\Self$ before type checking a method call.

\subsection{Filters and projections}
\label{sec:fap}

In Section \ref{sec:unions} we mentioned that our type system 
includes union types to encode three common Lua idioms:
the use of optional values, the overloading based on the tags of
input parameters, and the overloading on the return type of functions.
Now, we will discuss how our type system uses the $\mathbf{or}$
operator to handle optional values, how it uses filter types to
discriminate unions of first-level types, and how it uses
projection types to discriminate overloaded return types,
which are unions of second-level types.

Typed Lua includes five typing rules for handling the $\mathbf{or}$
logical operator and its common idioms:
\[
\begin{array}{c}
\begin{array}{c}
\mylabel{T-OR1}\\
\dfrac{\env_{1}, \penv \vdash e_{1}:t, \env_{2} \;\;\;
       \Nil \not\lesssim t \;\;\;
       \False \not\lesssim t}
      {\env_{1}, \penv \vdash e_{1} \; \mathbf{or} \; e_{2}:t, \env_{2}}
\end{array}
\\ \\
\begin{array}{c}
\mylabel{T-OR2}\\
\dfrac{\env_{1}, \penv \vdash e_{1}:\Nil, \env_{2} \;\;\;
       \env_{2}, \penv \vdash e_{2}:t, \env_{3}}
      {\env_{1}, \penv \vdash e_{1} \; \mathbf{or} \; e_{2}:t, \env_{3}}
\end{array}
\\ \\
\begin{array}{c}
\mylabel{T-OR3}\\
\dfrac{\env_{1}, \penv \vdash e_{1}:\False, \env_{2} \;\;\;
       \env_{2}, \penv \vdash e_{2}:t, \env_{3}}
      {\env_{1}, \penv \vdash e_{1} \; \mathbf{or} \; e_{2}:t, \env_{3}}
\end{array}
\\ \\
\begin{array}{c}
\mylabel{T-OR4}\\
\dfrac{\env_{1}, \penv \vdash e_{1}:\Nil \cup \False, \env_{2} \;\;\;
       \env_{2}, \penv \vdash e_{2}:t, \env_{3}}
      {\env_{1}, \penv \vdash e_{1} \; \mathbf{or} \; e_{2}:t, \env_{3}}
\end{array}
\\ \\
\begin{array}{c}
\mylabel{T-OR5}\\
\dfrac{\env_{1}, \penv \vdash e_{1}:t_{1}, \env_{2} \;\;\;
       \env_{2}, \penv \vdash e_{2}:t_{2}, \env_{3}}
      {\env_{1}, \penv \vdash e_{1} \; \mathbf{or} \; e_{2}:filter(filter(t_{1}, \Nil), \False) \cup t_{2}, \env_{3}}
\end{array}
\end{array}
\]

Rule \textsc{T-OR1} is the rule that defines the short circuit.
We use the consistent-subtyping relation in this rule to guarantee that
the type system checks the second expression when the first one
has the dynamic type, as it can be hiding a false value.

Rules \textsc{T-OR2}, \textsc{T-OR3}, and \textsc{T-OR4} guarantee
that the final result is the type of the second expression, because the
first one is certainly a false value.

Rule \textsc{T-OR5} is the most general rule, but it is also
the rule that handles the common $\mathbf{or}$ idioms.
It uses the auxiliary function \emph{filter} to filter possible false
values that might be part of the type of the first expression.
We can use pattern matching to the define the recursive function
\emph{filter} as follows:
\begin{align*}
filter(t_{1} \cup t_{2}, t_{1}) & = filter(t_{2}, t_{1})\\
filter(t_{1} \cup t_{2}, t_{2}) & = filter(t_{1}, t_{2})\\
filter(t_{1} \cup t_{2}, t_{3}) & = filter(t_{1}, t_{3}) \cup filter(t_{2}, t_{3})\\
filter(t_{1}, t_{2}) & = t_{1}
\end{align*}

Rule \textsc{T-OR5} allows our type system to type check the following example:
\begin{center}
\begin{tabular}{ll}
\multicolumn{2}{l}{$\mathbf{local} \; x:\String \cup \Nil = \mathbf{nothing} \; \mathbf{in}$}\\
& \multicolumn{1}{l}{$\mathbf{local} \; y:\String = x \; \mathbf{or} \; ``Hello"$}
\end{tabular}
\end{center}

Without the \emph{filter} function,
the expression $x \; \mathbf{or} \; ``Hello"$ would have type
$\String \cup \Nil \cup ``Hello"$.
The \emph{filter} function removes the type $\Nil$ from the result,
leaving the type $\String \cup ``Hello"$.
This means that type checking the expression $x \; \mathbf{or} \; ``Hello"$
results in the type $\String$, because union types are disjoint and
$``Hello" \lesssim \String$.

Another common idiom that programmers use in Lua is to overload
the type of function parameters, and use the function \texttt{type}
to execute different actions according to their types.
The rule \textsc{T-IF2} shows the case where our type system
discriminates an union of first-level types based on the tag \texttt{string}:
\[
\begin{array}{c}
\mylabel{T-IF2}\\
\dfrac{\begin{array}{c}
       \env_{1}(n) = t_{1} \cup t_{2}\\
       closeall(\env_{1}[n \mapsto \phi(t,\String)]), \penv \vdash s_{1}, \env_{2} \\
       closeall(\env_{1}[n \mapsto \phi(t,filter(t, \String))), \penv \vdash s_{2}, \env_{3}\\
       \env_{4} = openset(closeset(\env_{1}[n \mapsto t], fav(s_{1}) \cup fav(s_{2})),rv(s_{1}) \cup rv(s_{2}))
      \end{array}}
      {\env_{1}, \penv \vdash \mathbf{if} \; type(n) == ``string" \; \mathbf{then} \; s_{1} \; \mathbf{else} \; s_{2}, \env_{4}}

\end{array}
\]

In Typed Lua, \emph{type} is a primitive that triggers a type change.
In the rule \textsc{T-IF2}, the type of a variable $n$ changes from
$t_{1} \cup t_{2}$ to $\phi(t_{1} \cup t_{2},\String)$ inside the
$\mathbf{if}$ branch, and it changes the type of a variable $n$ from
$t_{1} \cup t_{2}$ to $\phi(t,filter(t_{1} \cup t_{2},\String))$
inside the $\mathbf{else}$ branch.

Rule \textsc{T-IF2} changes the type of a local variable $n$ to a
filter type because it gives the discriminated type when we use $n$ as
an expression, and it restores the original type when we use $n$
in an assignment.
In Section \ref{sec:refinement} we mentioned that rule \textsc{T-IDREAD}
uses the function \emph{read} to produce the discriminated type,
while rule \textsc{T-IDWRITE} uses the function \emph{write} to
restore the original type.
Now, we will give the definition of these functions.

We can define \emph{read} as follows:
\begin{align*}
read(\penv, \phi(t_{1},t_{2})) & = t_{2}\\
read(\penv, \pi_{i}^{x}) & = proj(\penv, x, i)\\
read(\penv, t) & = t
\end{align*}

The function \emph{read} uses the auxiliary function \emph{proj}
to project a union of first-level types, based on an union of
second-level types and an index from a projection type.
In this section, we will discuss how our type system uses
projection types to handle overloaded return types.

We can define \emph{write} as follows:
\begin{align*}
write(\phi(t_{1},t_{2})) & = t_{1}\\
write(t) & = t
\end{align*}

As an example, assuming that \emph{rep} is in the environment and has type
$\String \times \Integer \times \String \cup \Nil \times \Void \rightarrow \String \times \Void$,
the following code type checks through rule \textsc{T-IF2}:
\begin{center}
\begin{tabular}{llll}
\multicolumn{4}{l}{$\mathbf{local} \; o:\String \times \String \cup \Integer \times \Void \rightarrow \String \times \Void =$}\\
& \multicolumn{3}{l}{$\mathbf{fun} \; (a:\String, b:\String \cup \Integer):\String \times \Void$}\\
& & \multicolumn{2}{l}{$\mathbf{local} \; r:\String = ``" \; \mathbf{in}$}\\
& & & \multicolumn{1}{l}{$\mathbf{if} \; type(b) == ``string" \; \mathbf{then} \; r = a \;{..}\;b \; \mathbf{else} \; r = rep(a, b)_{e}\; ;$}\\
& & & \multicolumn{1}{l}{$\mathbf{return} \; r$}\\
\multicolumn{4}{l}{$\mathbf{in} \; o(``foo", 2)_{s}$}
\end{tabular}
\end{center}

In this example, the type of $b$ changes to
$\phi(\String \cup \Integer, \String)$ inside the $\mathbf{if}$ branch, and
it changes to $\phi(\String \cup \Integer, \Integer)$ inside the $\mathbf{else}$ branch.
Thus, reading $b$ results in the type $\String$ inside the $\mathbf{if}$ branch,
and it results in the type $\Integer$ inside the $\mathbf{else}$ branch.
Outside the condition, $b$ has its original type: $\String \cup \Integer$.

Typed Lua also includes similar rules to handle the tags \texttt{nil},
\texttt{boolean}, and \texttt{number}.
There is also a similar rule for handling the tag \texttt{integer}, but
it is limited to Lua 5.3, as it depends on the function \texttt{math.type}.
This function appears only in Lua 5.3 and it returns the tag \texttt{integer}
when its input parameter is a number that has an integer representation,
the tag \texttt{float} when its input parameter is a number that has a
floating point representation, or \texttt{nil} otherwise.

Lua programmers also overload the return type of functions to denote errors,
and our type system uses projection types to handle this idiom.

As an example, assuming that \emph{idiv} and \emph{print} are in the
environment with the respective types
$\Integer \times \Integer \times \Void \rightarrow \Integer \times \Integer \times \Void \sqcup \Nil \times \String \times \Void$
and
$\Value{*} \rightarrow \Void$,
the following code type checks in our type system:
\begin{center}
\begin{tabular}{ll}
\multicolumn{2}{l}{$\mathbf{local} \; q:\pi_{1}^{x}, r:\pi_{2}^{x} = idiv(1, 2)_{m}^{x} \; \mathbf{in}$}\\
& \multicolumn{1}{l}{$\mathbf{if} \; q \; \mathbf{then} \; print(q + r)_{s} \; \mathbf{else} \; print(``ERROR: " \; .. \; r)_{s}$}
\end{tabular}
\end{center}

There are three typing rules involved in the type checking of this example,
which we will explain now.

First, that example uses rule \textsc{T-EXPLIST6} for type checking
$idiv(1, 2)_{m}^{x}$:
\[
\begin{array}{c}
\mylabel{T-EXPLIST6}\\
\dfrac{\begin{array}{c}
       \env_{1}, \penv \vdash me^{x}:r, \env_{2}\\
       r = t_{1} \times ... \times t_{n} \times \Void \sqcup t_{1}' \times ... \times t_{n}' \times \Void
       \end{array}}
      {\env_{1}, \penv \vdash me^{x}:\pi_{1}^{x} \times ... \times \pi_{n}^{x} \times \Nil{*}, \env_{2}, (x,r)}
\end{array}
\]

Rule \textsc{T-EXPLIST6} uses the auxiliary relation
$\env_{1}, \penv \vdash el : r_{1}, \env_{2}, (x,r_{2})$.
This relation means that given a type environment $\env_{1}$ and
a projection environment $\penv$, an expression list $el$ has type
$r_{1}$ and produces a new type environment $\env_{2}$ and
produces a pair $(x,r_{2})$.
This pair means that the last expression of an expression list $el$
produces a second-level type $r_{2}$ that should be bound to
a variable $x$ in the projection environment,
as the resulting type of this expression is a tuple of projection
types $\pi_{i}^{x}$.

Our type system uses this pair to allow rule \textsc{T-LOCAL} to bound
a second-level type $r_{2}$ to a variable $x$ in $\penv$,
after rule \textsc{T-LOCAL} checks whether the type of an expression
list $el$ matches the types in the annotations:
\[
\begin{array}{c}
\mylabel{T-LOCAL}\\
\dfrac{\env_{1}, \penv \vdash el:r_{1}, \env_{2}, (x,r_{2}) \;\;\;
       r_{1} \lesssim \vec{t} \times \Value{*} \;\;\;
       \env_{2}[\vec{n} \mapsto \vec{t}], \penv[x \mapsto r_{2}] \vdash s, \env_{3}}
      {\env_{1}, \penv \vdash \mathbf{local} \; \vec{n{:}t} = el \; \mathbf{in} \; s, \env_{3} - \{\vec{n} \mapsto \vec{t}\}}
\end{array}
\]

After type checking the statement $s$, rule \textsc{T-LOCAL} produces a
new environment without the variables that it introduced before type checking $s$.

Introducing a variable $x$ in the projection environment allows our
type system to discriminate projection types $\pi_{i}^{x}$.
The rule \textsc{T-IF3} shows the case where our type system
discriminates a projection type based on the tag \texttt{nil}:
\[
\begin{array}{c}
\mylabel{T-IF3}\\
\dfrac{\begin{array}{c}
       \env_{1}(n) = \pi_{i}^{x} \;\;\; \penv(x) = r\\
       closeall(\env_{1}), \penv[x \mapsto fpt(r,\Nil,i)] \vdash s_{1}, \env_{2} \\
       closeall(\env_{1}), \penv[x \mapsto gpt(r,\Nil,i)] \vdash s_{2}, \env_{3} \\
       \env_{4} = openset(closeset(\env_{1}, fav(s_{1}) \cup fav(s_{2})),rv(s_{1}) \cup rv(s_{2}))
      \end{array}}
      {\env_{1}, \penv \vdash \mathbf{if} \; n \; \mathbf{then} \; s_{1} \; \mathbf{else} \; s_{2}, \env_{4}}
\end{array}
\]

Rule \textsc{T-IF3} uses the auxiliary functions \emph{fpt} and \emph{gpt}
to filter a projection $x$, affecting all variables that bind to the same projection.
For instance, our previous example type checks through rule \textsc{T-IF3},
because it makes the rule \textsc{T-IF3} use the function call
\[
fpt(\Integer \times \Integer \times \Void \sqcup \Nil \times \String \times \Void, \Nil, 1)
\]
to discriminate the projection $x$ to the single tuple
$\Integer \times \Integer \times \Void$ inside the $\mathbf{if}$ branch,
and the function call
\[
gpt(\Integer \times \Integer \times \Void \sqcup \Nil \times \String \times \Void, \Nil, 1)
\]
to discriminate the projection $x$ to the single tuple
$\Nil \times \String \times \Void$ inside the $\mathbf{else}$ branch.
Thus, reading $q$ and $r$ projects $\pi_{1}^{x}$ to $\Integer$ and
$\pi_{2}^{x}$ to $\Integer$ inside the $\mathbf{if}$ branch,
but it projects $\pi_{1}^{x}$ to $\Nil$ and $\pi_{2}^{x}$ to $\String$
inside the $\mathbf{else}$ branch.
Outside the condition, $q$ and $r$ use the original projection, that is,
they project to $\Integer \cup \Nil$ and $\Integer \cup \String$, respectively.

\subsection{Arithmetic operations}
\label{sec:arith}

Lua has operator overloading, and allows programmers to redefine
the behavior of some operations.
For instance, programmers can use metatables to redefine the
behavior of arithmetic operations.
Even though Typed Lua does not support operator overloading yet,
it includes typing rules that allow programmers to use the
dynamic type when they are using overloaded operations.
The following typing rules show how Typed Lua uses the dynamic type
to handle the overloading of arithmetic operations:
\[
\begin{array}{c}
\begin{array}{c}
\mylabel{T-ARITH5}\\
\dfrac{\env_{1}, \penv \vdash e_{1}:\Any, \env_{2} \;\;\;
       \env_{2}, \penv \vdash e_{2}:t, \env_{3}}
      {\env_{1}, \penv \vdash e_{1} + e_{2}:\Any, \env_{3}}
\end{array}
\\ \\
\begin{array}{c}
\mylabel{T-ARITH6}\\
\dfrac{\env_{1}, \penv \vdash e_{1}:t, \env_{2} \;\;\;
       \env_{2}, \penv \vdash e_{2}:\Any, \env_{3}}
      {\env_{1}, \penv \vdash e_{1} + e_{2}:\Any, \env_{3}}
\end{array}
\end{array}
\]

The rule \textsc{T-ARITH5} allows type checking the following
example:
\begin{center}
\begin{tabular}{l}
$\mathbf{local} \; x{:}\Any = 1 \; \mathbf{in} \; x = x + 1$
\end{tabular}
\end{center}

This example is safe, but the following it is not:
\begin{center}
\begin{tabular}{l}
$\mathbf{local} \; x{:}\Integer, \;y{:}\Any = 1 \; \mathbf{in} \; x = x + y$
\end{tabular}
\end{center}

Although this last example is not safe, it shows that optional
type systems still preserve the flexibility of dynamically
typed languages along with the benefits of static type checking.


\chapter{Evaluation}
\label{chap:evaluation}

We performed some case studies on existing Lua libraries
to evaluate the design of our type system.
For each library, we used Typed Lua to either rewrite its modules
or to write statically typed interfaces to its modules through
Typed Lua's description files.
In this chapter we present our evaluation results, discuss
some interesting cases, and compare our type system to
related work.

\begin{table}[!ht]
\begin{center}
\begin{tabular}{|l|c|c|c|c|c|c|}
\hline
\textbf{Case study} & \textbf{easy} & \textbf{poly} & \textbf{hard} & \textbf{Total} & \textbf{\%} \\
\hline
Lua Standard Libraries & 92 & 4 & 34 & 130 & 71\% \\
\hline
MD5 & 13 & 0 & 0 & 13 & 100\% \\
\hline
Lua Socket & 85 & 22 & 19 & 126 & 67\% \\
\hline
\end{tabular}
\end{center}
\caption{Evaluation results for each case study}
\label{tab:evalbycase}
\end{table}

Table \ref{tab:evalbycase} sumarizes our evaluation results for each
case study that we used Typed Lua for typing their members.
We split the members that we typed into three categories:
\emph{easy}, \emph{poly}, and \emph{hard}.
The \emph{easy} category represents the members that we could give
a precise static type for them.
The \emph{poly} category represents the members that we believe that
parametric polymorphism would help give a precise static type for them,
and we had to rely on the dynamic type to type them because our
type system does not support parametric polymorphism.
The \emph{hard} category represents the members that even parametric
polymorphism would not help give a precise static type for them,
and we also had to rely on the dynamic type to type them;
for instance, this category includes functions that require
intersection types to describe their precise static type.
The last column of the table shows the percentage of members that
are under the \emph{easy} category for each case study.
This percentage is our evaluation of Typed Lua, as it
represents how much static typing we could introduce to each one of
our case studies.

Before comparing our type system to related work, we will discuss
each case study in more detail.
For each case study, we will split the evaluation results according
to the modules that each one of them include.
This shall allow us to better discuss the contributions and limitations
of our type system.

\section{Lua Standard Libraries}

The Lua Standard Libraries \citep{luamanual} were our first case study.
We started to think about how we would type them at the same time that
we started to design our type system, as they could give us some hints
on our type system.
And they did: optional parameters and overloading on the return type
are two Lua features that our type system should handle to allow us
typing some of the functions that the standard libraries provide.

All libraries are separated C modules, and we used Typed Lua's description
files to give a statically typed interface to each module.
The \texttt{debug} module is the only one that we did not include in our
evaluation results, because it provides several functions that violate
basic assumptions about Lua code \citep{luamanual}.
For instance, we can use its function \texttt{setlocal} to change the value
of a local variable that is outside of the scope.
Table \ref{tab:evallsl} sumarizes the evaluation results for the Lua Standard Libraries.

\begin{table}[!ht]
\begin{center}
\begin{tabular}{|l|c|c|c|c|c|c|}
\hline
\textbf{Case study} & \textbf{Module} & \textbf{easy} & \textbf{poly} & \textbf{hard} & \textbf{Total} & \textbf{\%} \\
\hline
\multirow{9}{*}{Lua Standard Libraries}
& base & 8 & 0 & 18 & 26 & 31\% \\
\cline{2-7}
& coroutine & 0 & 0 & 6 & 6 & 0\% \\
\cline{2-7}
& package & 5 & 0 & 3 & 8 & 62\% \\
\cline{2-7}
& string & 14 & 0 & 0 & 14 & 100\% \\
\cline{2-7}
& table & 1 & 4 & 1 & 6 & 17\% \\
\cline{2-7}
& math & 28 & 0 & 1 & 29 & 97\% \\
\cline{2-7}
& bit32 & 12 & 0 & 0 & 12 & 100\% \\
\cline{2-7}
& io & 14 & 0 & 4 & 18 & 78\% \\
\cline{2-7}
& os & 10 & 0 & 1 & 11 & 91\% \\
\hline
\end{tabular}
\end{center}
\caption{Evaluation results for Lua Standard Libraries}
\label{tab:evallsl}
\end{table}

We could give precise static types to most of the modules,
but some of them are still hard to type with the current type
system.
The only modules that we could type all their members are
\texttt{string} and \texttt{bit32}.
Now we will discuss in more detail the particularities that
we found on each module, and the limitations that we found
on our type system.

The \texttt{base} module was quite hard to type because it
includes several functions that have dynamic behavior.
For instance, the function \texttt{next} traverses all
fields of a table, but the order that the indices are
enumerated is not specified.
The function \texttt{pairs} is also difficult to type, as
our type system cannot garantee that it will no try to
access a field that does not exist.
The functions \texttt{rawget} and \texttt{rawset} are also
examples of functions that their behavior depend on the
C implementation.

There are some functions that have a dynamic behavior and
the return type depend on the input type.
For instance, \texttt{assert} and \texttt{select}.

Moreover, the \texttt{base} module also include some functions
that the type of one input parameter depends on the type of
another.
For instance, there are two static types that we can
assign to the function \texttt{tonumber}:
\texttt{(value) -> (number?)} and
\texttt{(string, number) -> (number?)}.
More precisely, the first argument of \texttt{tostring} can be
a value of any type if it is the only argument, but it must
be a value of type string if there is a second argument,
which must be a value of type number.

We could not type the \texttt{coroutine} module because our
type system does not include the type \emph{thread}.
Lua has one-shot delimited continuations \citep{james2011yield}
in the form of \emph{coroutines} \citep{moura2009rc}, and
effect systems \citep{nielson1999type} are an approach that we
could use to describe control transfers with continuations.
However, for now coroutines are out of the scope of our type
system, and we use an empty \emph{userdata} declaration
to represent the type \emph{thread}.

We could type most the members of the \texttt{package} module,
but we could not type any member that is a table.
More precisely, we could not type the table \texttt{loaded}
that stores loaded modules, and the tables \texttt{preload}
and \texttt{searchers} that store module \emph{loaders}.
They are difficult to type because their type depend on the
modules a program loads.

The \texttt{table} module is specially difficult to type because
all of its members require parametric polymorphism.
All the functions of this module either receive or return a list
of elements, and parametric polymorphism would help us to describe
them with a generic type.

However, the lack of parametric polymorphism did not prevent us from
giving a precise type to \texttt{concat}, as it operates over lists
where all elements are strings or numbers.

Even if our type system had parametric polymorphism, it would
be difficult to type \texttt{insert}, as its type depends on
the calling arity.
We can call \texttt{insert} passing two or three parameters.
The first parameter is always a list.
If we call it with two arguments, then the second parameter
is the value to be inserted in the end of the list.
If we call it with three arguments, then the third parameter
is the value to be inserted in the list, and the second
parameter is the position where it should be inserted.
This function also does not follow the semantics of Lua on
discarding extra arguments.
Lua generates a run-time error whenever we pass more than three
arguments to \texttt{insert}, even if the first three parameters
match its definition.

Even though the \texttt{math} module is straightforward to type,
it includes a special case that is hard to type: the function \texttt{random}.
This function is difficult to type because its type also depends
on the calling arity.
We can call \texttt{random} passing zero, one, or three parameters.
If we pass no parameter, then it returns a float number.
If we pass one or two parameters, then all the parameters should be
integers and it returns an integer number.
Another problem with this function is that it also does not follow
the semantics of Lua on discarding extra arguments, and generates
a run-time error whenever we pass more than three arguments.
There is also a problem on its documentation, as it suggests that
the two integers are optional parameters, but, for \texttt{random},
optional parameters behave in a different way.
Usually, Lua functions replace optional parameters with a default value
when they are \texttt{nil}, but \texttt{random} generates a run-time
error instead.

We could type most members from \texttt{io} module, but we could not
precisely type \texttt{close} and \texttt{lines}.
The function and the method \texttt{close} are difficult to type
because their return type depend whether the file handle that is
being closed was created with \texttt{popen} or not.
If it was created with \texttt{popen}, then \texttt{close}
returns \texttt{(boolean) | (nil, string, number)};
otwerwise \texttt{close} returns \texttt{(boolean) | (nil, string)}.
The function and the method \texttt{lines} are difficult to type
because they number or string depending on the type of the argument.

The function \texttt{execute} is the only member from \texttt{os}
module that we could not give a precise type.
It is difficult to type because its return type depend on the
input type.
Its type can be either \texttt{() -> (boolean)} or
\texttt{(string) -> (boolean?, string, number)}.

To increase the static typing of the Lua Standard Libraries
we need to include effect types, parametric polymorphism,
and intersection types.
The first will allow us to type coroutines, the second will
improve static typing of table types, and the third is
important to describe functions that return different types
according to their input types.

\section{MD5}

The MD5 library \citep{lmd5} is the case study that we used to
introduce Typed Lua's description files.
We chose this case study because it is simple and contains just
one module, being a good example for introducing how we can use
Typed Lua's description files for giving a statically typed
interface to a Lua library that is written in C, and to 
introduce how we can type an \emph{userdata}.
Table \ref{tab:evalmd5} sumarizes the evaluation results for MD5.

\begin{table}[!ht]
\begin{center}
\begin{tabular}{|l|c|c|c|c|c|c|}
\hline
\textbf{Case study} & \textbf{Module} & \textbf{easy} & \textbf{poly} & \textbf{hard} & \textbf{Total} & \textbf{\%} \\
\hline
\multirow{1}{*}{MD5}
& md5 & 13 & 0 & 0 & 13 & 100\% \\
\hline
\end{tabular}
\end{center}
\caption{Evaluation results for MD5}
\label{tab:evalmd5}
\end{table}

Even tough the MD5 library is quite simple and it was quite easy to type,
we found a little discrepancy between its documentation and its static
typing.
The documentation suggested that the typing of the function \texttt{update}
should be \texttt{(md5\string_context, string) -> (md5\string_context)},
and a comment in the source code suggested that it should be
should be \texttt{(md5\string_context, string, string*) -> (md5\string_context)}.
However, while testing it and reading its source code, we noticed that
its actual type is \texttt{(md5\string_context, string*) -> (md5\string_context)},
that is, we can pass zero or more strings to \texttt{update}.

This case study shows a connection between documentation and software testing.
Even though type annotations are a good source of documentation, this case
study shows that identifying what functions discard extra parameters can be
not so obvious for someone that is typing an external library.
Reading the source code and the library's test script was essential to confirm
the typing of \texttt{update}.

\section{Lua Socket}

\begin{table}[!ht]
\begin{center}
\begin{tabular}{|l|c|c|c|c|c|c|}
\hline
\textbf{Case study} & \textbf{Module} & \textbf{easy} & \textbf{poly} & \textbf{hard} & \textbf{Total} & \textbf{\%} \\
\hline
\multirow{7}{*}{Lua Socket}
& socket & 50 & 3 & 7 & 60 & 83\% \\
\cline{2-7}
& ftp & 6 & 1 & 1 & 8 & 75\% \\
\cline{2-7}
& http & 3 & 0 & 2 & 5 & 60\% \\
\cline{2-7}
& ltn12 & 10 & 10 & 0 & 20 & 50\% \\
\cline{2-7}
& mime & 2 & 8 & 7 & 17 & 12\% \\
\cline{2-7}
& smtp & 6 & 0 & 2 & 8 & 75\% \\
\cline{2-7}
& url & 8 & 0 & 0 & 8 & 100\% \\
\hline
\end{tabular}
\end{center}
\caption{Evaluation results for Lua Socket}
\label{tab:evalsocket}
\end{table}

\section{Related Work}


\chapter{Related Work}
\label{chap:related}
In this chapter we review related work, and we split it into two sections:
in the first section we review other Lua projects,
while in the second section we review other projects that are not related to Lua.

\section{Other Lua projects}

Metalua \cite{metalua} is a Lua compiler that supports compile-time
metaprogramming (CTMP).
CTMP is a kind of macro system that allows the programmers to interact
with the compiler \cite{fleutot2007contrasting}.
Metalua extends Lua 5.1 syntax to include its macro system,
and allows programmers to define their own syntax.
Metalua can provide syntactical support for several object-oriented
styles, and can also provide syntax for turning simple type
annotations into run-time assertions.

MoonScript \cite{moonscript} is a programming language that supports
class-based object-oriented programming.
MoonScript compiles to idiomatic Lua code, but
it does not perform compile-time type checking.

LuaInspect \cite{luainspect} is a tool that uses MetaLua to perform
some code analysis.
For instance, it flags unknown global variables and table fields,
it checks the number of function arguments against signatures, and
it infers function return values.
However, it does not try to analyze object-oriented code and
it does not perform compile-time type checking.

Tidal Lock \cite{tidallock} is a prototype of another optional type
system for Lua, which is written in Metalua.
Tidal Lock covers a little subset of Lua.
Statements include declaration of local variables, multiple assignment,
function application, and the return statement.
This means that Tidal Lock does not include any control-flow statement.
Expressions include primitive literals, table indexing, function application,
function declaration, and the table constructor, but they do not include
binary operations.

A remarkable feature of Tidal Lock is the refinement of table types.
This feature inspired us to also include it in Typed Lua,
but in a simpler way and with different formalization.

The table type from Tidal Lock can only represent records, that is,
it cannot describe hash tables and arrays yet, though we can refine them.
Tidal Lock also includes field types to describe the type of the fields
of a table type.
The field types describe if a table field is mutable or immutable
in a table type.
Field types are the feature that allow the refinement of table types in
Tidal Lock.

Tidal Lock is also a structural type system that relies on subtyping and
local type inference.
However, it does not support union types, recursive types, and variadic types.
It also does not type any object-oriented idiom.

Sol \cite{sol} is an experimental optional type system for Lua.
Its type system is similar to ours, as it includes literal types,
union types, and function types that handle variadic functions.
However, it does not handle the refinement of tables and it
includes different types for tables.
Sol types tables as lists, maps, and objects.
Its object types handle a specific object-oriented idiom that
Sol introduces.

Lua Analyzer \cite{luaanalyzer} is an optional type system for Lua
that is specially designed to work in the Löve Studio,
an IDE for game developing using the Löve framework.
It works in Lua 5.1 only, and uses type annotations inside comments.
It is unsound by design because its dynamic type is both
top and bottom in the subtyping relation.

Lua Analyzer shares some features with Typed Lua, and also
has some interesting features that we do not have in Typed Lua.
It has similar rules for handling the \texttt{or} idiom and
discriminating union types inside conditions.
However, these rules are limited to the \texttt{nil} tag only.
It also includes different types for typing tables.
It includes regular record types that maps names to types,
array types, and map types.
Even though it does not support the refinement of tables,
it allows the definition of nominal table types that simulate classes.
This system allows it to type check custom class systems,
which are common in Lua.
Function types also support multiple return values and
variadic functions, but they do not support overloading the
return type.
Recently, it included experimental support for type aliases and generics.

Luacheck \cite{luacheck} is a tool that performs static analysis on Lua code.
It can flag access to undeclared globals and unused local variables,
but it does not perform static type checking.

Ravi \cite{ravi} is an experimental Lua dialect.
Ravi introduces optional static typing for Lua to improve run-time performance.
To do that, Ravi extends the Lua Virtual Machine to include new
operations that take into account static type information.
Currently, Ravi extends the Lua Virtual Machine to support few types:
\texttt{integer}, \texttt{number}, arrays of integers, and arrays of numbers.

\section{Other projects}

Typed Racket \cite{tobin-hochstadt2008ts} is a statically typed version
of the Racket language, which is a Scheme dialect.
The main purpose of Typed Racket is to allow programmers to combine
untyped modules, which are written in Racket, with typed modules, which are
written in Typed Racket.
It also uses local type inference to deduce the type of unannotated expressions.

The main feature of Typed Racket's type system is \emph{occurrence typing}
\cite{tobin-hochstadt2010ltu}.
It is a novel way to use type predicates in control flow statements
to refine union types.
Occurrence typing is not sound in the presence of mutation.
As these kinds of checks are common in other languages, related systems
have appeared \cite{guha2011tlc,winther2011gtp,pearce2013ccf}.

The type system of Typed Racket also includes function types, recursive
types, and structure types.
Its function types also handle multiple return values, and there is
also a way to describe function types that have optional arguments.
Its structure types are similar to our interfaces, as they describe record types.
The type system is also structural and based on subtyping.
It also includes the dynamic type \texttt{Any}, which is the top type in the system.
Typed Racket also supports polymorphic functions and data structures.

Typed Clojure \cite{bonnaire-sergeant2012typed-clojure} is an
optional type system for Clojure.
Although Clojure is a Lisp dialect that runs on the Java Virtual Machine,
Common Language Runtime, and JavaScript, Typed Clojure runs only on
the Java Virtual Machine.
Perhaps, this restriction pushed Typed Clojure to support Java classes
and some Java types such as \texttt{Long}, \texttt{Double}, and \texttt{String}.
Typed Clojure also provides optional type annotations and uses
local type inference to deduce the type of unannotated expressions.
It also assigns the type \texttt{Any} to unannotated function parameters,
which is the top type in the type system.

The type system of Typed Clojure includes polymorphic function types,
union types, intersection types, lists, vectors, maps, sets, and recursive types.
Function types can also have rest parameters, which are similar
to our variadic types, but can only appear on the input parameter
of function types.
In fact, its function types cannot return multiple results.
It also uses occurrence typing to allow control flow statements to
refine union types.
The type system is also structural and based on subtyping.

Dart \cite{dart} is a new class-based object-oriented programming
language.
It includes optional type annotations and compiles to JavaScript.
The type system of Dart is nominal and includes base types,
function types, lists, and maps.
It also supports generics, and the programmer can define
generic functions, lists, and maps.
Unlike Typed Lua, Dart is unsound by design.

Even though Dart has optional typing and static types by
default do not affect run-time semantics, it has an
execution mode that affects run-time.
The \emph{checked mode} inserts run-time assertions that
verifies whether static types match run-time tags.
The \emph{production mode} is the default execution mode
that does not include any assertions.

TypeScript \cite{typescript} is a JavaScript extension
that includes optional type annotations and class-based
object-oriented programming.
It also uses local type inference to deduce the type
of unannotated expressions.
The type system of TypeScript is structural, based
on subtyping, and supports generics.
It includes the dynamic type, primitive types, union types,
function types, array types, tuple types, recursive types, and
object types.
Unlike Typed Lua, TypeScript uses arrays to represent variadic
functions and multiple return values.

Even though TypeScript is unsound by design,
Bierman et al. \cite{bierman2014typescript} shows how to
make TypeScript sound.
They use a reduced core of TypeScript to formalize a
sound type system for TypeScript, but also to formalize
its current unsound type system.

TeJaS \cite{lerner2013tejas} is a framework for the construction of
different type systems for JavaScript.
The authors created a base type system for JavaScript with
extensible typing rules that allow the experimentation of
different static analysis.
They used TeJaS to create a type system that simulates the
type system of TypeScript.

Politz et al. \cite{politz2012semantics} proposes semantics
and types for objects with first-class member names, a well-known
feature from scripting languages.
Their type system uses string patterns to describe the members of
an object, and define a complex subtyping relation to validate
these patterns.
They also provide an implementation of their system to JavaScript.

Gradualtalk \cite{allende2013gts} is a Smalltalk dialect that
supports gradual typing.
The type system combines nominal and structural typing.
It includes function types, union types, structural types,
nominal types, a self type, and parametric polymorphism.
The type system also relies on subtyping and consistent-subtyping.

Gradualtalk inserts run-time checks that ensure dynamically
typed code does not violate statically typed code.
Allende et al. \cite{allende2013cis} perform a careful
evaluation about cast insertion in Gradualtalk.
They report that usually cast insertions impact on execution
performance, so Gradualtalk also has an option that allows
programmers to turn them off, downgrading Gradualtalk
to an optional type system.

Reticulated Python \cite{vitousek2014deg} is a Python compiler
that supports gradual typing.
The type system is structural and based on subtyping.
It includes base types, the dynamic type, list types,
dictionary types, tuple types, function types, set types,
object types, class types, and recursive types.
It includes class and object types to differentiate the
type of class declarations and instances, respectively.
It also uses local type inference.
Besides static type checking, Reticulated Python also introduces
three different approaches for inserting run-time assertions.

Mypy \cite{mypy} is an optional type system for Python.
The type system of mypy is similar to the type system of
Reticulated Python, but mypy does not insert run-time checks
and it has parametric polymorphism.
In contrast, Reticulated Python can type variadic functions,
but mypy cannot.
Recently, Guido van Rossum, Python's author, proposed a
standard syntax for type annotations in Python \cite{PEP483}
that is extremely inspired by mypy \cite{PEP484}.
The main goal of this proposal is to make easier building
static analysis tools for Python.
Typing \cite{typing} is a tool that is being developed to
implement this proposal.

Hack \cite{hack} is a new programming language that runs on the
Hip Hop Virtual Machine (HHVM).
The HHVM is a virtual machine that executes Hack and PHP programs.
We can view Hack as an extension to PHP that combines static and
dynamic typing.
The type system of Hack includes generics, nullable types, collections,
and function types.

The Ruby Type Checker \cite{ren2013rtc} is a library that
performs type checking during run-time.
The library provides type annotations that the programmer
can use on classes and methods.
Its type system includes nominal types, union types,
intersection types, method types, parametric polymorphism,
and type casts.

Grace \cite{black2013sg} is an object-oriented language
with optional typing.
Grace is not a dynamically typed language that has been
extended with an optional type system, but a language
that has been designed from scratch to have both
static and dynamic typing.
Homer et al. \cite{homer2013modules} explores some
useful patterns that derive from Grace's use of objects as modules
and its brand of optional structural typing, which
can also be expressed with Typed Lua's modules as tables.


\chapter{Conclusions}
\label{chap:conc}
In this work we presented Typed Lua, an optional type system for Lua.
Typed Lua is a Lua extension that allows programmers to combine
static and dynamic typing in Lua code, making easier the evolution
of simple scripts into large programs.

Our main contribution is the formalization of a complete optional type
system that introduces several novel type system features to statically
type check some Lua idioms and features.
Even though Lua shares several features with other dynamically
typed languages such as JavaScript, Lua also has several unusual features.
These unusual features include tables (or associative arrays) as the solely data
structure mechanism, functions with multiple return values, and functions
with flexible arity that interact with multiple assignment.
We highlight the following novel features of our type system:
\begin{itemize}
\item the refinement of table types support the incremental evolution
of record and object types, playing an important role to statically
type check the idiomatic way that Lua programmers use tables to
define modules and objects;
\item projection types handle functions that are overloaded on the
number of return values, allowing programmers to narrow the type of
a local variable that depends on the type of another one;
\item union types and variadic types help our type system handling
functions with flexible arity, that is, union types are helpful to
describe optional parameters while variadic types are helpful to
describe the type of the vararg expression and the type of functions
that can receive or return any number of values.
\end{itemize}

A key feature in optional type systems is usability.
This means that optional type systems should not change the idioms
that programmers are already familiar with.
Instead, optional type system should fit existing idioms to
statically type check them.
In other words, design a simple type system that only relies on
the semantics and run-time tags of a dynamically typed language
can overload programmers, forcing them to code in a different way.
Thus, the most challenging aspect of optional type systems is
to design a complex type system that feels natural to the programmers.

Before starting the design of Typed Lua, we took usability into consideration.
We realized that we should not rely on the semantics of Lua only,
as this could lead to a cumbersome type system that would not support
several idiomatic Lua code.
For this reason, we performed a mostly automated survey of Lua idioms
and features to inform our design choices.

After designing and implementing Typed Lua, we performed several
case studies to evaluate the usability of our type system.
Our evaluation results showed that our type system can statically
type check several Lua idioms and features, though the evaluation
results also exposed several limitations of our type system.
We found that the three main limitations of our type system are
the lack of intersection types, parametric polymorphism, and operator overloading.
Overcoming these limitations is our major target for future work,
as it will allow us to statically type check more programs.

Unlike other optional type systems, we designed Typed Lua without
deliberated unsound parts.
However, we still do not have proofs that the novel features of
our type system are sound.
We see a soundness proof as another major future work, as it is
necessary to use static types for code optimization.

Finally, we believe that Typed Lua is a major contribution to the Lua community,
because it offers a framework that programmers can use to document,
test, and better structure their applications.
For libraries where a full conversion to static type checking should
prove unfeasible or too much work, the community can use Typed Lua
just to document the external interfaces of the libraries,
giving the benefits of static type checking to the users of these
libraries.
In fact, we already have user feedback from Lua programmers that are
using Typed Lua in their projects.
For instance, ZeroBrane Studio is an IDE for Lua development that is
starting to use Typed Lua to perform static analysis in Lua code.



\bibliography{thesis_andre}

\appendix

\chapter{Glossary}
\label{app:glossary}
Should we include Strongtalk, Dart, TypeScript, etc?
\begin{description}
\item[algebraic data type] TO DO.
\item[assertion] TO DO.
\item[associative array] TO DO.
\item[axiom] TO DO.
\item[bottom type] TO DO.
\item[coercion] TO DO.
\item[consistency] TO DO.
\item[consistent-subtyping] TO DO.
\item[contract] TO DO.
\item[contravariant] TO DO.
\item[covariant] TO DO.
\item[deduction system] TO DO.
\item[depth subtyping] TO DO.
\item[downcast] TO DO.
\item[dynamic type] TO DO.
\item[dynamic typing] TO DO.
\item[dynamically typed language] TO DO.
\item[effect system] TO DO.
\item[flow typing] TO DO.
\item[free assigned variable] TO DO.
\item[global type inference] TO DO.
\item[gradual type system] TO DO.
\item[gradual typing] TO DO.
\item[higher-order function] TO DO.
\item[immutable field] TO DO.
\item[invariant] TO DO.
\item[local type inference] TO DO.
\item[metaprogramming] TO DO.
\item[metatable] TO DO.
\item[mutable field] TO DO.
\item[nilable] TO DO.
\item[nominal type system] TO DO.
\item[optional type system] TO DO.
\item[optional typing] TO DO.
\item[pattern matching] TO DO.
\item[prototype object] TO DO.
\item[receiver] TO DO.
\item[referenced variable] TO DO.
\item[run-time check] TO DO.
\item[self-like delegation] TO DO.
\item[sound type system] TO DO.
\item[static analysis] TO DO.
\item[static typing] TO DO.
\item[statically typed language] TO DO.
\item[structural type system] TO DO.
\item[subtype] TO DO.
\item[subtyping] TO DO.
\item[supertype] TO DO.
\item[table constructor] TO DO.
\item[table refinement] TO DO.
\item[top type] TO DO.
\item[type alias] TO DO.
\item[type annotation] TO DO.
\item[type cast] TO DO.
\item[type checker] TO DO.
\item[type checking] TO DO.
\item[type environment] TO DO.
\item[type error] TO DO.
\item[type inference] TO DO.
\item[type safety] TO DO.
\item[type system] TO DO.
\item[type tag] TO DO.
\item[typing relation] TO DO.
\item[unsound type system] TO DO.
\item[upcast] TO DO.
\item[userdata] TO DO.
\item[vararg expression] TO DO.
\item[variadic function] TO DO.
\item[variance] TO DO.
\item[width subtyping] TO DO.
\end{description}


\chapter{The syntax of Typed Lua}
\label{app:syntax}
This appendix presents the complete syntax of Typed Lua.
\allowdisplaybreaks
\begin{align*}
\textit{chunk} & ::= \; \textit{block}\\
\textit{block} & ::= \; \{\textit{stat}\} \; [\textit{retstat}]\\
\textit{stat} & ::= \; \texttt{`;'}\\
& | \; \textit{varlist} \; \texttt{`='} \; \textit{explist}\\
& | \; \textit{functioncall}\\
& | \; \textit{label}\\
& | \; \textbf{break}\\ 
& | \; \textbf{goto} \; \textit{Name}\\
& | \; \textbf{do} \; \textit{block} \; \textbf{end}\\
& | \; \textbf{while} \; \textit{exp} \; \textbf{do} \; \textit{block} \; \textbf{end}\\
& | \; \textbf{repeat} \; \textit{block} \; \textbf{until} \; \textit{exp}\\
& | \; \textbf{if} \; \textit{exp} \; \textbf{then} \; \textit{block} \;
  \{\textbf{elseif} \; \textit{exp} \; \textbf{then} \; \textit{block}\} \;
  [\textbf{else} \; \textit{block}] \; \textbf{end}\\ 
& | \; \textbf{for} \; \textit{Name} \; \texttt{`='} \; \textit{exp} \;
  \texttt{`,'} \; \textit{exp} \; [\texttt{`,'} \; \textit{exp}] \;
  \textbf{do} \; \textit{block} \; \textbf{end}\\
& | \; \textbf{for} \; \textit{namelist} \; \textbf{in} \; \textit{explist} \;
  \textbf{do} \; \textit{block} \; \textbf{end}\\
& | \; [\textbf{const}] \; \textbf{function} \; \textit{funcname} \; \textit{funcbody}\\
& | \; \textbf{local} \; \textbf{function} \; \textit{Name} \; \textit{funcbody}\\
& | \; \textbf{local} \; \textit{namelist} \; [\texttt{`='} \; \textit{explist}]\\
& | \; [\textbf{local}] \; \textbf{interface} \; \textit{Name} \; \textit{interfacedec} \;
  \{\textit{interfacedec}\} \; \textbf{end}\\
\textit{retstat} & ::= \; \textbf{return} \; [\textit{explist}] \; [\texttt{`;'}]\\
\textit{label} & ::= \; \texttt{`::'} \; \textit{Name} \; \texttt{`::'}\\
\textit{funcname} & ::= \; \textit{Name} \; \{\texttt{`.'} \; \textit{Name}\} \; [\texttt{`:'} \; \textit{Name}]\\
\textit{varlist} & ::= \; [\textbf{const}] \; \textit{var} \;
  \{\texttt{`,'} \; [\textbf{const}] \; \textit{var}\}\\
\textit{var} & ::= \; \textit{Name} \; | \;
  \textit{prefixexp} \; \texttt{`['} \; \textit{exp} \; \texttt{`]'} \; | \;
  \textit{prefixexp} \; \texttt{`.'} \; \textit{Name}\\
\textit{namelist} & ::= \; \textit{Name} \; [\texttt{`:'} \; \textit{type}] \;
  \{\texttt{`,'} \; \textit{Name} \; [\texttt{`:'} \; \textit{type}]\}\\
\textit{explist} & ::= \; \textit{exp} \; \{\texttt{`,'} \; \textit{exp}\}\\
\textit{exp} & ::= \; \textbf{nil} \; | \;
  \textbf{false} \; | \;
  \textbf{true} \; | \;
  \textit{Number} \; | \;
  \textit{String} \; | \;
  \texttt{`...'} \; | \;
  \textit{functiondef}\\
& | \; \textit{prefixexp} \; | \;
  \textit{tableconstructor} \; | \;
  \textit{exp} \; \textit{binop} \; \textit{exp} \; | \;
  \textit{unop} \; \textit{exp}\\
\textit{prefixexp} & ::= \; \textit{var} \; | \;
  \textit{functioncall} \; | \;
  \texttt{`('} \; \textit{exp} \; \texttt{`)'}\\
\textit{functioncall} & ::= \; \textit{prefixexp} \; \textit{args} \; | \;
  \textit{prefixexp} \; \texttt{`:'} \; \textit{Name} \; \textit{args}\\
\textit{args} & ::= \; \texttt{`('} \; [\textit{explist}] \; \texttt{`)'} \; | \;
  \textit{tableconstructor} \; | \;
  \textit{String}\\
\textit{functiondef} & ::= \; \textbf{function} \; \textit{funcbody}\\
\textit{funcbody} & ::= \; \texttt{`('} \; [\textit{parlist}] \; \texttt{`)'} \;
  [\texttt{`:'} \; \textit{rettype}] \; \textit{block} \; \textbf{end}\\
\textit{parlist} & ::= \; \textit{namelist} \; [\texttt{`,'} \; \texttt{`...'} \;
  [\texttt{`:'} \; \textit{type}]] \; | \;
  \texttt{`...'} \; [\texttt{`:'} \; \textit{type}]\\
\textit{tableconstructor} & ::= \; \texttt{`\{'} \; [\textit{fieldlist}] \; \texttt{`\}'}\\
\textit{fieldlist} & ::= \; [\textbf{const}] \; \textit{field} \;
  \{\textit{fieldsep} \; [\textbf{const}] \; \textit{field}\} \; [\textit{fieldsep}]\\
\textit{field} & ::= \; \texttt{`['} \; \textit{exp} \; \texttt{`]'} \; \texttt{`='} \; \textit{exp} \; | \;
  \textit{Name} \; \texttt{`='} \; \textit{exp} \; | \;
  \textit{exp}\\
\textit{fieldsep} & ::= \; \texttt{`,'} \; | \; \texttt{`;'}\\
\textit{binop} & ::= \; \texttt{`+'} \; | \; \texttt{`-'} \; | \; \texttt{`*'} \; | \; \texttt{`/'} \; | \;
  \texttt{`\textasciicircum'} \; | \; \texttt{`\%'} \; | \; \texttt{`..'}\\
& | \; \texttt{`<'} \; | \; \texttt{`<='} \; | \; \texttt{`>'} \; | \; \texttt{`>='} \; | \;
  \texttt{`=='} \; | \; \texttt{`\textasciitilde='}\\
& | \; \textbf{and} \; | \; \textbf{or}\\
\textit{unop} & ::= \; \texttt{`-'} \; | \; \textbf{not} \; | \; \texttt{`\#'}\\
\textit{interfacedec} & ::= \; [\textbf{const}] \; \textit{Name} \;
  \{\texttt{`,'} \; [\textbf{const}] \; \textit{Name}\} \; \texttt{`:'} \; \textit{dectype}\\
\textit{dectype} & ::= \; \textit{type} \; | \; \textit{methodtype}\\
\textit{type} & ::= \; \textit{primarytype} \; [\texttt{`?'}]\\
\textit{primarytype} & ::= \; \textit{literaltype} \; | \;
  \textit{basetype} \; | \;
  \textbf{nil} \; | \;
  \textbf{value} \; | \;
  \textbf{any} \; | \;
  \textbf{self} \; | \;
  \textit{Name}\\
& | \; \textit{functiontype} \; | \;
  \textit{tabletype} \; | \;
  \textit{primarytype} \; \texttt{`|'} \; \textit{primarytype}\\
\textit{literaltype} & ::= \; \textbf{false} \; | \;
  \textbf{true} \; | \;
  \textit{Number} \; | \;
  \textit{String}\\
\textit{basetype} & ::= \; \textbf{boolean} \; | \;
  \textbf{number} \; | \;
  \textbf{string}\\
\textit{functiontype} & ::= \; \textit{tupletype} \; \texttt{`->'} \; \textit{rettypelist}\\
\textit{methodtype} & ::= \; \textit{tupletype} \; \texttt{`=>'} \; \textit{rettypelist}\\
\textit{tabletype} & ::= \; \texttt{`\{'} \; [\textit{tabletypebody}] \; \texttt{`\}'}\\
\textit{tupletype} & ::= \; \texttt{`('} \; [typelist] \; \texttt{`)'}\\
\textit{typelist} & ::= \; \textit{type} \; \{\texttt{`,'} \; \textit{type}\} \; [\texttt{`*'}]\\
\textit{rettypelist} & ::= \; \textit{unionlist} \; [\texttt{`?'}]\\
\textit{unionlist} & ::= \; \textit{tupletype} \; | \;
  \textit{unionlist} \; \texttt{`|'} \; \textit{unionlist}\\
\textit{tabletypebody} & ::= \; \textit{type} \; | \; \textit{fieldtypelist}\\
\textit{fieldtypelist} & ::= \; [\textbf{const}] \; \textit{fieldtype} \; \{\texttt{`,'} \; [\textbf{const}] \; \textit{fieldtype}\}\\ 
\textit{fieldtype} & ::= \; \textit{keytype} \; \texttt{`:'} \; \textit{type}\\
\textit{keytype} & ::= \; \textit{literaltype} \; | \;
  \textit{basetype} \; | \;
  \textbf{value} \; | \;
  \textbf{any}\\
\textit{rettype} & ::= \; \textit{type} \; | \; \textit{rettypelist}\\
\end{align*}


\chapter{The type system of Typed Lua}
\label{app:rules}
This appendix presents the complete set of typing rules of Typed Lua.
\\

\mylabel{T-SKIP}
\[
\env_{1} \vdash \mathbf{skip}:\env_{1}
\]

\mylabel{T-SEQ}
\[
\dfrac{\env_{1} \vdash s_{1}:\env_{2} \;\;\;
       \env_{2} \vdash s_{2}:\env_{3}}
      {\env_{1} \vdash s_{1} \; ; \; s_{2}:\env_{3}}
\]

\mylabel{T-ASSIGNMENT}
\[
\dfrac{\env_{1} \vdash el:r_{1}, \env_{2} \;\;\;
       \env_{2} \vdash \vec{l}:r_{2}, \env_{3} \;\;\;
       r_{1} \lesssim r_{2}}
      {\env_{1} \vdash \vec{l} = el:\env_{3}}
\]

\mylabel{T-WHILE}
\[
\dfrac{\env_{1} \vdash e:t, \env_{2} \;\;\;
       closeall(\env_{2}) \vdash s:\env_{3}}
      {\env_{1} \vdash \mathbf{while} \; e \; \mathbf{do} \; s:closeset(\env_{2}, fav(s))}
\]

\mylabel{T-IF1}
\[
\dfrac{\env_{1} \vdash e:t, \env_{2} \;\;\;
       closeall(\env_{2}) \vdash s_{1}:\env_{3} \;\;\;
       closeall(\env_{2}) \vdash s_{2}:\env_{4}}
      {\env_{1} \vdash \mathbf{if} \; e \; \mathbf{then} \; s_{1} \; \mathbf{else} \; s_{2}:closeset(\env_{2}, fav(s_{1}) \cup fav(s_{2}))}
\]

\mylabel{T-IF2}
\[
\dfrac{\begin{array}{c}
       \env_{1} \vdash type(n) == ``string":\Boolean, \env_{2} \\
       closeall(\env_{2}[n \mapsto \String]) \vdash s_{1}:\env_{3} \\
       closeall(\env_{2}[n \mapsto filter(\env_{2}(n), \String)) \vdash s_{2}:\env_{4}
      \end{array}}
      {\env_{1} \vdash \mathbf{if} \; type(n) == ``string" \; \mathbf{then} \; s_{1} \; \mathbf{else} \; s_{2}:closeset(\env_{2}, fav(s_{1}) \cup fav(s_{2}))}
\]

\mylabel{T-LOCAL}
\[
\dfrac{\env_{1} \vdash el:r_{1}, \env_{2} \;\;\;
       r_{1} \lesssim \vec{t} \;\;\;
       \env_{2}[\vec{n} \mapsto \vec{t}] \vdash s:\env_{3}}
      {\env_{1} \vdash \mathbf{local} \; \vec{n{:}t} = el \; \mathbf{in} \; s:\env_{3} - \{\vec{n} \mapsto \vec{t}\}}
\]

\mylabel{T-LOCALREC}
\[
\dfrac{\env_{1}[n \mapsto t] \vdash f:t_{1}, \env_{2} \;\;\;
       t_{1} \lesssim t \;\;\;
       \env_{2} \vdash s:\env_{3}}
      {\env_{1} \vdash \mathbf{rec} \; n{:}t = f \; \mathbf{in} \; s:\env_{3} - \{n \mapsto t\}}
\]

\mylabel{T-RETURN}
\[
\dfrac{\env_{1} \vdash el:r_{1}, \env_{2} \;\;\;
       \env_{2}(\ret) = r_{2} \;\;\;
       r_{1} \lesssim r_{2}}
      {\env_{1} \vdash \mathbf{return} \; el:\env_{2}}
\]

\mylabel{T-STMCALL1}
\[
\dfrac{\env_{1} \vdash e:p_{1} \rightarrow r_{1}, \env_{2} \;\;\;
       \env_{2} \vdash el:p_{2}, \env_{3} \;\;\;
       p_{2} \lesssim p_{1}}
      {\env_{1} \vdash e(el)_{s}:\env_{3}}
\]

\mylabel{T-STMCALL2}
\[
\dfrac{\env_{1} \vdash e:\Any, \env_{2} \;\;\;
       \env_{2} \vdash el:p, \env_{3}}
      {\env_{1} \vdash e(el)_{s}:\env_{3}}
\]

\mylabel{T-STMINVOKE1}
\[
\dfrac{\env_{1} \vdash e[n]:p_{1} \rightarrow r_{1}, \env_{2} \;\;\;
       \env_{2} \vdash el:p_{2}, \env_{3} \;\;\;
       p_{2} \lesssim p_{1}}
      {\env_{1} \vdash e{:}n(el)_{s}:\env_{3}}
\]

\mylabel{T-STMINVOKE2}
\[
\dfrac{\env_{1} \vdash e[n]:\Any, \env_{2} \;\;\;
       \env_{2} \vdash el:p, \env_{3}}
      {\env_{1} \vdash e{:}n(el)_{s}:\env_{3}}
\]

\mylabel{T-NIL}
\[
\env_{1} \vdash \mathbf{nil}:\Nil, \env_{1}
\]

\mylabel{T-FALSE}
\[
\env_{1} \vdash \mathbf{false}:\False, \env_{1}
\]

\mylabel{T-TRUE}
\[
\env_{1} \vdash \mathbf{true}:\True, \env_{1}
\]

\mylabel{T-INT}
\[
\env_{1} \vdash {\it int}:{\it int}, \env_{1}
\]

\mylabel{T-FLOAT}
\[
\env_{1} \vdash {\it float}:{\it float}, \env_{1}
\]

\mylabel{T-STR}
\[
\env_{1} \vdash {\it string}:{\it string}, \env_{1}
\]

\mylabel{T-EXPDOTS}
\[
\dfrac{\env_{1}({...}) = t}
      {\env_{1} \vdash {...}_{e}:t, \env_{1}}
\]

\mylabel{T-FUNCTION1}
\[
\dfrac{closeall(\env_{1}[\ret \mapsto r]) \vdash s:\env_{2}}
      {\env_{1} \vdash \mathbf{fun} \; (){:}r \; s:\Void \rightarrow r, closeset(\env_{1}, fav(\mathbf{fun} \; (){:}r \; s))}
\]

\mylabel{T-FUNCTION2}
\[
\dfrac{closeall(\env_{1}[{...} \mapsto t, \ret \mapsto r]) \vdash s:\env_{2}}
      {\env_{1} \vdash \mathbf{fun} \; ({...}{:}t){:}r \; s:t{*} \rightarrow r, closeset(\env_{1}, fav(\mathbf{fun} \; ({...}{:}t){:}r \; s))}
\]

\mylabel{T-FUNCTION3}
\[
\dfrac{closeall(\env_{1}[\vec{n} \mapsto \vec{t}, \ret \mapsto r]) \vdash s:\env_{2}}
      {\env_{1} \vdash \mathbf{fun} \; (\vec{n{:}t}){:}r \; s:\vec{t} \rightarrow r, closeset(\env_{1}, fav(\mathbf{fun} \; (\vec{n{:}t}){:}r \; s))}
\]

\mylabel{T-FUNCTION4}
\[
\dfrac{closeall(\env_{1}[\vec{n} \mapsto \vec{t}, {...} \mapsto t, \ret \mapsto r]) \vdash s:\env_{2}}
      {\env_{1} \vdash \mathbf{fun} \; (\vec{n{:}t},{...}{:}t){:}r \; s: \vec{t} \times t{*} \rightarrow r, closeset(\env_{1}, fav(\mathbf{fun} \; (\vec{n{:}t},{...}{:}t){:}r \; s))}
\]

\mylabel{T-CONSTRUCTOR1}
\[
\env \vdash \{ \mathbf{nothing} \}:\{ \}_{u}, \env 
\]

\mylabel{T-CONSTRUCTOR2}
\[
\dfrac{\env_{1} \vdash cl:k_{i}{:}v_{i}, ..., k_{n}{:}v_{n}, \env_{2}}
      {\env_{1} \vdash \{ \; cl \; \}:\{ k_{i}{:}v_{i}, ..., k_{n}{:}v_{n} \}_{u}, \env_{2}}
\]

\mylabel{T-ARITH1}
\[
\dfrac{\env_{1} \vdash e_{1}:t_{1}, \env_{2} \;\;\;
       \env_{2} \vdash e_{2}:t_{2}, \env_{3} \;\;\;
       t_{1} \subtype \Integer \;\;\;
       t_{2} \subtype \Integer}
      {\env_{1} \vdash e_{1} + e_{2}:\Integer, \env_{3}}
\]

\mylabel{T-ARITH2}
\[
\dfrac{\env_{1} \vdash e_{1}:t_{1}, \env_{2} \;\;\;
       \env_{2} \vdash e_{2}:t_{2}, \env_{3} \;\;\;
       t_{1} \subtype \Integer \;\;\;
       t_{2} \subtype \Number}
      {\env_{1} \vdash e_{1} + e_{2}:\Number, \env_{3}}
\]

\mylabel{T-ARITH3}
\[
\dfrac{\env_{1} \vdash e_{1}:t_{1}, \env_{2} \;\;\;
       \env_{2} \vdash e_{2}:t_{2}, \env_{3} \;\;\;
       t_{1} \subtype \Number \;\;\;
       t_{2} \subtype \Integer}
      {\env_{1} \vdash e_{1} + e_{2}:\Number, \env_{3}}
\]

\mylabel{T-ARITH4}
\[
\dfrac{\env_{1} \vdash e_{1}:t_{1}, \env_{2} \;\;\;
       \env_{2} \vdash e_{2}:t_{2}, \env_{3} \;\;\;
       t_{1} \subtype \Number \;\;\;
       t_{2} \subtype \Number}
      {\env_{1} \vdash e_{1} + e_{2}:\Number, \env_{3}}
\]

\mylabel{T-ARITH5}
\[
\dfrac{\env_{1} \vdash e_{1}:\Any, \env_{2} \;\;\;
       \env_{2} \vdash e_{2}:t, \env_{3}}
      {\env_{1} \vdash e_{1} + e_{2}:\Any, \env_{3}}
\]

\mylabel{T-ARITH6}
\[
\dfrac{\env_{1} \vdash e_{1}:t, \env_{2} \;\;\;
       \env_{2} \vdash e_{2}:\Any, \env_{3}}
      {\env_{1} \vdash e_{1} + e_{2}:\Any, \env_{3}}
\]

\mylabel{T-CONCAT1}
\[
\dfrac{\env_{1} \vdash e_{1}:t_{1}, \env_{2} \;\;\;
       \env_{2} \vdash e_{2}:t_{2}, \env_{3} \;\;\;
       t_{1} \subtype \String \;\;\;
       t_{2} \subtype \String}
      {\env_{1} \vdash e_{1} \; {..} \; e_{2}:\String, \env_{3}}
\]

\mylabel{T-CONCAT2}
\[
\dfrac{\env_{1} \vdash e_{1}:\Any, \env_{2} \;\;\;
       \env_{2} \vdash e_{2}:t, \env_{3}}
      {\env_{1} \vdash e_{1} \; {..} \; e_{2}:\Any, \env_{3}}
\]

\mylabel{T-CONCAT3}
\[
\dfrac{\env_{1} \vdash e_{1}:t, \env_{2} \;\;\;
       \env_{2} \vdash e_{2}:\Any, \env_{3}}
      {\env_{1} \vdash e_{1} \; {..} \; e_{2}:\Any, \env_{3}}
\]

\mylabel{T-EQUAL}
\[
\dfrac{\env_{1} \vdash e_{1}:t_{1}, \env_{2} \;\;\;
       \env_{2} \vdash e_{2}:t_{2}, \env_{3}}
      {\env_{1} \vdash e_{1} == e_{2}:\Boolean, \env_{3}}
\]

\mylabel{T-ORDER1}
\[
\dfrac{\env_{1} \vdash e_{1}:t_{1}, \env_{2} \;\;\;
       \env_{2} \vdash e_{2}:t_{2}, \env_{3} \;\;\;
       t_{1} \subtype \Number \;\;\;
       t_{2} \subtype \Number}
      {\env \vdash e_{1} < e_{2}:\Boolean, \env_{3}}
\]

\mylabel{T-ORDER2}
\[
\dfrac{\env_{1} \vdash e_{1}:t_{1}, \env_{2} \;\;\;
       \env_{2} \vdash e_{2}:t_{2}, \env_{3} \;\;\;
       t_{1} \subtype \String \;\;\;
       t_{2} \subtype \String}
      {\env_{1} \vdash e_{1} < e_{2}:\Boolean}
\]

\mylabel{T-ORDER3}
\[
\dfrac{\env_{1} \vdash e_{1}:\Any, \env_{2} \;\;\;
       \env_{2} \vdash e_{2}:t, \env_{3}}
      {\env_{1} \vdash e_{1} < e_{2}:\Any, \env_{3}}
\]

\mylabel{T-ORDER4}
\[
\dfrac{\env_{1} \vdash e_{1}:t, \env_{2} \;\;\;
       \env_{2} \vdash e_{2}:\Any, \env_{3}}
      {\env_{1} \vdash e_{1} < e_{2}:\Any, \env_{3}}
\]

\mylabel{T-AND1}
\[
\dfrac{\env_{1} \vdash e_{1}:\Nil, \env_{2} \;\;\;
       \env_{2} \vdash e_{2}:t, \env_{3}}
      {\env_{1} \vdash e_{1} \; \mathbf{and} \; e_{2}:\Nil, \env_{3}}
\]

\mylabel{T-AND2}
\[
\dfrac{\env_{1} \vdash e_{1}:\False, \env_{2} \;\;\;
       \env_{2} \vdash e_{2}:t, \env_{3}}
      {\env_{1} \vdash e_{1} \; \mathbf{and} \; e_{2}:\False, \env_{3}}
\]

\mylabel{T-AND3}
\[
\dfrac{\env_{1} \vdash e_{1}:t_{1} \cup \Nil, \env_{2} \;\;\;
       \env_{2} \vdash e_{2}:t_{2}, \env_{3}}
      {\env_{1} \vdash e_{1} \; \mathbf{and} \; e_{2}:\Nil \cup t_{2}, \env_{3}}
\]

\mylabel{T-AND4}
\[
\dfrac{\env_{1} \vdash e_{1}:t_{1} \cup \False, \env_{2} \;\;\;
       \env_{2} \vdash e_{2}:t_{2}, \env_{3}}
      {\env_{1} \vdash e_{1} \; \mathbf{and} \; e_{2}:\False \cup t_{2}, \env_{3}}
\]

\mylabel{T-AND5}
\[
\dfrac{\env_{1} \vdash e_{1}:t_{1}, \env_{2} \;\;\;
       \env_{2} \vdash e_{2}:t_{2}, \env_{3}}
      {\env_{1} \vdash e_{1} \; \mathbf{and} \; e_{2}:t_{1} \cup t_{2}, \env_{3}}
\]

\mylabel{T-OR1}
\[
\dfrac{\env_{1} \vdash e_{1}:\Nil, \env_{2} \;\;\;
       \env_{2} \vdash e_{2}:t, \env_{3}}
      {\env_{1} \vdash e_{1} \; \mathbf{or} \; e_{2}:t, \env_{3}}
\]

\mylabel{T-OR2}
\[
\dfrac{\env_{1} \vdash e_{1}:\False, \env_{2} \;\;\;
       \env_{2} \vdash e_{2}:t, \env_{3}}
      {\env_{1} \vdash e_{1} \; \mathbf{or} \; e_{2}:t, \env_{3}}
\]

\mylabel{T-OR3}
\[
\dfrac{\env_{1} \vdash e_{1}:t_{1} \cup \Nil, \env_{2} \;\;\;
       \env_{2} \vdash e_{2}:t_{2}, \env_{3}}
      {\env_{1} \vdash e_{1} \; \mathbf{or} \; e_{2}:t_{1} \cup t_{2}, \env_{3}}
\]

\mylabel{T-OR4}
\[
\dfrac{\env_{1} \vdash e_{1}:t_{1} \cup \False, \env_{2} \;\;\;
       \env_{2} \vdash e_{2}:t_{2}, \env_{3}}
      {\env_{1} \vdash e_{1} \; \mathbf{or} \; e_{2}:t_{1} \cup t_{2}, \env_{3}}
\]

\mylabel{T-OR5}
\[
\dfrac{\env_{1} \vdash e_{1}:t_{1}, \env_{2} \;\;\;
       \env_{2} \vdash e_{2}:t_{2}, \env_{3}}
      {\env_{1} \vdash e_{1} \; \mathbf{or} \; e_{2}:t_{1} \cup t_{2}, \env_{3}}
\]

\mylabel{T-NOT1}
\[
\dfrac{\env_{1} \vdash e:\Nil, \env_{2}}
      {\env_{1} \vdash \mathbf{not} \; e:\True, \env_{2}}
\]

\mylabel{T-NOT2}
\[
\dfrac{\env_{1} \vdash e:\False, \env_{2}}
      {\env_{1} \vdash \mathbf{not} \; e:\True, \env_{2}}
\]

\mylabel{T-NOT3}
\[
\dfrac{\env_{1} \vdash e:t, \env_{2}}
      {\env_{1} \vdash \mathbf{not} \; e:\Boolean, \env_{2}}
\]

\mylabel{T-MINUS1}
\[
\dfrac{\env_{1} \vdash e:t, \env_{2} \;\;\;
       t \subtype \Integer}
      {\env_{1} \vdash - e:\Integer, \env_{2}}
\]

\mylabel{T-MINUS2}
\[
\dfrac{\env_{1} \vdash e:t, \env_{2} \;\;\;
       t \subtype \Number}
      {\env_{1} \vdash - e:\Number, \env_{2}}
\]

\mylabel{T-MINUS3}
\[
\dfrac{\env_{1} \vdash e:\Any, \env_{2}}
      {\env_{1} \vdash - e:\Any, \env_{2}}
\]

\mylabel{T-LEN1}
\[
\dfrac{\env_{1} \vdash e:t, \env_{2} \;\;\;
       t \subtype \String}
      {\env_{1} \vdash \# \; e:\Integer, \env_{2}}
\]

\mylabel{T-LEN2}
\[
\dfrac{\env_{1} \vdash e:t, \env_{2} \;\;\;
       t \subtype \{\}_{c}}
      {\env_{1} \vdash \# \; e:\Integer, \env_{2}}
\]

\mylabel{T-LEN3}
\[
\dfrac{\env_{1} \vdash e:\Any, \env_{2}}
      {\env_{1} \vdash \# \; e:\Any, \env_{2}}
\]

\mylabel{T-EXPAPPLY1}
\[
\dfrac{\env_{1} \vdash e:p_{1} \rightarrow r, \env_{2} \;\;\;
       \env_{2} \vdash el:p_{2}, \env_{3} \;\;\;
       p_{2} \lesssim p_{1}}
      {\env_{1} \vdash e(el)_{e}:first(r), \env_{3}}
\]

\mylabel{T-EXPAPPLY2}
\[
\dfrac{\env_{1} \vdash e:\Any, \env_{2} \;\;\;
       \env_{2} \vdash el:p, \env_{3}}
      {\env_{1} \vdash e_{1}(\vec{e_{2}})_{e}:\Any, \env_{3}}
\]

\mylabel{T-EXPINVOKE1}
\[
\dfrac{\env_{1} \vdash e[n]:p_{1} \rightarrow r, \env_{2} \;\;\;
       \env_{2} \vdash el:p_{2}, \env_{3} \;\;\;
       p_{2} \lesssim p_{1}}
      {\env_{1} \vdash e{:}n(el)_{e}:first(r), \env_{3}}
\]

\mylabel{T-EXPINVOKE2}
\[
\dfrac{\env_{1} \vdash e[n]:\Any, \env_{2} \;\;\;
       \env_{2} \vdash el:p, \env_{3}}
      {\env_{1} \vdash e{:}n(el)_{e}:\Any, \env_{3}}
\]

\mylabel{T-CAST}
\[
\dfrac{t \subtype \env_{1}(n)}
      {\env_{1} \vdash {<}t{>} \; n:t, \env_{1}[n \mapsto t]}
\]

\mylabel{T-SELF}
\[
\dfrac{\env_{1} \vdash e:\Self, \env_{2} \;\;\;
       \env_{2}(\Self) = t}
      {\env_{1} \vdash e:t, \env_{2}}
\]

\mylabel{T-UNFOLD}
\[
\dfrac{\env_{1} \vdash e:\mu x.t, \env_{2}}
      {\env_{1} \vdash e:[x \mapsto \mu x.t]t, \env_{2}}
\]

\mylabel{T-FOLD}
\[
\dfrac{\env_{1} \vdash e:[x \mapsto \mu x.t]t, \env_{2}}
      {\env_{1} \vdash e:\mu x.t, \env_{2}}
\]

\mylabel{T-TERNARY}
\[
\dfrac{\env_{1} \vdash e_{1}:t_{1}, \env_{2} \;\;\;
       \env_{2} \vdash e_{2}:t_{2}, \env_{3} \;\;\;
       \env_{3} \vdash e_{3}:t_{2}, \env_{4}}
      {\env_{1} \vdash e_{1} \; \mathbf{and} \; e_{2} \; \mathbf{or} \; e_{3}:t_{2}, \env_{4}}
\]

\mylabel{T-ID}
\[
\dfrac{\env_{1}(n) = t}
      {\env_{1} \vdash n:t, \env_{1}}
\]

\mylabel{T-INDEX1}
\[
\dfrac{\env_{1} \vdash e_{1}:\{k_{1}{:}v_{1}, ..., k_{n}{:}v_{n}\}, \env_{1} \;\;\;
       \env_{2} \vdash e_{2}:t, \env_{3} \;\;\;
       \exists i \in 1{..}n \; t \lesssim k_{i}}
      {\env_{1} \vdash e_{1}[e_{2}]:v_{i}, \env_{3}}
\]

\mylabel{T-INDEX2}
\[
\dfrac{\env_{1} \vdash e_{1}:\Any, \env_{2} \;\;\;
       \env_{2} \vdash e_{2}:t, \env_{3}}
      {\env_{1} \vdash e_{1}[e_{2}]:\Any, \env_{3}}
\]

\mylabel{T-REFINE}
\[
\dfrac{\env_{1}(n) = \{ k_{1}{:}v_{1}, ..., k_{n}{:}v_{n} \}_{o|u} \;\;\;
       \env_{1} \vdash e:t_{1}, \env_{2} \;\;\;
       \not \exists i \in 1..n \; t_{1} \lesssim k_{i}}
      {\env_{1} \vdash n[e] {<}t{>}:t, \env_{2}[n \mapsto \{ k_{1}{:}v_{1}, ..., k_{n}{:}v_{n}, t_{1}{:}t\}_{o|u}]}
\]

\mylabel{T-EXPLIST1}
\[
\env \vdash \mathbf{nothing}:\Nil{*}, \env
\]

\mylabel{T-EXPLIST2}
\[
\dfrac{\env \vdash e_{k}:t_{k}, \env_{k} \;\;\;
       n = |\vec{e}|}
      {\env \vdash \vec{e}:t_{1} \times ... \times t_{n}, merge(\env_{1}, ..., \env_{n})}
\]

\mylabel{T-EXPLIST3}
\[
\dfrac{\env \vdash me:r, \env_{1}}
      {\env \vdash me:r, \env_{1}}
\]

\mylabel{T-EXPLIST2}
\[
\dfrac{\env \vdash e_{k}:t_{k}, \env_{k} \;\;\;
       \env \vdash me:r, \env_{n + 1} \;\;\;
       n = |\vec{e}|}
      {\env \vdash \vec{e}, me:t_{1} \times ... \times t_{n} \times r, merge(\env_{1}, ..., \env_{n+1})}
\]

\mylabel{T-APPLY1}
\[
\dfrac{\env_{1} \vdash e:p_{1} \rightarrow r, \env_{2} \;\;\;
       \env_{2} \vdash el:p_{2}, \env_{3} \;\;\;
       p_{2} \lesssim p_{1}}
      {\env_{1} \vdash e(el):r, \env_{3}}
\]

\mylabel{T-APPLY2}
\[
\dfrac{\env_{1} \vdash e:\Any, \env_{2} \;\;\;
       \env_{2} \vdash el:p, \env_{3}}
      {\env_{1} \vdash e_{1}(\vec{e_{2}}):\Any{*}, \env_{3}}
\]

\mylabel{T-INVOKE1}
\[
\dfrac{\env_{1} \vdash e[n]:p_{1} \rightarrow r, \env_{2} \;\;\;
       \env_{2} \vdash el:p_{2}, \env_{3} \;\;\;
       p_{2} \lesssim p_{1}}
      {\env_{1} \vdash e{:}n(el):r, \env_{3}}
\]

\mylabel{T-INVOKE2}
\[
\dfrac{\env_{1} \vdash e[n]:\Any, \env_{2} \;\;\;
       \env_{2} \vdash el:p, \env_{3}}
      {\env_{1} \vdash e{:}n(el):\Any{*}, \env_{3}}
\]

\mylabel{T-DOTS}
\[
\dfrac{\env_{1}({...}) = t}
      {\env_{1} \vdash {...}:t{*}, \env_{1}}
\]



\end{document}
