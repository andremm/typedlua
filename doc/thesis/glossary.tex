\begin{description}
\item[bottom type] It is a type that is subtype of all types.

\item[closed table type] It is a table type that does not allow adding fields.

\item[coercion] It is a relation that allows converting values from one type to values
of another type without error.

\item[consistency] It is a relation that gradual typing uses to control the insertion of
run-time checks between dynamically typed and statically typed code.

\item[consistent-subtyping] It is a relation that combines consistency and subtyping.

\item[contravariant] Subtyping is contravariant when it reverses the subtyping order,
that is, it orders types from a more generic type to a more specific type.

\item[covariant] Subtyping is covariant when it preserves the subtyping order,
that is, it orders types from a more specific type to a more general type.

\item[depth subtyping] It allows changing the type of individual fields of a table type.

\item[dynamic type] It is a type that allows combining dynamic and static typing in the same code.
It is neither the bottom nor the top type in the subtyping relation, but
a subtype only of itself.
Gradual typing uses the dynamic type along with the consistency relation to
identify where run-time casts should be inserted to prevent that dynamically
typed code violates statically typed code.

\item[filter type] It is a type that allows Typed Lua to discriminate the type of local variables
inside control flow statements.

\item[fixed table type] It is a table type that does not allow adding fields. It also does not allow
width subtyping to make single inheritance safe.

\item[flow typing] It is a combination of static typing and flow analysis.

\item[free assigned variable] It is a free variable that appears in an assignment.

\item[gradual type system] It is a type system that uses the consistency relation to check the interaction
among the dynamic type and other types.

\item[gradual typing] It is an approach that uses a gradual type system to allow static and dynamic
typing in the same code, but inserting run-time checks between statistically
typed and dynamically typed code.

\item[invariant] Subtyping is invariant when it does not allow ordering types, that is,
it is a way that express type equality through subtyping.

\item[metatable] It is a Lua table that allows changing its behavior.

\item[nominal type system] It is a type system that uses name declarations to define the equivalence among types.

\item[open table type] It is a table type that allows adding fields, and usually refers to table types that
have aliases.

\item[optional type system] It is a type system that allows combining static and dynamic typing in the same
language, but without affecting the run-time semantics.

\item[projection environment] It is a type environment that Typed Lua uses to assign projection variables to
second-level types.

\item[projection type] It is a type that allows Typed Lua to discriminate the type of local variables
that have a dependency relation.

\item[prototype object] It is an object that works like a class, that is, it is a prototype for creating
new objects of a given class.

\item[self-like delegation] It is a technique to implement inheritance in dynamically typed languages.

\item[sound type system] It is a type system that does not type check all programs that contain a type error.

\item[structural type system] It is a type system that uses the type structure to define the equivalence among types.

\item[table refinement] It is an operation that allows changing a table type to include new fields or
to specialize existing fields.

\item[top type] It is a type that is supertype of all types.

\item[type environment] It is an environment that assigns variable names to types.

\item[type tag] In dynamically typed languages a type tag describes the type of a value during
run-time.

\item[unique table type] It is a table type that allows adding fields, and it is also the table type
that describe the type of the table constructor.

\item[unsound type system] It is a type system that type checks certain programs that contain type errors.

\item[userdata] It is a way to define new types in an application or library that is written in C.

\item[vararg expression] It is a Lua expression that can result in an arbitrary number of values.

\item[variadic function] It is a Lua function that can receive an arbitrary number of arguments.

\item[variance] It is the way types are ordered.

\item[width subtyping] It allows adding more fields to a table type.

\end{description}
