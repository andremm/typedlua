Dynamically typed languages such as Lua avoid static types in favor of
simplicity and flexibility, because the absence of static types means
that programmers do not need to bother with abstracting types that
should be validated by a type checker.
Instead, dynamically typed languages use run-time \emph{type tags}
to classify the values they compute, so their implementation can use
these tags to perform run-time (or dynamic) type checking
\cite{pierce2002tpl}.

This simplicity and flexibility allows programmers to write code that
might require a complex type system to statically type check,
though it may also hide bugs that will be caught only after deployment
if programmers do not properly test their code.
In contrast, static type checking helps programmers detect many
bugs during the development phase.
Static types also provide a conceptual framework that helps
programmers define modules and interfaces that can be combined to
structure the development of programs.

Thus, early error detection and better program structure are two
advantages of static type checking that can lead programmers to
migrate their code from a dynamically typed to a statically
typed language, when their simple scripts evolve into complex programs
\cite{tobin-hochstadt2006ims}.
Dynamically typed languages certainly help programmers during the
beginning of a project, because their simplicity and flexibility
allows quick development and makes it easier to change code according to
changing requirements.
However, programmers tend to migrate from dynamically typed to
statically typed code as soon as the project has consolidated its
requirements, because the robustness of static types helps
programmers link requirements to abstractions.
This migration usually involves different languages that have
different syntaxes and semantics, which usually requires a complete
rewrite of existing programs instead of incremental evolution from
dynamic to static types.

Ideally, programming languages should offer programmers the
option to choose between static and dynamic typing:
\emph{optional type systems} \cite{bracha2004pluggable} and
\emph{gradual typing} \cite{siek2006gradual} are two similar
approaches for blending static and dynamic typing in the same
language.
The aim of both approaches is to offer programmers the option
to use type annotations where static typing is needed,
allowing the incremental migration from dynamic to static typing.
The difference between these two approaches is the way they treat
run-time semantics.
While optional type systems do not affect run-time semantics,
gradual typing uses run-time checks to ensure that dynamically typed
code does not violate the invariants of statically typed code.

Programmers and researchers sometimes use the term \emph{gradual typing}
to mean the incremental evolution of dynamically typed code into
statically typed code.
For this reason, gradual typing may also refer to optional type
systems and other approaches that blend static and dynamic typing to
help programmers incrementally migrate from dynamic to static typing
without having to switch to a different language, though all these
approaches differ in the way they handle static and dynamic typing
together.
We use the term \emph{gradual typing} to refer to the work of
Siek and Taha \cite{siek2006gradual}.

In this work we present the design and evaluation of Typed Lua:
an optional type system for Lua that is rich enough to
preserve some of the Lua idioms that programmers are already familiar with,
but that also includes new constructs that help programmers
structure Lua programs.

Lua is a small imperative language with first-class functions
(with proper lexical scoping) where the only data structure
mechanism is the \emph{table} --
an associative array that can represent arrays, records, maps, modules, objects, etc.
Tables also have syntactic sugar and metaprogramming support
through operator overloading built into the language.
Unlike other scripting languages, Lua has very limited coercion
among different data types.

Lua prefers to provide mechanisms instead of fixed policies due
to its primary use as an embedded language for configuration and
extension of other applications.
This means that even features such as a module system and
object orientation are a matter of convention instead of default
language constructs.
The result is a fragmented ecosystem of libraries, and different
ideas among Lua programmers on how they should use the language
features, or how they should structure programs.

The lack of standard policies is a challenge for the design of
an optional type system for Lua.
For this reason, we are not relying entirely on the semantics of
the language to design our type system.
We also run a mostly automated survey of Lua idioms used in a
large corpus of Lua libraries, which also has helped in the design of Typed Lua.

So far, Typed Lua is a Lua extension that allows statically typed
code to coexist and interact with dynamically typed code
through optional type annotations.
In addition, it adds default constructs that programmers can use
to better structure Lua programs.
The Typed Lua compiler warns programmers about type errors,
but always generates Lua code that runs in unmodified Lua implementations.
Programmers can enjoy some of the benefits of static types
even without converting existing Lua modules to Typed Lua --
they can export a statically typed interface to a dynamically typed module,
and statically typed users of the module can use the Typed Lua compiler
to check their use of the module.
Thus, implementing an optional type system for Lua offers Lua
programmers one way to obtain most of the advantages of static typing
without compromising the simplicity and flexibility of dynamic typing.
We have an implementation of the Typed Lua compiler that is
available online\footnote{https://github.com/andremm/typedlua}.

Typed Lua's intended use is as an application language, and
we believe that policies for organizing a program in modules and writing
object-oriented programs should be part of the language and
checked by its optional type system.
An application language is a programming language that helps
programmers develop applications from scratch until these
applications evolve into complex systems rather than just scripts.
We will show that Typed Lua introduces the refinement of
tables to support the common idioms that Lua programmers use
to encode both modules and objects.

We also believe that Typed Lua helps programmers give more
formal documentation to already existing Lua code, as static types
are also a useful source of documentation in languages that provide
type annotations, because type annotations are always validated by
the type checker and therefore never get outdated.
Thus, programmers can use Typed Lua to define axioms about the
interfaces and types of dynamically typed modules.
We enforce this point by using Typed Lua to statically type
the interface of the Lua standard library and other commonly used
Lua libraries, so our compiler can check Typed Lua code that uses
these libraries.

Typed Lua performs a very limited form of local type inference
\cite{pierce2000lti}, as static typing does not necessarily mean
that programmers need to insert type annotations in the code.
Several statically typed languages such as Haskell provide some
amount of type inference that automatically deduces the types of
expressions.
Still, Typed Lua only requires a small amount of type annotations
due to the nature of its optional type system.

Typed Lua does not deal with code optimization, although another
important advantage of static types is that they help the compiler
perform optimizations and generate more efficient code.
However, we believe that the formalization of our optional type
system is precise enough to aid optimization in some Lua implementations.

We use some of the ideas of gradual typing to formalize Typed Lua.
Even though Typed Lua is an optional type system and thus does not
include run-time checks between dynamic and static regions of the
code, we believe that using the foundations of gradual typing to
formalize our optional type system will allow us to include run-time
checks in the future.

Finally, we believe that designing an optional type system for Lua may
shed some light on optional type systems for scripting languages
in general, as Lua is a small scripting language that shares
some features with other scripting languages such as JavaScript.

This work is split into seven chapters.
In Chapter \ref{chap:review} we review the literature about blending
static and dynamic typing in the same language, we discuss the differences
between optional type systems and gradual typing, and we also
present the results of our survey on Lua idioms.
In Chapter \ref{chap:typedlua} we use code examples to present the
design of Typed Lua.
In Chapter \ref{chap:system} we present our type system.
In Chapter \ref{chap:evaluation} we discuss the evaluation
results that we obtained while using Typed Lua to type existing Lua code.
In Chapter \ref{chap:related} we present some related work.
In Chapter \ref{chap:conc} we outline our contributions.

